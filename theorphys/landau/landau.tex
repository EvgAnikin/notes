\documentclass{article}
\usepackage{amsmath}
\usepackage{graphicx}
\usepackage[utf8]{inputenc}
\usepackage[T1, T2A]{fontenc}
\usepackage[english,russian]{babel}

\newcommand{\bra}{\langle}
\newcommand{\ket}{\rangle}

\title{Теория Ландау}
\author{Anikin Evgeny, 121}

\begin{document}
\maketitle
\section{}
Для начала рассмотрим цилиндр в магнитном поле, направленном вдоль его оси.
Тогда легко показать, что работа внешних сил (например, батарейки) равна
	\begin{equation}
		\delta A = -\frac{c}{4\pi} HdB
	\end{equation}
Тогда имеем неравенство:
	\begin{equation}
		TdS \le dU - \frac{c}{4\pi} HdB 
	\end{equation}
Отсюда следует, что
	\begin{equation}
		dF \le \frac{c}{4\pi} H dB - S dT
	\end{equation}
Термодинамическое равновесие отвечает минимуму свободной энергии при заданном магнитном поле
$B$.
Введём другой потенциал:
	\begin{equation}
		\Phi = F - \frac{c}{4\pi} BH + \frac{cH^2}{8\pi}
	\end{equation}
Тогда
	\begin{equation}
		d \Phi \le -\frac{c}{4\pi} MdH - SdT
	\end{equation}
Равновесие отвечает минимуму этого потенциала при заданном $H$. Это более физично, так как
именно $H$ определяется внешними условиями, например, током катушки.

\subsection{Пример}
Догадаемся, какой потенциал $\Phi$ приведёт к закону Кюри--Вейсса. Это
\begin{equation}
	\Phi = aTM^2 - MH
\end{equation}
Его минимум достигается при $M = \frac{H}{2aT}$. Отсюда получится
\begin{equation}
	B = H + \frac{4\pi}{c}M = \left(1 + \frac{2\pi}{act}\right)H = \mu H
\end{equation}
А свободная энергия $F$ вообще получается
\begin{equation}
	F = \frac{c\mu H^2}{8\pi}
\end{equation}
(может быть, это что-то значит?)
	
Действительно! Свободная энергия --- это изотермическая работа. Если
считать, что магнитная проницаемость зависит только от температуры, то так 
и получается.
\end{document}
