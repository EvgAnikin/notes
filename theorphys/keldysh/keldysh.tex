\documentclass{article}
\usepackage{amsmath}
\usepackage{graphicx}
\usepackage[utf8]{inputenc}
\usepackage[T1, T2A]{fontenc}
\usepackage[english,russian]{babel}
\usepackage{feynmf}

\newcommand{\p}{\partial}

\title{Техника Келдыша}
\author{Anikin Evgeny, 121}

\begin{document}
\maketitle
\section{Всякая всячина}
Утверждение, которое вызвало у меня вопросы, следующее:
\begin{equation}
	\Sigma^{++} + \Sigma^{--} + \Sigma^{+-} + \Sigma^{-+} = 0
\end{equation}
Здесь $\Sigma$ --- это собственно--энергетическая часть. Для определённости
будем считать, что рассматривается система из фермионов с парным 
взаимодействием.

Для функций Грина аналогичное утверждение имеет вид
\begin{equation}
	G^{++} + G^{--} = G^{+-} + G^{-+} 
\end{equation}
Доказательство следует из следующего факта:
\begin{equation}
	G^{--}(t,t') = G^{+-}(t,t') \quad \mbox{при } t > t'
\end{equation}
\begin{equation}
	G^{++}(t,t') = G^{-+}(t,t') \quad \mbox{при } t < t'
\end{equation}
На самом деле такие же утверждения можно сделать и для каждой диаграммы,
входящей в функцию Грина или собственно--энергетическую часть.

Для определённости рассмотрим какую--нибудь диаграмму:

\begin{figure}[h]	
	\centering
	\begin{fmffile}{diagram}
		\begin{fmfgraph*}(80,50)
			\fmfright{i,if}
			\fmfleft{o,of}

			\fmf{plain,tension=10}{i,a}
			\fmf{electron,tension=4}{a,b}
			\fmf{electron,tension=4}{b,c}
			\fmf{plain,tension=10}{c,o}

			\fmffreeze
			\fmf{phantom,tension=1}{if,u}
			\fmf{phantom,tension=0.4}{u,v,w}
			\fmf{phantom,tension=1}{w,of}

			\fmf{photon,tension=0.5}{a,u}
			\fmf{photon,tension=0.5}{v,b}
			\fmf{photon,tension=0.5}{c,w}
			\fmffreeze
			\fmf{electron,right=0.4}{u,w}
			\fmf{electron,right=0.2}{w,v}
			\fmf{electron,right=0.2}{v,u}

			\fmflabel{$t'-$}{a}
			\fmflabel{$t-$}{c}
			\fmflabel{$t_1+$}{b}
			
		\end{fmfgraph*}
	\end{fmffile}
	\caption{Вклад в собственно--энергетическую часть}
\end{figure}
Рассмотрим выражение, соответствующее такой диаграмме, с фиксированными
внутренними временами. Выберем максимальное из всех времён $t$, $t'$, $t_i$.
\paragraph{Случай 1}
Максимальное время --- $t_k$, то есть внутреннее время. Это значит, что 
если мы изменим знак при этой вершине, то поменяет знак всё выражение.
Так как по знакам производится суммирование, эти выражения сократят друг
друга. Значит, обозначенная область интегрирования по временам не вносит
вклада в собственно--энергетическую часть.
\paragraph{Случай 2}
Максимальное время --- $t$ или $t'$. Тогда при этом времени можно сменить
знак. 

В результате получается ровно то, что требовалось:
\begin{equation}
	\Sigma^{--}(t,t') = -\Sigma^{+-}(t,t') \quad \mbox{при } t > t'
\end{equation}
\begin{equation}
	\Sigma^{++}(t,t') = -\Sigma^{-+}(t,t') \quad \mbox{при } t < t'
\end{equation}

\end{document}
