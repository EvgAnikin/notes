\documentclass{article}
\usepackage{amsmath}
\usepackage{graphicx}
\usepackage[utf8]{inputenc}
\usepackage[T1, T2A]{fontenc}
\usepackage[english,russian]{babel}

\newcommand{\p}{\partial}

\title{Адиабатический инвариант}
\author{Anikin Evgeny, 128}

\begin{document}
\maketitle
Пусть гамильтонова система описывается одной координатой $q$. Докажем, что величина
\begin{equation}
	I = \frac{1}{2\pi} \oint p\,dq
\end{equation}
не меняется при медленном изменении гамильтониана во времени. 

\begin{equation}
	\frac{dI}{dt} = \frac{\partial I}{\partial E} \dot{E} + \frac{\partial I}{\partial t};
\end{equation}
\begin{equation}
	\frac{dE}{dt} = \frac{\partial H}{\partial t},
\end{equation}
\begin{equation}
	\frac{\partial I}{\partial E} = \oint \frac{\partial p}{\partial E} \,dq
\end{equation}
В последнем равенстве частная производная берётся при постояной координате.
\end{document}
