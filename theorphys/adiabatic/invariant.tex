\documentclass{article}
\usepackage{amsmath}
\usepackage{graphicx}
\usepackage[utf8]{inputenc}
\usepackage[T1, T2A]{fontenc}
\usepackage[english,russian]{babel}

\newcommand{\p}{\partial}

\title{Адиабатический инвариант и переменные действие--угол}
\author{Anikin Evgeny, 128}

\begin{document}
\maketitle
\section{Адиабатический инвариант}
Пусть гамильтонова система описывается одной координатой $q$. Докажем, что величина
\begin{equation}
	I = \frac{1}{2\pi} \oint p\,dq
\end{equation}
не меняется при медленном изменении гамильтониана во времени. Пусть гамильтониан 
зависит от параметра $\lambda$, который, в свою очередь, зависит от $t$. 

\begin{equation}
	\frac{dI}{dt} = \frac{\partial I}{\partial E} \dot{E} + 
        \frac{\partial I}{\partial \lambda} \dot{\lambda}
\end{equation}
\begin{equation}
	\frac{dE}{dt} = \frac{\partial H}{\partial t},
\end{equation}
\begin{equation}
	\frac{\partial I}{\partial E} = \oint \frac{\partial p}{\partial E} \,dq = 
        \oint \frac{dq}{\frac{\partial H}{\partial p}} = T,
\end{equation}
\begin{equation}
	\frac{\partial I}{\partial \lambda} = \oint \frac{\partial p}{\partial \lambda} \,dq = 
        -\oint \frac{\frac{\partial{H}}{\partial{\lambda}}}{\frac{\partial H}{\partial p}}\,dq
\end{equation}
Таким образом,
\begin{equation}
    \label{dIdt}
	\frac{dI}{dt} = T\frac{\partial H}{\partial \lambda} \dot{\lambda} 
        -\dot{\lambda}
        \oint \frac{\frac{\partial{H}}{\partial{\lambda}}}{\frac{\partial H}{\partial p}}\,dq
\end{equation}
Усредним теперь $\frac{dI}{dt}$ по периоду. Тогда первое слагаемое в \eqref{dIdt} сократится
со вторым, что доказывает, что $I$ -- адиабаический инвариант.
\section{Переменные действие--угол}
Сделаем каноническое преобразование так, чтобы $I$ стало координатой. 

\end{document}
