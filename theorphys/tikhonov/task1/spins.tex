\documentclass{article}
\usepackage{amsmath}
\usepackage{graphicx}
\usepackage[utf8]{inputenc}
\usepackage[T1]{fontenc}
\usepackage[english]{babel}

\newcommand{\bra}{\langle}
\newcommand{\ket}{\rangle}

\title{Two spins in magnetic field}
\author{Anikin Evgeny}

\begin{document}
\maketitle
\section{Constant magnetic field}
Two spins in magnetic field with exchange interaction are described by a Hamiltonian
\begin{equation}
    \hat{H} = J\vec{S}_1\cdot\vec{S}_2 + \vec{B}\cdot(\vec{S}_1+\vec{S}_2)
\end{equation}
First, let us find the eigenvalues and eigenvectors at constant magnetic field. Assume that
$\vec{B} = (0, 0, B_z)$. Then the Hamiltonian becomes
\begin{equation}
    \hat{H} = \frac{J}{2}( S^2 - S_1^2 - S_2^2) + 
                B S_z,
\end{equation}
where $\vec{S} = \vec{S}_1 + \vec{S}_2$.
It is a standard task to diagonalize the operator of total angular momentum. Here it's obvious
that the total spin $s$ can be either $1$ or $0$.

At $s = 1$:
\begin{equation}
    E_{1,m} = \frac{J}{4} + Bm, \, m \in \{1,0,-1\}
\end{equation}
\begin{equation}
    \begin{gathered}
        |1,1\rangle = |\uparrow\rangle |\uparrow \rangle\\
        |1,0\rangle = \frac{1}{\sqrt{2}}(|\uparrow\rangle |\downarrow \rangle + 
                                        |\downarrow\rangle |\uparrow \rangle)\\
        |1,-1\rangle = |\downarrow\rangle |\downarrow \rangle\\
    \end{gathered}
\end{equation}
At $s = 0$:
\begin{equation}
    E_{1,m} = -\frac{3J}{4}
\end{equation}
\begin{equation}
        |0,0\rangle = \frac{1}{\sqrt{2}}(|\uparrow\rangle |\downarrow \rangle - 
                                        |\downarrow\rangle |\uparrow \rangle)\\
\end{equation}
\section{Berry phase in rotating magnetic field}
Now let us assume that the magnetic field slowly rotates around the $z$ axis: 
$\vec{B} = (B\sin\theta \cos{\Omega t}, B\sin\theta \sin{\Omega t}, B\cos\theta)$. After
a $2\pi$ rotation the wavefunction generally acquires a phase. To compute this phase,
it is possible to transform the wavefunction as follows:
\begin{equation}
   |\psi \rangle = e^{-i\hat{s}_z \Omega t} |\psi' \rangle
\end{equation}
After that, the dependence on time in Schrodinger equation cancels, but the additional term
arises:
\begin{equation}
    i\partial_t |\psi'\rangle = \hat{H}_\mathrm{eff}|\psi'\rangle = 
    \hat{H}_0|\psi'\rangle - \Omega\hat{s_z}|\psi'\rangle
\end{equation}
Here $\hat{H}_0 = J\vec{S}_1\cdot\vec{S}_2 + \vec{B}_0\cdot(\vec{S}_1+\vec{S}_2)$, 
$\vec{B}_0 = (B\sin\theta, 0, B\cos\theta)$.

It's obvious that the phase acquired by the wavefunction is 
\begin{equation}
   e^{i\beta} =  e^{-2\pi i(E - E_0)/\Omega},
\end{equation}
where $E$ is
the energy corresponding to $\hat{H}_\mathrm{eff}$, and $E_0$ --- to $\hat{H}_0$.

As $\Omega$ is small, we should treat $\Omega \hat{s_z}$ as a perturbation. In the first order
$E - E_0 = \Omega\langle \hat{s_z} \rangle$, where $s_z$ is averaged over the unperturbed
eigenfunctions. The latter were found in the previous section for $B \parallel Oz$, but 
it is easy to modify them for arbitrary $\vec{B}$.

The wavefunctions of the $\frac{1}{2}$ spins corresponding to the different projections
on a given axis can be found by diagonalizing the operator $\vec{n} \cdot \vec{s}$:
\begin{equation}
    \vec{n} \cdot \vec{s} = \begin{pmatrix}
                                \cos{\theta} && \sin{\theta} e^{-i\phi} \\
                                \sin{\theta} e^{i\phi} && \cos{\theta}
                            \end{pmatrix}
\end{equation}
The answer is below:
\begin{equation}
    \begin{gathered}
        u_\uparrow = \begin{pmatrix}
                        \cos{\frac{\theta}{2}} \\
                        \sin{\frac{\theta}{2}e^{i\phi}}
                    \end{pmatrix}\\
        u_\downarrow = \begin{pmatrix}
                        -\sin{\frac{\theta}{2}e^{-i\phi}}\\
                        \cos{\frac{\theta}{2}} \\
                    \end{pmatrix}
    \end{gathered}
\end{equation}

Now it is straightforward to compute the phase of any state after a rotation. The system 
should be initially in the ground state, so let us consider the possible scenarios for
ground states. Without loss of generality, $B > 0$ and $\Omega > 0$.

If $J > B$, the ground state will be $|0,0\rangle$. As it is a scalar, 
$\langle 0,0|s_z|0,0\rangle = 0$. Therefore, {\bf the phase is $0$}.

Otherwise, the ground state will be 
\begin{equation}
    |1, -1\rangle = 
        u_\downarrow\otimes u_\downarrow
\end{equation}
A simple calculation is below:
\begin{equation}
    \langle 1, -1 | s_z | 1, -1\rangle = 
    \langle 1, -1 | (s_{1z} + s_{2z}) | 1, -1\rangle = 
    2\langle u_\downarrow | s_z | u_\downarrow \rangle = -\cos{\theta}
\end{equation}
So,
\begin{equation}
    \beta = 2\pi \langle s_z \rangle = -2\pi \cos{\theta}
\end{equation}
As the phase is defined modulo $2\pi$,
\begin{equation}
    \beta = 4\pi \sin^2{\frac{\theta}{2}}
\end{equation}
\end{document}
