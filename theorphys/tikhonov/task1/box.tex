\documentclass{article}
\usepackage{amsmath}
\usepackage{graphicx}
\usepackage[utf8]{inputenc}
\usepackage[T1]{fontenc}
\usepackage[english]{babel}

\newcommand{\bra}{\langle}
\newcommand{\ket}{\rangle}

\title{Degenerate levels in a rotating box}
\author{Anikin Evgeny}

\begin{document}
\maketitle
In this problem we are interested in the two first excited levels of a square potential
well. Let us take the potential in the form  
\begin{equation}
    V(x,y) = \left\{\begin{matrix}
                \infty && \text{if} \, |x| > \frac{a}{2} \, \text{or} |y| > \frac{a}{2}\\
                0 && \text{otherwise}
             \end{matrix}\right.
\end{equation}

The normalized wavefunctions of these states are
\begin{equation}
    \begin{gathered}
        \psi_{1} = \frac{2}{a} \cos{\frac{\pi x}{a}} \sin{\frac{2\pi y}{a}}\\
        \psi_{2} = \frac{2}{a} \sin{\frac{2\pi x}{a}}\cos{\frac{\pi y}{a}} \\
    \end{gathered}
\end{equation}

Now let us apply the concept of Berry phase to degenerate levels. Assume that the potential
well is slowly rotating around the $z$ axis. That means that the eigenfunctions also 
become angle--dependent. 

Let us search for a solution of the Schrodinger equation in the form
\begin{equation}
    \psi(t) = (\alpha(t)\psi_1^\theta + \beta(t) \psi_2^\theta) e^{-iEt} 
\end{equation}
Here $\theta$ is the rotation angle and $E$ is the energy of the states. After substituting
this $\psi$ into Schrodinger equation, we get
\begin{equation}
    \dot{\alpha} \psi_1 + \dot{\beta} \psi_2 + 
       \alpha \partial_\theta \psi_1\dot{\theta} + 
       \beta  \partial_\theta \psi_2\dot{\theta} = 0
\end{equation}
Taking the scalar product with $\psi_1^\theta$, $\psi_2^\theta$ we get
\begin{equation}
   i\frac{d}{d\theta} 
                          \begin{pmatrix}
                            \alpha \\
                            \beta
                          \end{pmatrix}
                          = -i\begin{pmatrix}
                            \langle \psi_1 \partial_\theta \psi_1 \rangle&&
                            \langle \psi_1 \partial_\theta \psi_2 \rangle\\
                            \langle \psi_2 \partial_\theta \psi_1 \rangle&&
                            \langle \psi_2 \partial_\theta \psi_2 \rangle\\
                          \end{pmatrix}
                          \begin{pmatrix}
                            \alpha \\
                            \beta
                          \end{pmatrix}
\end{equation}
The matrix elements can be easily calculated after the following observation:
\begin{equation}
    \psi^\theta = e^{-i\hat{l}_z \theta}\psi^0
\end{equation}
and
\begin{equation}
    \partial_\theta \psi= -i\hat{l}_z \psi
\end{equation}
So, 
\begin{equation}
   i\frac{d}{d\theta}
                          \begin{pmatrix}
                            \alpha \\
                            \beta
                          \end{pmatrix}
                         = - \begin{pmatrix}
                            0 &&
                            \langle \psi_1 \hat{l}_z \psi_2\rangle\\
                            \langle \psi_2 \hat{l}_z \psi_1\rangle &&
                            0
                          \end{pmatrix}
                          \begin{pmatrix}
                            \alpha \\
                            \beta
                          \end{pmatrix}
\end{equation}
The diagonal elements vanish because of the mirror symmetry.

The last step is to compute the matrix element of $l_z$. This is quite straightforward:
\begin{multline}
    \langle \psi_1 \hat{l}_z \psi_2 \rangle = 
     -\frac{4i}{a^2} \int dx\,dy \cos{\frac{\pi x}{a}} \sin{\frac{2\pi y}{a}}
                                 (x\partial_y - y\partial_x)
                                 \sin{\frac{2\pi x}{a}}\cos{\frac{\pi y}{a}}\\
    = \frac{8\pi i}{a^3} \int dx\, x \cos{\frac{\pi x}{a}}\sin{\frac{2\pi x}{a}} 
                         \int dy \sin{\frac{\pi y}{a}} 
                                 \sin{\frac{2\pi y}{a}} =\\
    =  \frac{8i}{\pi^2} \int_{-\frac{\pi}{2}}^{\frac{\pi}{2}} dx\, x \cos{x}\sin{2x} 
                        \int_{-\frac{\pi}{2}}^{\frac{\pi}{2}} dy \sin{y} 
                                 \sin{2y} = \frac{256i}{27\pi^2}
\end{multline}
Finally,
\begin{equation}
   i\frac{d}{d\theta}
         \begin{pmatrix}
           \alpha \\
           \beta
         \end{pmatrix}
        =\frac{256}{27\pi^2} \sigma_y
         \begin{pmatrix}
           \alpha \\
           \beta
         \end{pmatrix}
\end{equation}
\begin{equation}
        \begin{pmatrix}
          \alpha \\
          \beta
        \end{pmatrix} = 
        \exp\left\{-\frac{256i}{27\pi^2} \sigma_y \theta\right\}
        \begin{pmatrix}
          \alpha_0 \\
          \beta_0
        \end{pmatrix} = 
        \begin{pmatrix}
            \cos{\frac{256}{27\pi^2}\theta} &&
            -\sin{\frac{256}{27\pi^2}\theta} \\
            \sin{\frac{256}{27\pi^2}\theta} &&
            \cos{\frac{256}{27\pi^2}\theta}
        \end{pmatrix}
        \begin{pmatrix}
          \alpha_0 \\
          \beta_0
        \end{pmatrix}
\end{equation}
\end{document}
