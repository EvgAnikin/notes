\documentclass{article}
\usepackage{amsmath}
\usepackage{graphicx}
\usepackage[utf8]{inputenc}
\usepackage[T1]{fontenc}
\usepackage[english]{babel}

\newcommand{\bra}{\langle}
\newcommand{\ket}{\rangle}

\title{Confined particle on a ring}
\author{Anikin Evgeny}

\begin{document}
\maketitle
Let us consider the particle on a ring with delta potential in presence of magnetic flux:
\begin{equation}
    \hat{H} = \frac{1}{2m}\left(\hat{p} - \frac{eA}{c}\right)^2 
            - \frac{\kappa_0}{m}\delta(x - x_0),
    \, x + L = x
\end{equation}
We are interested in the Berry phase acquired by a ground state 
after a slow rotation of $x_0$ around the ring.

Assume that we know the wavefunction of the ground state. The Berry phase is given
by the expression
\begin{equation}
    \phi = \int \mathcal{A} dx_0 = L\mathcal{A}
\end{equation}
The Berry connection is 
\begin{equation}
    \mathcal{A} = i\langle \psi | \partial_{x_0} | \psi \rangle =
            \langle \hat{p} \rangle
\end{equation}
As $\hat{p} = m\hat{v} + \frac{eA}{c}$,
\begin{equation}
    \mathcal{A} = m\langle \psi | \hat{v} | \psi \rangle 
             +\frac{eA}{c}
\end{equation}
Furthermore, we can notice that 
\begin{equation}
    \hat{v} = -\frac{c}{e}\frac{\partial \hat{H}}{\partial A}
\end{equation}
Therefore, by Ehrenfest theorem,
\begin{equation}
    \langle \hat{v} \rangle = -\frac{c}{e} \frac{\partial E}{\partial A},
\end{equation}
where $E$ is the energy of the ground state.
Thus,
\begin{equation}
    \mathcal{A} = -\frac{mc}{e}\frac{\partial E}{\partial A}
             +\frac{eA}{c}
\end{equation}
Before any computations it is obvious that in the limit $\kappa L \to \infty$ the mean 
velocity is zero, or equivalently, the energy does not depend on $A$. Therefore, in this limit
\begin{equation}
    \phi = \frac{eLA}{c} = \frac{e\Phi}{c}
\end{equation}

Now let us perform the computation of the ground state energy. For convenience we set 
$x_0 = 0$. Therefore, in the region 
$0 < x < L$ the wavefunction takes the form
\begin{equation}
    \psi = \alpha\exp{\left(\kappa x +  \frac{ieAx}{c}\right)} + 
           \beta\exp{\left(-\kappa x + \frac{ieAx}{c}\right)},
\end{equation}
where $E = -\frac{\kappa^2}{2m}$. The boundary conditions read
\begin{equation}
    \begin{gathered}
        \psi(0) = \psi(L)\\
        \left. \left( \frac{\partial}{\partial x}-\frac{ieA}{c}\right)\psi\right|_{L-0}^{+0}=
        -2\kappa_0 \psi(0)
    \end{gathered}
\end{equation}
After some calculations, we obtain the equation which determines the eigenvalues:
\begin{equation}
   \operatorname{det}
        \begin{pmatrix}
            \exp{\left( \kappa L + \frac{ieAL}{c}\right)} - 1 &&
            \exp{\left(-\kappa L + \frac{ieAL}{c}\right)} - 1 \\
            \exp{\left( \kappa L + \frac{ieAL}{c}\right)} - 1 - \frac{2\kappa_0}{\kappa} &&
           -\exp{\left(-\kappa L + \frac{ieAL}{c}\right)} + 1 - \frac{2\kappa_0}{\kappa} &&
        \end{pmatrix} = 0
\end{equation}
The latter can be simplified:
\begin{equation}
    \cosh{\kappa L} - \frac{\kappa_0}{\kappa} \sinh{\kappa L} = \cos{\frac{eAL}{c}}
\end{equation}
So,
\begin{equation}
   \frac{d\kappa}{dA} = \frac{e\kappa}{c\kappa_0} \cdot
                        \frac{\sin\frac{eAL}{c}}
                                        {\cosh{\kappa L} - \left(\frac{\kappa}{\kappa_0} +
                                                 \frac{1}{\kappa L}\right)\sinh{\kappa L}}
\end{equation}
The Berry phase is
\begin{equation}
    \phi = \frac{e\Phi}{c} + \frac{\kappa^2 L}{\kappa_0} \frac{\sin\frac{e\Phi}{c}}
                {\cosh{\kappa L} - \left(\frac{\kappa}{\kappa_0} +
                \frac{1}{\kappa L}\right)\sinh{\kappa L}} 
\end{equation}
At large $\kappa_0 L$
\begin{equation}
    \phi \approx \frac{e\Phi}{c} - \frac{\kappa_0 L \sin\frac{e\Phi}{c}}{\sinh{\kappa_0 L}} 
\end{equation}
\end{document}
