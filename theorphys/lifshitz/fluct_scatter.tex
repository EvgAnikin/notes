\documentclass{article}
\usepackage{amsmath}
\usepackage{graphicx}
\usepackage[top=2cm, bottom=2cm]{geometry}
\usepackage[utf8]{inputenc}
\usepackage[T1, T2A]{fontenc}
\usepackage[english,russian]{babel}

\newcommand{\bra}{\langle}
\newcommand{\ket}{\rangle}

\newtheorem{theorem}{Теорема}

\title{Рассеяние на флуктуациях в модели Лифшица}
\author{Аникин Евгений}

\begin{document}
\maketitle
    Гамильтониан модели Лифшица задаётся так:
    \begin{equation}
        H_{ij} = \left\{\begin{matrix}
                    t e^{-\frac{|r_i - r_j|}{a}} \quad \text{при} \quad i\ne j \\
                    0 \quad \text{при} \quad i = j
                 \end{matrix}\right.
    \end{equation}
    Предположим, среднее расстояние между узлами мало, то есть $na^3 \gg 1$. Тогда можно
    использовать континуальное приближение, и уравнение Шрёдингера запишется так:
    \begin{equation}
        t\int e^{-\frac{|r - r'|}{a}} n(r') \Psi(r') d^3 r' = E\Psi(r)
    \end{equation}
    Здесь $n(r')$ 
    можно заменить на $n_0 + \delta n$, где $n_0$ --- среднее значение плотности, а 
    $\delta n$ --- флуктуации. В пренебрежении флуктуациями уравнение Шрёдингера тривиально
    решается в Фурье--представлении:
    \begin{equation}
        \epsilon(k) = t \frac{8\pi n_0a^3}{(1 + (ka)^2)^2}
    \end{equation}
    Теперь изучим влияние флуктуаций. Оператор взаимодействия
    с флуктуациями записывается в виде
    \begin{equation}
        V(r,r') = te^{-\frac{|r - r'|}{a}} \delta n(r')
    \end{equation}
    В крестовой диаграммной технике главный вклад в собственно--энергетическую часть имеет вид
    \begin{multline}
         \Sigma(r,r') = \left\langle \int d^3 r_1 d^3 r_2 V(r,r_1)G_0(r_1-r_2) V(r_2,r') 
                        \right\rangle = \\
                        = t^2\int d^3 r_1 d^3 r_2 e^{-\frac{|r - r_1| + |r_2 - r'|}{a}}
                        \langle \delta n(r_1) \delta n(r_2) \rangle G_0(r_1-r_2) 
    \end{multline}
    Здесь $G_0(r)$ --- функция Грина. При усреднении можно считать, что 
    $\langle \delta n(r_1) \delta n(r_2) \rangle = n_0 \delta(r_1 - r_2)$. Тогда 
    \begin{equation}
        \Sigma(r,r')= n_0t^2G_0(0)\int d^3 r'' e^{-\frac{|r - r''| + |r'' - r'|}{a}} =
                       t^2G_0(0) \int \frac{d^3 k}{(2\pi)^3} 
                            \left(\frac{8\pi a^3 }{(1 + (ka)^2)^2}\right)^2 e^{i\vec{k}\vec{r}}
    \end{equation}
    Таким образом,
    \begin{equation}
        \Sigma(k) = n_0t^2G_0(0) 
                \left(\frac{8\pi a^3 }{(1 + (ka)^2)^2}\right)^2
    \end{equation}
    Время жизни квазичастицы определяется соотношением
    \begin{equation}
        \frac{1}{2\tau} = -\operatorname{Im} \Sigma(k)
    \end{equation}
    Найдём мнимую часть $G_0(0)$:
    \begin{equation}
        \operatorname{Im} G_0(0) = \operatorname{Im} \int \frac{d^3 k}{(2\pi)^3}
                        \frac{1}{\omega - \epsilon(k) + i\delta} = 
                        -\pi \int \frac{d^3 k}{(2\pi)^3} \delta(\omega - \epsilon(k)) = 
                        -\frac{k^2}{2\pi v}
    \end{equation}
    В этом выражении $v = \frac{d\epsilon}{dk}$ --- скорость. Используя выражение для 
    скорости
    \begin{equation}
        v = -\frac{32\pi n_0 a^5 t}{(1 + (ka)^2)^3} k,
    \end{equation} 
    можно выписать ответ для $\tau^{-1}$:
    \begin{equation}
        \tau^{-1} = \frac{n_0 t^2 k^2}{\pi v} \left(\frac{8\pi a^3 }{(1 + (ka)^2)^2}\right)^2 =
         t \frac{2ka}{1 + (ka)^2}
    \end{equation}
    Длина свободного пробега --- 
    \begin{equation}
        l = v\tau = \frac{16\pi a \cdot n_0 a^3}{(1 + (ka)^2)^2}
    \end{equation}
    Как и следовало ожидать, длина свободного пробега  велика при большом $n_0 a^3$. Это 
    оправдывает то, что мы ограничились первым порядком теории возмущений.
\end{document}
