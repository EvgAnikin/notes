\documentclass{article}
\usepackage{amsmath}
\usepackage{graphicx}
\usepackage[top=2cm, bottom=2cm]{geometry}
\usepackage[utf8]{inputenc}
\usepackage[T1, T2A]{fontenc}
\usepackage[english,russian]{babel}

\newcommand{\bra}{\langle}
\newcommand{\ket}{\rangle}

\newtheorem{theorem}{Теорема}

\title{Ограниченность энергии снизу в модели Лифшица}
\author{Аникин Евгений}

\begin{document}
\maketitle
    Гамильтониан модели Лифшица задаётся так:
    \begin{equation}
        H_{ij} = \left\{\begin{matrix}
                    t e^{\frac{|r_i - r_j|}{a}} \quad \text{при} \quad i\ne j \\
                    0 \quad \text{при} \quad i = j
                 \end{matrix}\right.
    \end{equation}
    Хочется доказать, что его спектр ограничен cнизу значением $-t$. 
    Это равносильно положительной 
    определённости (в нестрогом смысле) матрицы
    \begin{equation}
        h_{ij} = e^{\frac{|r_i - r_j|}{a}}
    \end{equation}
    Здесь уже диагональные элементы равны единице, а не нулю. 
    Положительная определённость означает,
    что для любого набора $y_i$ 
    \begin{equation}
        \sum h_{ij} y_i^* y_j = \sum e^{\frac{|r_i - r_j|}{a}} y_i^* y_j \ge 0
    \end{equation}
    Воспользуемся разложением экспоненты в трёхмерном пространстве в интеграл Фурье:
    \begin{equation}
        e^{-\frac{|\vec{r}|}{a}} = \int \frac{d^3 k}{(2\pi)^3} 
                            \frac{8\pi a^3 e^{i\vec{k}\vec{r}}}{(1 + (ka)^2)^2}
    \end{equation} 
    Это позволяет переписать $\sum h_{ij} y_i^* y_j$ в виде интеграла по $k$.
    \begin{equation}
        \sum h_{ij} y_i^* y_j = \int \frac{d^3 k}{(2\pi)^3} 
                           \frac{8\pi a^3}{(1 + (ka)^2)^2}
                            \sum e^{i\vec{k}(\vec{r}_i - \vec{r}_j)} y_i^* y_j
    \end{equation}
    Нетрудно видеть, что сумма в правой части --- это квадрат модуля некоторого выражения.
    Получается
    \begin{equation}
        \sum h_{ij} y^i y^j = \int \frac{d^3 k}{(2\pi)^3} 
                            \frac{8\pi a^3}{(1 + (ka)^2)^2}
                            \left| \sum e^{-i\vec{k} \vec{r}_i} y_i \right|^2
    \end{equation}
    Интеграл в правой части неотрицателен, поэтому $\sum h_{ij} y_i^* y_j$ тоже $\ge 0$  
    . Это доказывает, что энергия в модели Лифшица всегда $\ge -t$. Более того, энергия 
    может достигать $-t$ только в случае, когда все узлы расположены в одной точке. Иначе 
    сумма экспонент обязательно будет где-то отлична от нуля, и интеграл будет строго
    положителен.
    
    Заметим также, что рассуждение использует только тот факт, что интеграл Фурье от
    $\exp{(-r/a)}$ положителен. Значит, вместо экспонент можно использовать любые другие 
    функции, обладающие этим свойством, например, $\exp{(-r^2/a^2)}$.
\end{document}
