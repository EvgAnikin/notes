\documentclass{article}
\usepackage{amsmath}
\usepackage{graphicx}
\usepackage[utf8]{inputenc}
\usepackage[T1, T2A]{fontenc}
\usepackage[english,russian]{babel}

\newcommand{\bra}{\langle}
\newcommand{\ket}{\rangle}

\newtheorem{theorem}{Теорема}

\title{Ограниченность энергии снизу в модели Лифшица}
\author{Anikin Evgeny, 121}

\begin{document}
\maketitle
    Гамильтониан модели Лифшица задаётся так:
    \begin{equation}
        H_{ij} = \left\{\begin{matrix}
                    t e^{\frac{|r_i - r_j|}{a}} \quad \text{при} \quad i\ne j \\
                    0 \quad \text{при} \quad i = j
                 \end{matrix}\right.
    \end{equation}
    Хочется доказать, что его спектр ограничен cнизу значением $-t$. Это равносильно положительной 
    определённости (в нестрогом смысле) матрицы
    \begin{equation}
        h_{ij} = e^{\frac{|r_i - r_j|}{a}}
    \end{equation}
    Здесь уже диагональные элементы равны единице, а не нулю. Положительная определённость означает,
    что для любого набора $x_i$ 
    \begin{equation}
        \sum e^{\frac{|r_i - r_j|}{a}} x_i x_j \ge 0
    \end{equation}
    Докажем это по индукции. База очевидна, так как для одного узла энергия равна $t$. Переход
    индукции проведём...
\end{document}
