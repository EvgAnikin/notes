\documentclass{article}
\usepackage{amsmath}
\usepackage{graphicx}
\usepackage[utf8]{inputenc}
\usepackage[T1, T2A]{fontenc}
\usepackage[english,russian]{babel}

\newcommand{\bra}{\langle}
\newcommand{\ket}{\rangle}
\DeclareMathOperator{\ln}{ln}

\title{Минимум свободной энергии}
\author{Anikin Evgeny, 128}

\begin{document}
\maketitle
По определению,
$$
	F = U - TS,
$$
$$
	U = \sum p_n E_n,
$$
$$
	S = -\sum p_n \ln{p_n},
$$
где суммирование ведётся по всем возможным состояниям системы (например, по энергетическим
уровням). Внутренняя энергия, энтропия и свободная энергия определены для произвольного
распределения $\{p_n\}$. 

Поставим задачу минимизировать $F$ по всем возможным распределениям при фиксированном $T$.
Дополнительное техническое условие заключается ещё в том, что $\sum p_n = 1$. Воспользуемся
методом множителей Лагранжа:
$$
	\frac{\partial}{\partial p_n} \sum p_k (E_k + T \ln{p_k}) + \lambda \sum p_k = 0
$$
\end{document}
