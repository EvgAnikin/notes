\documentclass{article}
\usepackage{amsmath}
\usepackage{graphicx}
\usepackage[utf8]{inputenc}
\usepackage[T1, T2A]{fontenc}
\usepackage[english,russian]{babel}

\newcommand{\p}{\partial}

\title{Флуктуации сопротивления между контактами конечного размера}

\author{Аникин Евгений}

\begin{document}
\maketitle
Сначала я рассматриваю случайную сетку из сопротивлений, задаваемых формулой
\begin{equation}
    R(\xi) = R_0 e^{\Lambda F(\xi)},
\end{equation}
где $\xi$ --- случайная величина от $0$ до $1$, $F$ --- монотонная функция, а $\Lambda \gg 1$.
Затем повторяю рассуждения для сетки Миллера--Абрахамса.

\section{Точечные контакты}
Для начала рассмотрим флуктуации сопротивления между двумя точечными контактами. Физическая
картина здесь такова: нефлуктуирующая часть определяется критической подсеткой, а флуктуирующая
--- путём до критической подсетки. Чтобы оценить флуктуации, можно рассуждать следующим
образом: сначала разорвём все сопротивления, а затем будем "включать" их по возрастанию до 
появления перколяции. Будем называть \emph{подсеткой} $\xi$ подсетку, где включены
все сопротивления, меньшие $R(\xi)$. 

Можно считать, что флуктуации определяются тем сопротивлением, которое
было включено последним. Оно, очевидно, находится не в критической подсетке, а где--то
недалеко от контактов, в пути до критической подсетки.

Найдём распределение для этого последнего сопротивления. 
Пусть расстояние между контактами 
много больше корреляционной длины. Тогда можно сказать следующее: контакты принадлежат
бесконечному кластеру подсетки $\xi$ тогда и только тогда, когда последнее сопротивление
$< R(\xi)$. Значит, вероятность того, что последнее сопротивление $<R(\xi)$, равна 
\begin{equation}
    P(R_{\mathrm{last}} < R(\xi)) = \rho(\xi)^2,
\end{equation}
где $\rho(\xi) \propto (\xi - \xi_{\mathrm{crit}})^\beta$ --- вероятность того, что контакт 
принадлежит бесконечному кластеру, то есть плотность бесконечного кластера.
Таким образом, функция распределения $\xi$ ---
\begin{equation}
    \frac{dP}{d\xi} \propto (\xi - \xi_{\mathrm{crit}})^{2\beta - 1}
\end{equation}
Для плоской решётки $\beta \approx 0.14$, поэтому функция распределения имеет узкий максимум
при $\xi = \xi_\mathrm{crit}$. Однако из--за степенного хвоста типичные $\xi$ больше
критического, а флуктуация $\xi$ порядка единицы.

\section{Контакты конечного размера}
Попытаемся применить ту же самую логику к случаю контактов размера $b$. Если 
$b < l_\mathrm{corr}$, то справедлива та же самая картина: во флуктуации вносит вклад самое
большое сопротивление. Значит,
\begin{equation}
    P(R_{\mathrm{last}} < R(\xi)) = \rho(\xi)^2,
\end{equation}
где $\rho_b(\xi)$ --- вероятность того, что контакт размера $b$ перекрывается с бесконечным 
кластером. Эта вероятность $\propto b^{2 - d_f}$. С другой стороны, для $b \sim a$, где 
$a$ --- расстояние между узлами, $\rho \propto (\xi - \xi_\mathrm{crit})^\beta$. Значит,
\begin{equation}
    \rho_b(\xi) \propto \left(\frac{b}{a}\right)^{2 - d_f} (\xi - \xi_\mathrm{crit})^\beta
\end{equation}
Эта формула справедлива до тех пор, пока $\rho_b(\xi) < 1$. Значит, можно окончательно написать
\begin{equation}
    \rho_b(\xi) \propto \left\{ 
                 \begin{matrix}   
                 \left(\frac{b}{a}\right)^{2 - d_f} (\xi - \xi_\mathrm{crit})^\beta, & 
                       0 < \xi-\xi_\mathrm{crit} < \left( \frac{a}{b}\right)^{\frac{1}{\nu}}\\
                  1, &  \xi - \xi_\mathrm{crit} > \left( \frac{a}{b}\right)^{\frac{1}{\nu}}
                 \end{matrix} \right.
\end{equation}
Плотность вероятности ---
\begin{equation}
    \frac{dP}{d\xi} \propto \left\{ 
                 \begin{matrix}   
                 \left(\frac{b}{a}\right)^{4 - 2d_f} (\xi - \xi_\mathrm{crit})^{2\beta - 1}, & 
                       0 < \xi-\xi_\mathrm{crit} < \left( \frac{a}{b}\right)^{\frac{1}{\nu}}\\
                  0, &  \xi - \xi_\mathrm{crit} > \left( \frac{a}{b}\right)^{\frac{1}{\nu}}
                 \end{matrix} \right.
\end{equation}
Было использовано, что $2 - d_f = \beta/\nu$.

Таким образом, ширина распределения по $\xi$ --- 
\begin{equation}    
    \Delta \xi \approx \left( \frac{a}{b}\right)^{\frac{1}{\nu}}
\end{equation}
Флуктуация логарифма $R$ --- 
\begin{equation}
    \Delta \log\frac{R}{R_0} \sim \Lambda \frac{dF}{d\xi}\left( \frac{a}{b}\right)^{\frac{1}{\nu}}
\end{equation}

\subsection{Сетка Миллера--Абрахамса}
Рассуждения переносятся на сетку Миллера--Абрахамса практически без изменений. Для сетки 
Миллера--Абрахамса
\begin{equation}
    R_{ij} = R_0 e^{-\frac{r_{ij}}{a}}
\end{equation}
Вероятность того, что контакт размера $b$ перекрывается с бесконечным кластером подсетки,
где разораны сопротивления $> r$ ---
\begin{equation}
    \rho_b(\xi) \propto \left\{ 
                 \begin{matrix}   
                 \left(\frac{b}{a}\right)^{2 - d_f}(n\pi r^2 - B_{\mathrm{cr}})^\beta, & 
                       0 < n\pi r^2 - B_{\mathrm{cr}} < 
                                    \left( \frac{a}{b}\right)^{\frac{1}{\nu}}\\
                  1, & n\pi r^2 - B_{\mathrm{cr}} > \left( \frac{a}{b}\right)^{\frac{1}{\nu}}
                 \end{matrix} \right.
\end{equation}
Отсюда аналогичными рассуждениями получаем
\begin{equation}
    \Delta \log{\frac{R}{R_0}} = \frac{1}{\sqrt{4\pi na^2 B_\mathrm{cr}}}
                                    \left( \frac{a}{b}\right)^{\frac{1}{\nu}}
\end{equation}

\end{document}
