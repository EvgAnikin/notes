\documentclass{article}
\usepackage{amsmath}
\usepackage{graphicx}
\usepackage[utf8]{inputenc}
\usepackage[T1, T2A]{fontenc}
\usepackage[english,russian]{babel}

\title{Излучение движущихся зарядов}
\author{Евгений Аникин, 128}

\begin{document}
\maketitle
\section{Функция Грина волнового уравнения}
Функция Грина $G(x,t)$ по определению равна запаздывающему решению уравнения
\begin{equation}
    \partial_\mu \partial^\mu \phi = \delta(x,t)
\end{equation}
Преобразованием Фурье получим
\begin{equation}
    G(x,t) = \int \frac{d\omega}{(2\pi)^2} \frac{d^3k}{(2\pi)^3} 
            \frac{e^{-i\omega t + ikx}}{\omega^2 - k^2}
\end{equation}
Обход полюсов при интегрировании по $\omega$ нужно производить сверху, чтобы получить
запаздывающую функцию.
В результате получается
\begin{equation}
    G(x,t) = -\frac{1}{2\pi} \delta(t^2 - x^2) \theta(t)
\end{equation}
\end{document}
