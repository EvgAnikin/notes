\documentclass{article}
\usepackage{amsmath}
\usepackage{graphicx}
\usepackage[utf8]{inputenc}
\usepackage[T1, T2A]{fontenc}
\usepackage[english,russian]{babel}
\usepackage[normalem]{ulem}

\newcommand{\bra}{\langle}
\newcommand{\ket}{\rangle}
%\DeclareMathOperator{\ln}{ln}

\title{Минимум свободной энергии}
\author{Anikin Evgeny, 128}

\begin{document}
\maketitle
\section{Статистическая точка зрения}
По определению,
\begin{equation}
	\label{free_energy}
	F = U - TS,
\end{equation}
\begin{equation}
	U = \sum p_n E_n,
\end{equation}
\begin{equation}
	S = -\sum p_n \ln{p_n},
\end{equation}
где суммирование ведётся по всем возможным состояниям системы (например, по энергетическим
уровням). Внутренняя энергия, энтропия и свободная энергия определены для произвольного
распределения $\{p_n\}$. 

Поставим задачу минимизировать $F$ по всем возможным распределениям при фиксированном $T$.
Дополнительное техническое условие заключается ещё в том, что $\sum p_n = 1$. Воспользуемся
методом множителей Лагранжа:
\begin{equation}
	\frac{\partial}{\partial p_n}\left(\sum p_k(E_k+T\ln{p_k})+\lambda\sum p_k\right)=0
\end{equation}
Отсюда и получается распределение Гиббса.
\begin{equation}
	p_n \propto e^{-\frac{E_n}{T}}
\end{equation}
\section{Термодинамическая точка зрения}
Читая этот параграф, нужно всё время иметь в виду предыдущую ситуацию:
предполагается, что система помещена в термостат с температурой $T$ (именно она --- 
``настоящая'' температура). При этом система может и не находиться в равновесии с термостатом. 
Например, у неё может быть ``своя'' температура $T'$, отличная от $T$.
 
Если зафиксирована температура термостата, то
\begin{equation}
	dF = dU - TdS
\end{equation}
Согласно известному термодинамическому неравенству, для системы, помещённой в термостат,
\begin{equation}
	\label{famous_ineq}
	TdS \ge \delta Q = dU + \delta A
\end{equation}
(причём равенство достигается только в равновесии)
Значит,
\begin{equation}
	dF \le \delta A
\end{equation}
Если система заключена в ящик, то $\delta A = 0$. Получается, что $dF \le 0$, а в равновесии
$dF = 0$.

Остаётся доказать известное неравенство \eqref{famous_ineq}. 
Пусть некоторое тело находится в контакте с термостатом, причём 
температура термостата --- $T$, а
тело находится в термодинамически равновесном состоянии с температурой $T'$ (возможно,
отличной от $T$).
Раз тело находится в равновесном состоянии, можно записать 
\begin{equation}
	\label{entropy_definition}
	dS = \frac{\delta Q}{T'}
\end{equation}
Тогда 
\begin{equation}
	TdS = \frac{T}{T'} \delta Q
\end{equation}
Следует ещё раз подчеркнуть, что по смыслу формулы \eqref{famous_ineq} в неё следует
подставить именно $T$, а не $T'$: ``настоящая температура'' --- это температура термостата,
а не ``собственная'' температура тела. Тело может находиться в каком--нибудь крайне 
неравновесном состоянии, вообще не характеризующемся температурой, при этом 
\eqref{famous_ineq} всё равно будет выполняться. Напротив, \eqref{entropy_definition} ---
просто термодинамическое определение энтропии, и здесь предполагается, что тело находится
в равновесном состоянии с температурой $T'$.

Теперь вернёмся \sout{к нашим баранам} к доказательству. Воспользуемся тем фактом,
 что тепло передаётся от горячих тел к холодным. Тогда в случае, если $\frac{T}{T'} > 1$,
$\delta Q > 0$ и неравенство \eqref{famous_ineq} выполнено. В случае, если $\frac{T}{T'} < 1$,
$\delta Q < 0$ и неравенство снова выполнено.
\end{document}
