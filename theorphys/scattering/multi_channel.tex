\documentclass{article}
\usepackage{amsmath}
\usepackage{graphicx}
\usepackage[utf8]{inputenc}
\usepackage[T1, T2A]{fontenc}
\usepackage[english,russian]{babel}

\title{Многоканальное рассеяние и формула Ландауэра}
\author{Anikin Evgeny, 128}

\begin{document}
\maketitle
Пусть имеется волновод постоянного сечения вдоль координаты $z$, $z < 0$. Пусть 
около $z \sim 0$ находится стенка, отражающая электроны. Вдали от стенки
волновые функции можно выбрать в виде
\begin{equation}
    \Psi_m = \frac{e^{ik_mz}}{\sqrt{v_m}}\psi_m + 
             \sum_n r_{mn} \frac{e^{-ik_nz}}{\sqrt{v_n}} \psi_n
\end{equation}
Здесь $m,n$ --- номера каналов, $\psi_m(x,y)$ --- волновые функции поперечного сечения,
$r_{mn}$ --- матрица рассеяния,  Получим условие унитарности
для этой матрицы. 

Пусть изначально на стенку падает цуг длины $L$ в канале $m$. 
Тогда после рассеяния в $n$-ом канале
мы получим цуг длины $\frac{v_n}{v_m}L$. Таким образом,
\begin{equation}
    u_m(z)\psi_m\frac{e^{ik_mz}}{\sqrt{v_m}} \to 
             \sum_n r_{mn} u_n(z) \psi_n \frac{e^{-ik_nz}}{\sqrt{v_n}} 
\end{equation}
Очень важно, что плавные огибающие здесь имеют вид $u_k(z)/\sqrt{v_k}$ и одинаково 
нормированы. Отсюда и и унитарности оператора эволюции сразу получается
унитарность матрицы $\hat{r}$.
\end{document}
