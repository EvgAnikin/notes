\documentclass{article}
\usepackage{amsmath}
\usepackage{graphicx}
\usepackage[utf8]{inputenc}
\usepackage[T1, T2A]{fontenc}
\usepackage[english,russian]{babel}

\title{Теория рассеяния}
\author{Anikin Evgeny, 121}

\begin{document}
\maketitle
\section{$S$--матрица и амплитуда рассеяния}
\begin{multline}
    \Psi(x,t) = \int \frac{dk}{2\pi} \phi(k)\left(e^{ikz} + \frac{f(\theta)}{r}e^{ikr}\right)
                e^{-\frac{i\hbar k^2t}{2m} + ikR} = \\
               = \phi\left(\frac{m(r+R)}{\hbar t}\right) \frac{f(\theta)}{r}
                e^{\frac{im(r+R)^2}{2\hbar t}} \sqrt{\frac{m}{2\pi i\hbar t}}
\end{multline}
Здесь и далее считаем, что $\Re{\sqrt{z}} > 0$.

Далее сделаем преобразование Фурье:
\begin{equation}
   \Psi(\vec{k}) = \int d^3x\, e^{-i\vec{k}\vec{x}} \Psi(x,t) = 
    \int d^3x\, e^{-i\vec{k}\vec{x}}\phi\left(\frac{m(r+R)}{\hbar t}\right) \frac{f(\theta)}{r}
                e^{\frac{im(r+R)^2}{2\hbar t}} \sqrt{\frac{m}{2\pi i\hbar t}}
\end{equation}
Точка стационарной фазы ---
\begin{equation}
    \vec{k} = \frac{m(r+R)}{\hbar t} \vec{n},
\end{equation}
матрица вторых производных ---
\begin{equation}
    \partial_i \partial_j \frac{im(r+R)^2}{2\hbar t} = \frac{im}{\hbar t} 
                \left(\left(1 + \frac{R}{r}\right)\delta_{ij} - \frac{R}{r} n_i n_j\right)
\end{equation}
С помощью метода стационарной фазы получаем
\begin{equation}
    \Psi(\vec{k}) =  \frac{2\pi i}{k} \phi(k)  f(\theta) e^{-\frac{i\hbar k^2t}{2m} + ikR}
\end{equation}
Теперь уже можно положить $\phi(k) = \delta(k - k_0)$. Тогда, пользуясь формулой 
\begin{equation}
    \delta(k - k_0) = \frac{\hbar^2 k_0}{m} \delta(\epsilon - \epsilon_0),
\end{equation}
получим окончательно
\begin{equation}
    \Psi(\vec{k}) = \frac{2\pi i \hbar^2}{m} f(\theta) \delta(\epsilon - \epsilon_0)
\end{equation}
\section{Квазиклассическое приближение}
\subsection{Функция Эйри}
Функция Эйри --- решение уравнения
\begin{equation}
	u'' = xu
\end{equation}
Оно решается методом Лапласа, его решение ---
\begin{equation}
	u = \oint_\Gamma e^{ixz - \frac{iz^3}{3}}\,dz
\end{equation}
Контур $\Gamma$ должен начинаться и заканчиваться на бесконечности, так, чтобы 
$e^{-\frac{iz^3}{3}} \to 0$. Выберем его так, чтобы он шёл из третьей координатной четверти
к нулю и уходил на бесконечность вдоль $Ox$. Тогда 
\begin{equation}
	\Psi = \frac{\sqrt{\pi}}{|x|^{\frac14}}\left\{ \begin{matrix}
			e^{\frac23 x^{\frac32}}, & \quad x \gg 0 \\
			e^{\frac{2i}{3} |x|^{\frac32} + \frac{i\pi}{4}}, & \quad x \ll 0
			\end{matrix} \right.
\end{equation}
Методом перевала можно вычислить и мнимую часть $\Psi$. 
\begin{equation}
	\operatorname{Im}\Psi = \frac{\sqrt{\pi}}{|x|^{\frac14}}\left\{ \begin{matrix}
			\frac12 e^{-\frac23 x^{\frac32}}, & \quad x \gg 0 \\
			\sin{\left(\frac{2}{3} |x|^{\frac32} + \frac{\pi}{4}\right)}, & \quad x \ll 0
			\end{matrix} \right.
\end{equation}
\section{Квазиклассическая волновая функция}

\section{Разложение плоской волны по сферическим}
Для начала необходимо разложить плоскую волну по сферическим.
\begin{equation}
	e^{i\vec{k}\vec{r}} = e^{ikr\cos{\theta}} = \sum_l A_l(kr) P_l(\cos{\theta})
\end{equation}
Перепишем это разложение несколько в другом виде:
\begin{equation}
	e^{ixy} = \sum_l A_l(x) P_l(y)
\end{equation}
Отсюда
\begin{equation}
	A_l(x) = \frac{2l+1}{2} \int_{-1}^{1} P_l(y) e^{ixy} \, dy
\end{equation}
Вспоминаем формулу для полиномов Лежандра:
\begin{equation}
	P_l(y) = \frac{1}{2^l l!} \frac{d^l}{dy^l} (y^2 - 1)^l
\end{equation}
Тогда после подстановки и $n$--кратного интегрирования по частям получим 
\begin{equation}
	A_l(x) = \frac{(2l+1)(ix)^l}{2} \int_{-1}^{1} (1-y^2)^l e^{ixy} \, dy
\end{equation}
Здесь нас интересует случай больших $x$. Основной вклад в интеграл дают окрестности 
$y = \pm 1$ (можно деформировать контур так, чтобы это стало совсем очевидно).
{ \it Интересно, как точно вычислить этот интеграл? Ответ тут мне известен аж в двух
смыслах: во--первых, интеграл сводится к вырожденной гипергеометрии, 
во-вторых, должны получиться
функции Бесселя полуцелого порядка. }

После вычисления асимптотики получается следующий результат:
\begin{equation}
    e^{ixy} = \sum \frac{2l+1}{2ix} \left[ e^{ix} + (-1)^{l+1} e^{-ix} \right] P_l(y)
\end{equation}

\section{Задача рассеяния}
Пусть $R_l(r)$ --- радиальные функции. Так как на больших расстояниях потенциала нет,
они имеют асимптотический вид
\begin{equation}
    R_l(r) \approx \frac{1}{r}\left(e^{ikr} + (-1)^{l+1}e^{-ikr-i\alpha_l}\right)
\end{equation}
(Множитель $(-1)^{l+1}$ выбран для удобства в последующем.)    

Любая функция с цилиндрической симметрией должна представляться в виде
\begin{equation}
    \label{expansion}
    \Psi(r,\theta) = \sum_l R_l(r) P_l(\cos{\theta})
\end{equation}
Пусть плоская волна падает на рассеивающий центр. Тогда волновая функция имеет вид
\begin{equation}
    \Psi(r,\theta) = e^{ikr\cos{\theta}} + \frac{f(\theta)e^{ikr}}{r}
\end{equation}
Используя разложение плоской волны и \eqref{expansion}, можно найти $f(\theta)$. Ответ
получается таким:
\begin{equation}
    f(\theta) = \sum_l \frac{2l+1}{2ik} (e^{i\alpha_l} - 1) P_l(\cos{\theta})
\end{equation}

\section{Рассеяние в квазиклассическом случае}

\section{Рассеяние на непроницаемой сфере}
Непроницаемая сфера, возможно, --- единственный случай, когда радиальные функции можно
вычислить точно. Они являются линейными комбинациями функций Бесселя полуцелого порядка.

\end{document}
