\documentclass{article}
\usepackage{amsmath}
\usepackage{graphicx}
\usepackage[utf8]{inputenc}
\usepackage[T1, T2A]{fontenc}
\usepackage[english,russian]{babel}

\newcommand{\average}[1]{\langle #1 \rangle}
\DeclareMathOperator{\sign}{sign}

\title{Current--current correlator and Landau susceptibility}

\begin{document}
\maketitle
It is possible to calculate the diamagnetic susceptibility of electron gas using 
the linear responce of electric current. 

The perturbation caused by magnetic field:
\begin{equation}
    V = -A_\alpha j_\alpha(-k) e^{-i\omega t}
\end{equation}

The bare current operator:
\begin{equation}
    j(k) = \frac{e}{m} \int \frac{d^3 k'}{(2\pi)^3} 
                            \Psi^\dagger_{k'-\frac{k}{2}}
                            \Psi_{k'+\frac{k}{2}} k'
\end{equation}
The response of current to the perturbation:
\begin{equation}
    \frac{\average{J_\alpha}}{V} = -\frac{ne^2}{mc} A_\alpha e^{-i\omega t} +
        \frac{ie^{-i\omega t}}{cV} \int_0^{\infty} d\tau  e^{i\omega \tau}
            \average{[j_\alpha (k, \tau) j_\beta (-k,0)]} A_\beta
\end{equation}
We need to calculate the current--current Green's function:
\begin{multline}
    iG^F_{\alpha \beta} \equiv \frac{1}{V}\int_{-\infty}^{\infty} d\tau  e^{i\omega \tau}
            \average{\mathrm{T}j_\alpha (k, \tau) j_\beta (-k,0)} = \\
    = \frac{e^2}{m^2} \int \frac{d \omega'}{2\pi} 
        \int \frac{d^3 k'}{(2\pi)^3} 
            \frac{k_\alpha' k_\beta'}{(\omega + \omega' - 
                                       \xi_{k'+\frac{k}{2}} + i0\sign{\xi_{k'+\frac{k}{2}}})
                                      (\omega' - 
                                       \xi_{k'-\frac{k}{2}} + i0\sign{\xi_{k'-\frac{k}{2}}})}
\end{multline}
After integration by $\omega$ we get
\begin{equation}
    iG^F_{\alpha \beta} = \frac{-ie^2}{m^2} \left[
        \int_{(1)} \frac{d^3 k'}{(2\pi)^3} \frac{k_\alpha' k_\beta'}
                                {\xi_{k'+\frac{k}{2}} - \xi_{k'-\frac{k}{2}} - \omega - i0} +
        \int_{(2)} \frac{d^3 k'}{(2\pi)^3} \frac{k_\alpha' k_\beta'}
                                {\omega + \xi_{k'-\frac{k}{2}} - \xi_{k'+\frac{k}{2}} - i0} 
                                \right]
\end{equation}
Subscripts $(1)$ and $(2)$ denote the areas 
$\xi_{k'+\frac{k}{2}} > 0 > \xi_{k'-\frac{k}{2}}$ and
$\xi_{k'+\frac{k}{2}} < 0 < \xi_{k'-\frac{k}{2}}$ respectively.
As we are interested in diamagnetic susceptibility, we need to consider the limit
$\omega \to 0$. In such case, the integrals may be simplified so that $k$ expansion 
becomes easy.

Let us take $k$ in the form $(0,0,k)$.  The interesting term is the response of $J_x$ to 
$A_x$, which after some calculus reads
\begin{equation}
   \frac{J_x}{V} = A_x \frac{ne^2}{mc}\left(-1 + \frac{3}{2} \int_0^1 
        dx\,(1-x^2)\sqrt{1 - \frac{k^2}{k_f^2}(1-x^2)} 
            \left(1 - \frac{k^2}{4k_f^2} + \frac{k^2x^2}{2k_f^2}\right)\right)
\end{equation}
By $k^2$ expansion, it's easy to recover Landau susceptibility.
\end{document}
