\documentclass{article}
\usepackage{amsmath}
\usepackage{graphicx}
\usepackage[utf8]{inputenc}
\usepackage[T1, T2A]{fontenc}
\usepackage[english,russian]{babel}

\newcommand{\bra}{\langle}
\newcommand{\ket}{\rangle}

\title{Интегралы по траекториям}
\author{Anikin Evgeny, 121}

\begin{document}
\maketitle
\section{Интеграл по траекториям $x(t), p(t)$}
Для начала рассмотрим амплитуду 
\begin{equation}
U(x_n, x_0, t) = \bra x_n | e^{-i\hat{H}t} | x_0 \ket
\end{equation}
Её можно разбить на малые части:
\begin{equation}
\label{first}
U(x_n, x_0, t) = \int \prod_{k = 1}^{ n-1} dx_k\,
			\bra x_n | e^{-i\hat{H}\delta t} | x_{n-1} \ket
			\dots
			\bra x_1 | e^{-i\hat{H}\delta t} | x_{0} \ket
\end{equation}
Пусть $\hat{H}$ --- гамильтониан одномерной нерелятивистской частицы. 
Проделаем
кое-какие преобразования:
\begin{equation}
\bra x_k | e^{-i\hat{H}\delta t} | x_{k-1} \ket \approx
	\bra x_k | 1 - i\delta t\left(\frac{\hat{p}^2}{2m}
		+ \hat{V} \right) | x_{k-1} \ket
\end{equation}
Можно получить, что
\begin{equation}
\bra x_k | \frac{\hat{p}^2}{2m} + \hat{V} | x_{k-1} \ket = 
	\int \frac{d^3p}{(2\pi)^3}\left[ \frac{p^2}{2m} 
				+ V\left(\frac{x+y}{2}\right) \right]
				e^{ip(x-y)}
\end{equation} 
Тогда для оператора эволюции (\ref{first}) получим интеграл по траекториям
в пространстве координат и импульсов:

\begin{equation}
U(x_n, x_0, t) = \int \frac{\mathcal{D}x \mathcal{D}p}{(2\pi)^3}
				\mathrm{exp}\left\{
						i\int dt\, (p\dot{x} - H(p,x))
						\right\}
\end{equation}
\section{Пропагатор}
Я понял, почему контур поворачивается именно так:
$\t \arrow (1 - i\epsilon)\tau$.  Я понял, что пропагатор --- это матричный
элемент матрицы, обратной к действию.
\end{document}
