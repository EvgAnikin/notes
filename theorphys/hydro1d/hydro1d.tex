\documentclass{article}
\usepackage{amsmath}
\usepackage{graphicx}
\usepackage[utf8]{inputenc}
\usepackage[T1, T2A]{fontenc}
\usepackage[english,russian]{babel}

\title{Гидродинамика мелкой воды}

\begin{document}
\maketitle
Будем изучать "теорию мелкой воды", то есть гидродинамику в таком приближении,
когда вертикальное движение жидкости не учитывается. Соответственно, состояние жидкости
описывается двумя переменными: высотой в точке $h$ и скоростью в точке $v$. Предполагается, 
что есть сила тяготения.

\end{document}
