\documentclass{article} 
%\usepackage[margin=3cm]{geometry}
\usepackage{amsmath}
\usepackage[english, russian]{babel}
\usepackage[utf8]{inputenc}
\usepackage[T2A, T1]{fontenc}


\usepackage{gnuplottex}

\title{Модель Изинга}
\author{Евгений Аникин}

\begin{document}
	\maketitle
	В данном реферате рассмотрены одномерная модель Изинга и модель, состоящая из двух
	связанных цепочек спинов. Для одномерной модели Изинга найдена статистическая сумма 
	и корреляционная функция. Для модели из двух цепочек найдена статистическая сумма и 
	средняя энергия.

	\section{Одномерная модель Изинга}
		\subsection{Статсумма}
		Одномерная модель Изинга задаётся гамильтонианом
		\begin{equation}
			\label{ising1d}
			H = -J\sum s_k s_{k+1},
		\end{equation}
		где переменные $s_k$ (спины) принимаюют значения $\pm 1$. Статсумма для этой модели
		записывается в виде
		\begin{equation}
			Z = \sum_{s_k = \pm 1} e^{\beta J\sum s_k s_{k+1}}
		\end{equation}
		Эту статсумму можно найти точно. Для этого введём так называемую трансфер--матрицу:
		\begin{equation}
			\label{transfermatrix}
			T_{s_1s_2} = e^{\beta J\sum s_1 s_2}
		\end{equation}
		Будем для удобства считать, что имеются периодические граничные условия.
		Легко видеть, что в этом случае
		\begin{equation}
			Z = \mathop{\mathrm{Tr}} T^N = \mathop{\mathrm{Tr}} 
			\left (
			\begin{matrix}
				e^{\beta J} & e^{-\beta J} \\
				e^{-\beta J} & e^{\beta J}
			\end{matrix}
			\right )^N
		\end{equation}
		Легко проверить, что собственные значения трансфер--матрицы --- $2\cosh{\beta J}$ и 
		$2\sinh{\beta J}$. Таким образом, мы получим статсумму в виде
		\begin{equation}
			\label{partfunc}
			Z = 2^N(\cosh^N{\beta J} + \sinh^N{\beta J})
		\end{equation}
		Разумеется, второе слагаемое "вымирает" в термодинамическом пределе, то есть при 
		$N \to \infty$.
		
		\subsection{Корреляционная функция}
		Двухточечная корреляционная функция --- это среднее от произведения спинов в разных 
		узлах решётки:
		\begin{equation}
			\langle s_0 s_m \rangle =	
				\frac{1}{Z}\sum_{s_k = \pm 1} s_0 s_m e^{-\beta H}
		\end{equation}
		Два выделенных спина разделяют цепочку на две части (цепочка замкнута
		в кольцо) длины $m$ и $N - m$. Для каждой из этих частей можно посчитать статсумму при
		"замороженных" спинах $s_0$ и $s_m$. После этого можно будет легко посчитать среднее.
		
		Проделаем это. Перепишем среднее в виде
		\begin{multline}
			\label{corr_aux}
			\langle s_0 s_m \rangle = \frac{1}{Z} \sum_{s_0, s_m = \pm 1}
				s_0 s_m \left \{\sum_{s_1,\dots s_{m-1}} 
						T_{s_0 s_1} \dots T_{s_{m-1}s_m} \times \right.\\
				\left. \times\sum_{s_{m+1}, \dots s_{N-1}} 
						T_{s_m s_{m+1}} \dots T_{s_{N-1}s_m} \right\} = \\
					= \frac{1}{Z} \sum_{s_0, s_m = \pm 1}
				s_0 s_m (T^m)_{s_0 s_m} (T^{N-m})_{s_m s_0}
		\end{multline}
		В последнее выражение вошли степени от трансфер--матрицы. Легко доказать, например,
		по индукции, что
		\begin{equation}
			T^k = 2^{k-1}
			\left(
			\begin{matrix}
				\cosh^k{\beta J} + \sinh^k{\beta J} & \cosh^k{\beta J} - \sinh^k{\beta J} \\
				\cosh^k{\beta J} - \sinh^k{\beta J} & \cosh^k{\beta J} + \sinh^k{\beta J} 
			\end{matrix}
			\right)
		\end{equation}
		Подстановка в уравнение (\ref{corr_aux}) матричных элементов и статсуммы 
		(\ref{partfunc}),
		простые алгебраические преобразования и взятие предела $N \to \infty$ дают ответ
		\begin{equation}
			\langle s_0 s_m \rangle = \tanh J^m = e^{-m\log \coth J}
		\end{equation}
		Таким образом, корреляционная длина --- $\lambda^{-1} = \log \coth J$.

		При малых $J$ $\lambda^{-1} \approx \log\frac{1}{J}$, при больших $J$ 
		$\coth{J} \approx 1 + 2e^{-2J}$, и $\lambda^{-1} \approx 2e^{-2J}$


		\subsection{Ренормгруппа}
		Зависимость корреляционной длины от расстояния 
		можно найти с помощью метода ренормгруппы,
		мощного метода квантовой теории поля. Конечно, 
		в приложении к одномерной модели Изинга
		это может иметь только иллюстративное значение 
		(как, впрочем, и сама одномерная модель
		Изинга).

		Суть ренормгруппы состоит в следующем: 
		в статсумме \ref{ising1d} выполним суммирование
		не по всем, а лишь по нечётным спинам (будем считать, 
		что в цепочке чётное число спинов
		или она вовсе не замкнута). Полученную сумму будем интерпретировать 
		как $e^{-H_{\mathrm{eff}}}$,
		где $H_{\mathrm{eff}}$ --- эффективный гамильтониан. 
		Таким образом, от исходной системы мы
		переходим к системе, зависящей от в два раза меньшего числа спинов, с другим 
		гамильтонианом. При этом статсумма новой системы, разумеется, 
		будет совершенно такой же,
		а все физические величины испытают масштабное 
		преобразование. Например, корреляционная
		длина уменьшится в два раза.

		Посмотрим, какой именно получится эффективный гамильтониан. 
		Для удобства будем считать,
		что $\beta = 1$, и статсумма зависит только от константы связи $J$.
	
		Имеем
		\begin{equation}
			e^{-H_{eff}} = \sum_{s_{2k+1}} \prod_k T_{2k, 2k+1} T_{2k+1, 2k+2} = 
					\prod_k \widetilde{T}_{s_2k s_{2k+2}} 
		\end{equation}
		где $T$ --- всё то же самая трансфер--матрица, 
		а $\widetilde{T} = T^2$ --- эффективная
		трансфер--матрица.
		Легко проверить, что
		\begin{equation}
			T^2 =
			\left (
			\begin{matrix}
				e^{\beta J} & e^{-\beta J} \\
				e^{-\beta J} & e^{\beta J}
			\end{matrix}
			\right )^2 = 
			C\left (
			\begin{matrix}
				e^{ \beta \widetilde{J}} & e^{-\beta \widetilde{J}} \\
				e^{-\beta \widetilde{J}} & e^{ \beta \widetilde{J}}
			\end{matrix}
			\right )
		\end{equation}
		Здесь $C$ --- несущественная константа, а $\widetilde{J}$ --- эффективная константа
		связи, равная
		\begin{equation}
			\label{Jchange}
			\widetilde{J} = \frac{1}{2} \log \cosh{2J}
		\end{equation}
		Получается, что мы эффективно переходим к совершенно такой же трансфер--матрице (и
		гамильтониану), только с другой константой связи. 
		Это позволяет написать функциональное соотношение на корреляционную длину. Очевидно,
		что под действием преобразования ренормгруппы корреляционная длина уменьшается в два 
		раза. С другой стороны, мы показали, что константа связи меняется по 
		закону \ref{Jchange}.
		Значит,
		\begin{equation}
			\lambda\left(\frac{1}{2} \log \cosh{2J}\right) = \frac{1}{2} \lambda(J)
		\end{equation}
		Рассмотрим здесь пределы больших и маленьких $J$.
		Для малых $J$ соотношение переходит в 
		\begin{equation}
			\lambda(J^2) \approx \frac{1}{2}\lambda(J)
		\end{equation}
		Решением такого уравнения является
		\begin{equation}
			\lambda  \approx \frac{C}{\log{\frac{1}{J}}}
		\end{equation}
		Для больших $J$ 
		\begin{equation}
			\lambda\left(J - \frac{\log2}{2}\right) \approx \frac12 \lambda(J)
		\end{equation}
		Решением будет $\lambda(J) = Ce^{-2J}$.
		Эти результаты, как можно видеть, согласутся с результатами предыдущего параграфа.
%		\subsection{Магнитное поле}
%		Гамильтониан с учётом взаимодействия с магнитным полем может быть записан так:
%		\begin{equation}
%			\label{withfield}
%			H = -J\sum s_k s_{k+1} - \frac{B}{2}\sum s_k + s_{k+1}
%		\end{equation}
	\section{Двухслойная модель Изинга}
		Рассмотрим модель с гамильтонианом
		\begin{equation}
			H = -J\sum_k a_k a_{k+1} + b_k b_{k+1} + \frac{1}{2}(a_k b_k + a_{k+1} b_{k+1}),
		\end{equation}
		где $a_k, b_k = \pm 1$. Множитель $\frac{1}{2}$ нужен, потому что каждый член вида 
		$a_k b_k$ встречается в гамильтониане два раза.
		
		Введём трансфер--матрицу:
		\begin{equation}
			T_{a,b;\,\tilde{a},\tilde{b}} \equiv
				e^{\beta J(a\tilde{a}+ b\tilde{b} + 
						\frac{1}{2}ab + \frac{1}{2}\tilde{a}\tilde{b})}
		\end{equation}
		Статсумма перепишется через неё как
		\begin{equation}
			Z = \mathop{\mathrm{Tr}} T^N
		\end{equation}
		Явный вид трансфер--матрицы таков:
		\begin{equation}
			T = \left( 
				\begin{matrix}
					e^{3\beta J}	& 1 			& 1			& e^{-\beta J}	\\
					1			& e^{\beta J}	& e^{-3\beta J}& 1  		\\	
					1 			& e^{-3\beta J}& e^{\beta J}	& 1			\\
					e^{-\beta J}	& 1 			& 1			& e^{3\beta J}	
				\end{matrix}
				\right)
		\end{equation}
		Чтобы найти статсумму, нужно найти собственные значения трансфер--матрицы.
		Два собственных вектора очевидны сразу: $\vec{v_1} = (1,0,0,-1)$ и 
		$\vec{v_2} = (0,1,-1,0)$.
		Легко проверить, что
		\begin{equation}
			T v_1 = (e^{3\beta J} -  e^{-\beta J}) v_1
		\end{equation}
		\begin{equation}
			T v_2 = (e^{\beta J} -  e^{-3\beta J}) v_2
		\end{equation}
		Ещё два собственных вектора есть в подпространстве, натянутом на векторы
		$v_3 = (1,0,0,1)$ и $v_4 = (0,1,1,0)$. В этом подпространстве в базисе $v_3, v_4$
		трансфер--матрица имеет вид 
		\begin{equation}
			T_{\langle v_3, v_4 \rangle} =
			\left(
				\begin{matrix}
					e^{3\beta J} + e^{-\beta J}	& 2 \\
					2						& e^{\beta J} + e^{-3\beta J}
				\end{matrix}
			\right)
		\end{equation}
		Отсюда после не очень приятных вычислений имеем ещё два собственных значения:
		\begin{equation}
			\lambda_{3,4} = 2\left[ \cosh{\beta J} \cosh{2\beta J} \pm
					\sqrt{(\cosh{\beta J} \cosh{2\beta J})^2 - \sinh^2{2\beta J}}\right]
		\end{equation}
		Можно показать, что максимальное собственное значение при всех $\beta$ --- 
		\begin{equation}
			\lambda_{\mathrm{max}} = 2\left[ \cosh{\beta J} \cosh{2\beta J} +
					\sqrt{(\cosh{\beta J} \cosh{2\beta J})^2 - \sinh^2{2\beta J}}\right]
		\end{equation}
		В термодинамическом пределе 
		\begin{equation}
			Z = \lambda_{\mathrm{max}}^N 
		\end{equation}
		Среднюю энергию отсюда можно вычислить по формуле
		\begin{equation}
			\langle U \rangle = -\frac{\partial \log Z}{\partial \beta}
		\end{equation}
		Это выражение крайне громоздко, поэтому я его здесь не привожу.
		Ниже приведён график средней энергии в зависимости от $\beta J$ в сравнении со
		средней энергией простой одномерной модели Изинга, для 
		котороый средняя энергия --- 
		\begin{equation}
			\langle U \rangle = -J \tanh{\beta J}
		\end{equation}
		\begin{figure}[h]
			\begin{centering}
				\begin{gnuplot}[terminal=epslatex, terminaloptions=color dashed]
					set xrange [0:3]
					set yrange [0:1.3]
					set xlabel '$\beta J$'
					set ylabel '$-\frac{U}{J}$'
					plot	tanh(x) title "Одна цепочка",\
					     1./3*(((4*sinh(2*x)*cosh(2*x)*cosh(x)**2) + \
						2*sinh(x)*cosh(2*x)**2*cosh(x) - \
						4*sinh(2*x)*cosh(2*x)) /  \
						(2*sqrt(cosh(x)**2*cosh(2*x)**2 - sinh(2*x)**2)) \
						+ sinh(x)*cosh(2*x)+2*sinh(2*x)*cosh(x))\
						/(cosh(x)*cosh(2*x)+sqrt(cosh(x)**2*cosh(2*x)**2 - sinh(2*x)**2)) title "Две цепочки"
				\end{gnuplot}
			\end{centering}
		\end{figure}
\end{document}
