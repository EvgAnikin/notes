\documentclass{article}
\usepackage{amsmath}
\usepackage{graphicx}
\usepackage[utf8]{inputenc}
\usepackage[T1, T2A]{fontenc}
\usepackage[english,russian]{babel}

\newcommand{\bra}{\langle}
\newcommand{\ket}{\rangle}
\newcommand{\p}{\partial}

\title{Кое-что о переходе БКТ}
\author{Anikin Evgeny, 121}

\begin{document}
\maketitle
Рассмотрим гамильтониан
\begin{equation}
	H[\phi] = \int d^2x\, \frac{1}{2} \p_\mu \phi\, \p_\mu \phi - \alpha \cos{\beta \phi}
\end{equation}
Его статсумма:
\begin{equation}
	Z = \int D\phi \exp\{-H[\phi]\}
\end{equation}
Предполагая константу $\alpha$ малой, мы можем разложить статсумму в ряд.
\begin{equation}
	Z = \int D\phi\,\exp{\left[-\int d^2x\, \frac{1}{2} \p_\mu \phi\, \p_\mu \phi\right]}
			\sum \frac{\alpha^N}{N!}	\left(\int d^2x \, \cos{\beta \phi}\right)^N
\end{equation}
Но для начала вычислим следующие корреляционные фунцкции:
\begin{multline}
	\bra \cos{\beta \phi(x_1)} \dots \cos{\beta \phi{x_N}} \ket \sim \\ 
	\sim	\int D\phi \, \exp{\left[-\int d^2x\, \frac{1}{2} \p_\mu \phi\, \p_\mu \phi\right]}
		\cos{\beta \phi(x_1)} \dots \cos{\beta \phi({x_N})}
\end{multline}
Дальше каждый косинус разбивается на экспоненты.
\end{document}
