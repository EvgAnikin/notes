\documentclass{article}
\usepackage{amsmath}
\usepackage{graphicx}
\usepackage[utf8]{inputenc}
\usepackage[T1, T2A]{fontenc}
\usepackage[english,russian]{babel}

\newcommand{\p}{\partial}

\title{Фейнмановские пропагаторы}
\author{Anikin Evgeny, 121}

\begin{document}
\maketitle
\section{Скалярное поле}
Оператор поля записывается через операторы рождения и уничтожения так:
\begin{equation}
\phi = \int\frac{d^3 p}{(2\pi)^3} \frac{1}{\sqrt{2E_p}}
	(a_p e^{-ipx} + a^{\dagger}_p e^{ipx})
\end{equation}
Фейнмановский пропагатор по определению --- это следующее выражение:
\begin{equation}
	D(x - y) = \langle 0 | \mathrm{T} \phi(x)^{\dagger} \phi(y) | 0 \rangle
\end{equation}
Пусть сначала $x_0 > y_0$. Тогда 
\begin{equation}
	D(x - y) = \int\frac{d^3 p\, d^3 p'}{(2\pi)^6}\frac{e^{-ipx + ip'y}}{\sqrt{2E_p 2E_{p'}}}
				\langle 0 | a_p a_{p'}^{\dagger} | 0 \rangle = 
					\int\frac{d^3 p}{(2\pi)^3} \frac{e^{-iE_pt + i(\vec{p} \vec{x})}}{2E_p}
\end{equation}
В противном случае $x$ и $y$ меняются местами. Это приводит к тому, что меняется знак при 
$iE_pt$.
Теперь нужно найти импульсное представление пропагатора.
После пространственного фурье-преобразования пропагатор выглядит так:
\begin{equation}
	D(t, \vec{p}) = \frac{e^{-iE_p|t|}}{2E_p}	
\end{equation}
Осталось сделать временное:
\begin{multline}
	D(p) = \int_{-\infty}^{+\infty} dt\, e^{ip_0t} \frac{e^{-iE_p|t|}}{2E_p} = \\
	=	\int_{0}^{+\infty} dt\, \frac{e^{i(p_0 - E_p)t}}{2E_p} + 
		\int_{0}^{+\infty} dt\, \frac{e^{-i(p_0 + E_p)t}}{2E_p} = \\
		= \frac{1}{2E_p} \left[ \frac{1}{-i(p_0 - E_p)  + \delta} + 
				\frac{1}{i(p_0 + E_p) + \delta}\right]
\end{multline}
Ура! Последнее --- это и есть нужное выражение для пропагатора. Окончательно:
\begin{equation}
	D(p) = \frac{i}{p_0^2 - E_p^2 + i\epsilon} = \frac{i}{p^2 - m^2 + i\epsilon}
\end{equation}
\section{Массивное векторное поле}
Действие:
\begin{equation}
S = \int d^4x \left[  -\frac14 F_{\mu\nu} F^{\mu\nu}  \right.
		\left.	+ \frac{m^2}{2} A_{\nu}A^{\nu} \right]
\end{equation}
Интегрируя по частям, получим
\begin{equation}
S = \frac12\int d^4 x\,  A^{\mu}
	 (g_{\mu\nu}(\p_{\alpha}\p^{\alpha} + m^2) - \p_{\mu}\p_{\nu}) A^{\nu}
\end{equation}
Пропагатор --- это вот что:
\begin{equation}
D^{\mu\nu}(x - y) = \frac{\int \mathcal{D}A\, A^{\mu}(x)A^{\nu}(y)\exp{iS}}
					{\int \mathcal{D}A\,\exp{iS}}
\end{equation}
Нужно найти матричный элемент оператора, обратного к оператору в скобках. Это и будет, с
точностью до множителя, пропагатор.
В импульсном представлении уравнение на пропагатор выглядит так:
	\begin{equation}
		i(g_{\mu\nu}(k^2 - m^2) - k_{\mu}k_{\nu})G^{\nu\rho} = \delta_{\mu}^{\rho}
	\end{equation}
Ответ:
	\begin{equation}
		G^{\nu\rho} = \frac{-ig^{\nu\rho}}{k^2 - m^2} + 
						\frac{ik^{\nu} k^\rho}{m^2(k^2 - m^2)}
	\end{equation}
Нужно ещё вспомнить, как повёрнут контур интегрирования по времени, и соответственно дописать
$i\epsilon$  в нужное место.

\section{Функция Грина системы ферми--частиц}
Теперь обратимся к твёрдому телу. Для многочастичных систем вводят операторы поля:
	\begin{equation}
		\Psi(t,x) = \frac{1}{\sqrt{V}}\sum_p e^{-iE_pt + ipx}a_p
	\end{equation}
Функция Грина определяется так:
	\begin{equation}
		G(t - t', x - x') = -i\langle \Omega |\mathrm{T} 
					\Psi(t,x) \Psi^\dagger (t', x') | \Omega \rangle
	\end{equation}
Отличие от предыдущего определения (из Пескина--Шрёдера) только в множителе $-i$.
$|\Omega\rangle$ --- основное состояние системы из $N$ частиц.
Подставляя операторы поля в определение, получим
	\begin{multline}
		\mathrm{T}\Psi(t,x) \Psi^\dagger (t', x')=\\
			 =\frac{1}{V}\sum_{p,p'} e^{-i E_p t + iE_{p'} t' + ipx - ip'x'} 
				\left\{
				\begin{array}{cl}
					a_p a^{\dagger}_{p'}, & \mbox{если } t > t' \\
					-a^{\dagger}_{p'} a_p, & \mbox{если } t < t' 
				\end{array}
				\right.
	\end{multline}
	\begin{equation}
		G(t - t', x - x') = -i\frac{1}{V}\sum_p e^{-iE_p(t-t') + ip(x-x')}
				\left\{
				\begin{array}{cl}
					1-n_p, & \mbox{если } t > t' \\
					n_p,   & \mbox{если } t < t' 
				\end{array}
				\right.
	\end{equation}
	Тогда в представлении $p,t$ функция Грина выглядит так:
	\begin{equation}
		G(t,p) = e^{-iE_pt}
				\left\{
				\begin{array}{cl}
					1-n_p, & \mbox{если } t > 0 \\
					n_p,   & \mbox{если } t < 0 
				\end{array}
				\right.
	\end{equation}
	Чтобы найти функцию Грина в частотно--импульсном представлении, нужно сделать ещё
	одно фурье--преобразование.
	Ответ получается таким:
	\begin{equation}
		G(\omega, p) = \frac{1-n_p}{\omega - E_p + i\epsilon} + 
						\frac{n_p}{\omega - E_p - i\epsilon}
	\end{equation}	
	\section{Вопрос про спектр возбуждений}
	В Абрикосове--Горькове, в параграфе про аналитические свойства
	функций Грина предполагается, что c точностью $1/N$ ($N$ --- число 
	частиц) 
\end{document}
