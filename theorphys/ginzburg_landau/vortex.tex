\documentclass{article}
\usepackage{amsmath}
\usepackage{graphicx}
\usepackage[utf8]{inputenc}
\usepackage[T1, T2A]{fontenc}
\usepackage[english,russian]{babel}

\newcommand{\bra}{\langle}
\newcommand{\ket}{\rangle}

\title{Вихрь Абрикосова}
\author{Anikin Evgeny, 128}

\begin{document}
\maketitle
Свободная энергия сверхпроводника в магнитном поле ---
\begin{multline}
    F = \int d^3x\,\left[\left|\left(\nabla  - \frac{ie}{c} \vec{A}\right)\Psi\right|^2 - 
                \alpha (T_c - T) |\Psi|^2 + \beta |\Psi|^4  +
                \frac{(H - H_{\mathrm{ext}})^2}{8\pi}\right]
\end{multline}
Будем искать одиночный вихрь --- решение уравнений Гинзбурга--Ландау, имеющее вид
\begin{equation}
    \begin{gathered}
        \Psi = \Psi(r)e^{i\theta}\\
        A_\theta = \frac{c}{er} \Phi(r)\\
        A_r = 0
    \end{gathered}
\end{equation}
Для справки: ротор в криволинейных координатах ---
\begin{equation}
    (\nabla \times \vec{A})_3 = \frac{1}{H_1 H_2}
         \left (\frac{\partial}{\partial x_1} H_2 A_2 - 
                \frac{\partial}{\partial x_2} H_1 A_1 \right)
\end{equation}
Электрический ток получается равен
\begin{equation}
    j_\theta = \frac{2e}{r} |\Psi|^2 (1 - \Phi)
\end{equation}
Магнитное поле --- 
\begin{equation}
    B = \frac{c}{er} \Phi'
\end{equation}
Уравнения получаются такими:
\begin{equation}
    \begin{gathered}
        \chi'' + \frac{\chi'}{r} - \frac{(1 - \Phi)^2}{r^2} \chi + 
            \frac{1}{\xi^2} (\chi - \chi^3) = 0\\
        \Phi'' - \frac{\Phi'}{r} + \frac{\chi^2}{\lambda^2}(1 - \Phi) = 0
    \end{gathered}
\end{equation}
\end{document}
