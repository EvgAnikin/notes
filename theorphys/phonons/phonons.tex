\documentclass{article}
\usepackage{amsmath}
\usepackage{graphicx}
\usepackage[utf8]{inputenc}
\usepackage[T1, T2A]{fontenc}
\usepackage[english,russian]{babel}

\title{Взаимодействие электронов и фононов}
\author{Евгений Аникин, 128}

\begin{document}
\maketitle
\section{Звук в упругой среде}
Деформацию однородной и изотропной среды можно задать вектором смещений $\xi$, зависящим от 
координаты $x$. Энергия деформации зависит от производных $\xi$ и квадратична по ним, поэтому
задаётся тензором:
\begin{equation}
    dW = \frac{1}{2} T_{ijkl}\frac{\partial\xi_i}{\partial x_j}
                    \frac{\partial\xi_k}{\partial x_l}\, d^3x
\end{equation}
Общий вид изотропного тензора $T_{ijkl}$ ---
\begin{equation}
    T_{ijkl} = A\delta_{ij}\delta_{kl} + B\delta_{ik}\delta_{jl} + C\delta_{il}\delta_{jk}
\end{equation}
Связь коэффициентов $A$, $B$, $C$ даётся условием, что энергия деформации равна нулю, когда
$\frac{\partial\xi_i}{\partial\xi_j}$ --- матрица бесконечно малого поворота, то есть 
антисимметрична. Отсюда получается, что $B = C$. Таким образом, потенциальная энергия
поля деформаций ---
\begin{equation}
    W = \frac{1}{2}\int A\left(\frac{\partial\xi_i}{\partial x_i}\right)^2 + 
                               B\left(\frac{\partial\xi_i}{\partial x_j}
                                 \frac{\partial\xi_i}{\partial x_j} + 
                                 \frac{\partial\xi_i}{\partial x_j}
                                 \frac{\partial\xi_j}{\partial x_i}\right)\, d^3 x
\end{equation}
Кинетическая энергия записывается обычным образом:
\begin{equation}
    T = \int \frac{\rho}{2}\left( \frac{\partial \xi}{\partial t}\right)^2 \,d^3x
\end{equation}
Уравнения движения получаются такими:
\begin{equation}
    \rho\frac{\partial^2 \xi_i}{\partial t^2} = 
        \frac{\partial}{\partial x_j}\left( A\frac{\partial\xi_k}{\partial x_k}\delta_{ij} + 
                                            B\left(\frac{\partial\xi_i}{\partial x_j} + 
                                                   \frac{\partial\xi_j}{\partial x_i}\right)
                                                   \right)
\end{equation}
В импульсном представлении
\begin{equation}
    [(\omega^2\rho - Bk^2)\delta_{ij} - (A + B)k_i k_j]\xi_j = 0
\end{equation}
Это уравнение определяет две поперечные и одну продольную моду.  
\section{Квантование фононного поля}
Гамильтониан фононов ---
\begin{equation}
    H = \int d^3x\,\left[\frac{\pi_i^2}{2\rho} + 
    \frac{A}{2}\left(\frac{\partial\xi_i}{\partial x_i}\right)^2 + 
                               \frac{B}{2}\left(\frac{\partial\xi_i}{\partial x_j}
                                 \frac{\partial\xi_i}{\partial x_j} + 
                                 \frac{\partial\xi_i}{\partial x_j}
                                 \frac{\partial\xi_j}{\partial x_i}\right)\right]
\end{equation}
Процедура квантования сводится к замене координат и импульсов операторами с 
коммутационными соотношениями
\begin{equation}
    [\xi_i(x), \pi_j(x')] = (2\pi)^3\hbar\delta(x - x')\delta_{ij}
\end{equation}
Переход к операторам рождения и уничтожения фононов удобно делать в несколько шагов.
Вначале перейдём к импульсному представлению:
\begin{equation}
    \begin{split}
        \xi_i(x) = &  \int \frac{d^3 k}{(2\pi)^3}\, \xi_i(k) e^{ikx} \\
        \pi_i(x) = &  \int \frac{d^3 k}{(2\pi)^3}\, \pi_i(k) e^{-ikx} \\
    \end{split}
\end{equation}
Вещественность $\xi$, $\pi$ даёт
\begin{equation}
    \begin{split}
        \xi(k) = & \xi(-k)^\dagger,\\
        \pi(k) = & \pi(-k)^\dagger
    \end{split}
\end{equation}
Гамильтониан принимает вид
\begin{equation}
    H = \int \frac{d^3 k}{(2\pi)^3} \left[ \frac{\pi_i^\dagger \pi_i}{2\rho} + 
                         \frac{1}{2}((A + B)k_i k_j + Bk^2\delta_{ij})\xi_i^\dagger\xi_j
                                        \right]
\end{equation}
Далее сделаем замену 
\begin{equation}
    \begin{split}
        \xi_i = & e_{i\alpha} \xi_\alpha\\
        \pi_i = & e_{i\alpha}^* \pi_\alpha
    \end{split}
\end{equation}
Теперь
\begin{equation}
    H = \int \frac{d^3 k}{(2\pi)^3} \left[ \frac{\pi_\alpha^\dagger \pi_\alpha}{2\rho} + 
                         \frac{\rho \omega_\alpha^2}{2} \xi_\alpha^\dagger \xi_\alpha
                                        \right]
\end{equation}
Наконец, можно ввести операторы рождения и уничтожения.
\begin{equation}
    \begin{split}
        a_{\alpha} = & \sqrt{\frac{\rho \omega_\alpha}{2\hbar}}\xi_\alpha + 
                        i\sqrt{\frac{1}{2\rho\omega_\alpha \hbar}}\pi_\alpha^\dagger\\
        a_{\alpha}^\dagger = & \sqrt{\frac{\rho \omega_\alpha}{2\hbar}}\xi_\alpha^\dagger - 
                        i\sqrt{\frac{1}{2\rho\omega_\alpha \hbar}}\pi_\alpha\\
        b_{\alpha}^\dagger= & \sqrt{\frac{\rho \omega_\alpha}{2\hbar}}\xi_\alpha - 
                        i\sqrt{\frac{1}{2\rho\omega_\alpha \hbar}}\pi_\alpha^\dagger\\
        b_{\alpha} = & \sqrt{\frac{\rho \omega_\alpha}{2\hbar}}\xi_\alpha^\dagger + 
                        i\sqrt{\frac{1}{2\rho\omega_\alpha \hbar}}\pi_\alpha\\
    \end{split}
\end{equation}
Отсюда можно выразить $\xi_\alpha$, $\pi_\alpha$, а затем $\xi_i$, $\pi_i$:
\begin{equation}
    \begin{split}
        \xi_i & = \sum_\alpha\sqrt{\frac{\hbar}{2\rho \omega_\alpha}}
                            (e_{i\alpha} a_\alpha + e_{i\alpha} b^\dagger_\alpha)\\
        \pi_i & = -i\sum_\alpha \sqrt{\frac{\rho \omega_\alpha \hbar}{2}}
                            (e_{i\alpha}b_\alpha - e_{i\alpha} a_\alpha^\dagger)
    \end{split}
\end{equation}
Кроме того, нужно ещё учесть, что операторы $\xi_i$, $\pi_i$ вещественны, и есть
соотношение между операторами
\begin{equation}
    e_{i\alpha}(k) \xi_\alpha(k) = e_{i\alpha}^*(-k) \xi_{\alpha}^\dagger(-k)
\end{equation}
Окончательно получаем
%\begin{equation}
%    \xi_i(x) = \int \frac{d^3 k}{(2\pi)^3} 
%                (e_{i\alpha}(k) a_\alpha (k)
%\end{equation}
\end{document}
