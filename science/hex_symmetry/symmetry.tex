\documentclass{article}
\usepackage{amsmath}
\usepackage{graphicx}
\usepackage[utf8]{inputenc}
\usepackage[T1, T2A]{fontenc}
\usepackage[english,russian]{babel}

\newcommand{\tpi}{\tilde{T}}
\newcommand{\beq}{\begin{equation}}
\newcommand{\eeq}{\end{equation}}

\title{Симметрийные соображения применительно к шестиугольной решётке}
\author{Anikin Evgeny, 121}

\begin{document}
\maketitle
Пусть решётка построена из шестиугольников или из треугольников так, что она обладает 
симметрией относительно вращений на $\pi n/3$. Будем также считать, что начало
координат находится в центре шестиугольной ячейки, система 
$T$--инвариантна и в ней выбран базис волновых функций $\psi^s_p$, таких что
\begin{equation}
	\begin{gathered}
		\psi^s_{-p} = (\psi^s_p)^*\\
		\psi^s_{p}(-x) = \psi^s_{-p}(x)
	\end{gathered}
\end{equation}

%Рассмотрим антиунитарный оператор $\tpi$, равный комозиции $T$ и поворота на $\pi$ относительно
%центра шестиугольной ячейки. $\hat{H}$, разумеется, $\tpi$--инвариантен. Значит, если 
%$\psi_p$ --- собственное состояние с квазиимпульсом $p$, то и $\tpi\psi_p$ --- 
%состояние с 
%квазиимпульсом $p$.
%
%Несложно показать, что у $\tpi$--симметричного гамильтониана можно выбрать базис,
%каждый вектор которого $\tpi$--инвариантен. Для двумерного пространства
%нужно действовать следующим образом. Пусть $\psi_p$ --- собственный вектор. Рассмотрим 
%векторы
%\begin{equation}
%	a_p = \psi_p + \psi_{-p}^*, \quad b_p = i(\psi_p - \psi_{-p}^*)
%\end{equation}
%Они, как несложно видеть, $\tpi$--инвариантны. Остаётся диагонализовать гамильтониан,
%не нарушая $\tpi$--инвариантности базиса.
%
%Аналогично можно рассуждать и для более высоких размерностей.

Вращения действуют на состояния, разумеется, как унитарные операторы. Вращения инвариантны
относительно обращения времени, а также самих вращений (в том числе вращения на $\pi$). 
Из этого следует, что матрица $\Lambda$ поворота на $\pi n/3$ вещественна.
В самом деле, из инвариантности относительно вращений на $\pi$ следует, что
\begin{equation*}
	\langle \psi_p | \Lambda | \psi_{p'} \rangle = 
		\langle \psi_{-p} | \Lambda | \psi_{-p'} \rangle,
\end{equation*}
а из $T$--инвариантности --- 
\begin{equation*}
	\langle \psi_p | \Lambda | \psi_{p'} \rangle = 
		\langle \psi_p^* | \Lambda | \psi_p'^* \rangle^*
\end{equation*}
Используя соотношение $\psi_{-p} = \psi_p^*$, получим 
\begin{equation}
		\langle \psi_{p} | \Lambda | \psi_{p'} \rangle =
		\langle \psi_{p} | \Lambda | \psi_{p'} \rangle^*,
\end{equation}
что и требовалось.

В шестиугольной зоне Бриллюэна есть четыре точки, инвариантные относительно вращений на 
$pi/3$. Таким образом, в каждой из этих точек матрицы $\Lambda$ образуют представление
треугольной группы.


\end{document}
