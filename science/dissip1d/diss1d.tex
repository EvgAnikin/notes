\documentclass{article}
\usepackage{amsmath}
\usepackage{graphicx}
\usepackage[utf8]{inputenc}
\usepackage[T1, T2A]{fontenc}
\usepackage[english,russian]{babel}

\newcommand{\bra}{\langle}
\newcommand{\ket}{\rangle}

\newtheorem{theorem}{Теорема}

\title{Длина свободного пробега}
\author{Anikin Evgeny, 121}

\begin{document}
\maketitle
В этом листке я ищу длину свободного пробега частиц в одномерной цепочке с хаотически 
расположенными примесями. Гамильтониан невозмущенной цепочки --- 
\begin{equation}
	\label{chain}
	\hat{H} = \sum_n Ea_n^\dagger a_n + Ca_{n+1}^\dagger a + Ca_{n-1}^\dagger a
\end{equation}
\section{Полевой подход}
Пусть к камильтониану (\ref{chain}) добавлен случайный потенциал 
$\hat{U} = \sum_k U(k) a_k^\dagger a_k$, причём 
$\langle U(k) U(l)\rangle = \epsilon^2\delta_{kl}$. Тогда несложно показать, что в самом
низком порядке
\section{''Общефизический`` подход}
Рассмотрим сначала рассеяние плоской волны с квазиимпульсом $k$ на одиночной примеси,
которой соответствует возмущение $\Delta E a_0^\dagger a_0$.
Коэффициент отражения,
как нетрудно показать, равен
\begin{equation}
	R(p) = \frac{\Delta E^2}{\Delta E^2 + 4t^2 \sin^2{k}}
\end{equation}
Пусть концентрация примесей --- $n$. Тогда длина свободного пробега ---  
$\lambda^{-1} = R(p)n$. 
Вместо концентрации примесей и величины $\Delta E$ удобно ввести средний квадрат 
хаотического потенциала:
\begin{equation}
	\epsilon^2 = \langle \Delta E^2 \rangle \approx n\Delta E^2
\end{equation}
Групповая скорость электронов --- $v = -2t\sin k$. 
Получается формула
\begin{equation}
	\lambda = \frac{\Delta E^2 + v^2}{\epsilon^2} \approx \left(\frac{v}{\epsilon}\right)^2
\end{equation}
\end{document}
