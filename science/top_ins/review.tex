\documentclass{article}
\usepackage{afterpage}
\usepackage{amsmath}
\usepackage{amssymb}
\usepackage{bbm}
\usepackage[legalpaper,margin=2cm,top=2cm,bottom=2cm]{geometry}
%\usepackage[legalpaper]{geometry}
\usepackage{graphicx}
\usepackage[utf8]{inputenc}
\usepackage[T1, T2A]{fontenc}
\usepackage[english,russian]{babel}
\usepackage{pdflscape}

\bibliographystyle{unsrt}

\DeclareMathOperator{\diag}{diag}
\DeclareMathOperator{\Arg}{Arg}

\newcommand{\bra}{\langle}
\newcommand{\ket}{\rangle} 

\title{Литература по топологическим изоляторам}
\author{Anikin Evgeny, 128}

\begin{document}
    \maketitle
    \section{Основания}
    Теоретическое предсказание топологичности CdTe--HgTe--CdTe было сделано в 
    \cite{Bernevig2006}, а первое экспериментальное подтверждение --- \cite{Konig2007}.

    \section{Эксперименты}
    \cite{Konig2007} --- пионерская работа.
    \cite{Gusev2011} --- эксперимент Квона и др.
    \section{Edge reconstruction}
    \cite{Wang2017} --- спонтанная намагниченность у края TI с плавным потенциалом.
    
    \section{Квантовые точки}
    В \cite{Li2014} обсуждаются квантовые точки на базе CdTe--HgTe--CdTe с 
    учётом кулоновского взаимодействия. Квантовая точка окружена стенкой.

    В \cite{Xin2015} --- эффект Кондо в квантовых точках на основе Bi2Se3.

    В \cite{Herath2014} --- квантовая точка на основе Bi2Se3, изучается оптический 
    спектр, взаимодействие не учитывается. Точка представляет из себя утолщение 
    на поверхности.

    \section{Эффекты беспорядка}
    \cite{Li2009} --- топологический изолятор Андерсона.
    \cite{Girschik2015} --- что--то ещё про топологический андерсоновский изолятор. 

    \cite{Entin2015} --- локализация из--за перепрыгивания с края на край (ЖЭТФ).
    \subsection{Немагнитные примеси}
    \cite{Lu2011}, https://arxiv.org/pdf/1408.2629.pdf
    \subsection{Магнитные примеси}
    \cite{Altshuler2013} --- локализация краевых мод в присутствии массива примесей Кондо. 
    \cite{Yevtushenko2015} --- эффекты взаимодействия в предыдущей статье.
    
    \newpage
    \bibliography{literature}
\end{document}
