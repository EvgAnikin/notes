\documentclass{article}
\usepackage{afterpage}
\usepackage{amsmath}
\usepackage{amssymb}
\usepackage{bbm}
\usepackage[legalpaper,margin=2cm,top=2cm,bottom=2cm]{geometry}
%\usepackage[legalpaper]{geometry}
\usepackage{graphicx}
\usepackage[utf8]{inputenc}
\usepackage[T1, T2A]{fontenc}
\usepackage[english,russian]{babel}

\newcommand{\bra}{\langle}
\newcommand{\ket}{\rangle} 

\title{Симметрия относительно обращения времени}
\author{Anikin Evgeny, 128}

\begin{document}
\maketitle
Нетрудно проверить, что для частицы со спином $\frac{1}{2}$
\begin{equation}
    \bra \chi | T\psi \ket = -\bra \psi | T\chi\ket
\end{equation}
Тогда $\bra \psi | \hat{A}  | T\psi \ket = 0$ для любого 
$|\psi \ket$, если
$\hat{A}$ удовлетворяет условию $-T\hat{A}^\dagger T = \hat{A}$.
Действительно,
\begin{equation}
    \bra \psi | \hat{A}  | T\psi \ket = \bra \hat{A}^\dagger \psi | T\psi \ket = 
       -\bra \psi | T \hat{A}^\dagger| \psi \ket = -\bra \psi | (-T\hat{A}^\dagger T) | T\psi \ket
\end{equation}
Оператор эволюции $T$--инвариантной системы удовлетворяет именно такому условию:
\begin{equation}
    -T\hat{U}T = U^{-1} = U^\dagger
\end{equation}
Поэтому рассеяние $|\psi\ket$ и $|T\psi\ket$ друг в друга запрещено.

\end{document}
