\documentclass{article}
\usepackage{amsmath}
\usepackage{graphicx}
\usepackage[utf8]{inputenc}
\usepackage[T1, T2A]{fontenc}
\usepackage[english,russian]{babel}

\title{Локализованное состояние\\ в подходе сильной связи}
\author{Евгений Аникин}

\begin{document}
\maketitle
\section{Гамильтониан в подходе сильной связи}
Гамильтониан для электронов в решётке можно записать так:
\begin{equation}
	H = \sum_p \epsilon_p a_p^{\dagger} a_p,
\end{equation}
Здесь $a_p$ --- оператор уничтожения частицы в состоянии с квазиимпульсом $p$ и энергией
$\epsilon_p$. В периодическом потенциале можно ввести новый базис из состояний, волновые
функции которых называются функциями Ванье. Определим новые операторы уничтожения $w_n$:
\begin{equation}
	w_n = \frac{1}{\sqrt{N}} \sum_p e^{-ipR_n} a_p
\end{equation}
Выразив $a_p$ через $w_n$, можно получить гамильтониан в приближении сильной связи:
\begin{equation}
	\label{hamiltonian}
	H = \sum_k E\,w_k^{\dagger}w_k + \sum_{k\ne l} C_{k-l}\, w_k^{\dagger} w_l
\end{equation}
Энергии частиц выражаются через новые коэффициенты $E$, $C_n$ таким образом:
\begin{equation}
	\epsilon_p = E + \sum_{n\ne 0} C_n e^{ipR_n}
\end{equation}
В дальнейшем нам потреуются функции Грина в узельном представлении, определённые через
операторы $w_n(t)$. Например, мацубаровская функция Грина --- 
\begin{equation}
	\mathcal{G}_0(\omega_n,R_m - R_n) = 
%		-i\langle \mathop{T} w_m(\tau) w_n^{\dagger}(\tau') \rangle = 
			\frac{1}{N}\sum_p  \frac{e^{-ip(R_m - R_n)}}{i\omega_n - E_p}.
\end{equation}
Аналогично, запаздывающая функция Грина --- 
\begin{equation}
	G_0^R(\omega,R_m - R_n) = 
			\frac{1}{N}\sum_p \frac{e^{-ip(R_m - R_n)}}{\omega - E_p + i\delta}.
\end{equation}
\section{Примесный атом}
Добавим к исходному гамильтониану возмущение:
\begin{equation}
	V = \Delta E w_0^{\dagger} w_0
\end{equation}
Это соответствует примесному атому, находящемуся в начале координат.
Чтобы найти новую функцию Грина, нужно решить уравнение Дайсона, которое в данном случае
примет вид
\begin{equation}
	\mathcal{G}(\omega_n,m,n) = 
		\mathcal{G}_0(\omega_n, m,n) + 
		\Delta E \mathcal{G}_0(\omega_n, m,0)\mathcal{G}(\omega_n, 0,n)
\end{equation}
Решение этого уравнения --- 
\begin{equation}
	\mathcal{G}(\omega_n, m,n) = \mathcal{G}_0(\omega_n, m,n) + 
		\frac{\Delta E\mathcal{G}_0(\omega_n, m,0)\mathcal{G}_0(\omega_n, 0, n)}{1 - \Delta E \mathcal{G}_0(\omega_n,0,0)}
\end{equation}
Заменяя здесь $i\omega$ на $\omega + i\delta$, найдём запаздывающую функцию Грина.
\begin{equation}
	\label{solution}
	G^R(\omega, m,n) = G^R_0(\omega, m,n) + 
		\frac{\Delta EG^R_0(\omega, m,0)G^R_0(\omega, 0, n)}{1 - \Delta E G^R_0(\omega,0,0)}
\end{equation}
Полюса запаздывающей функции Грина определяют энергетический спектр системы. Здесь полюса ---
это решения уравнения
\begin{equation}
	\Delta E G^R_0(\omega, 0,0) = 1,
\end{equation}
то есть 
\begin{equation}
	 \frac{\Delta E}{N}\int  \frac{\rho(\epsilon)\,d\epsilon}{\omega - \epsilon + i\delta}= 1
\end{equation}
Кроме того, зная запаздывающую функцию Грина, можно найти волновую функцию связанного
состояния. Предположим, что мы диагонализовали возмущённый гамильтониан и нашли
набор собственных состояний $|\lambda\rangle$. Тогда функция Грина $G^R$ --- 
\begin{equation}
	\label{general}
	G^R(\omega, m,n) = \sum_\lambda \psi_\lambda(m)\psi_\lambda^{*}(n) 
			\frac{1}{\omega - E_\lambda + i\delta}
\end{equation}
Сравнивая (\ref{solution}) и (\ref{general}), получим, что
\begin{equation}
	\psi_\Lambda(n) \propto G_0^{R}(E_\Lambda, n,0)
\end{equation}
\subsection{Слабосвязанное состояние в трёхмерной решётке}
Предположим, что $E_\Lambda$ близко к нижней границе зоны, а 
$\epsilon_p = \frac{p^2}{2m} + o(p^2)$. Положим для удобства $p_0^2 = -2mE_\Lambda$.

Тогда имеем:
\begin{equation}
	\psi_\Lambda(n) 
	\propto \frac{V}{N}\int \frac{d^3 p}{(2\pi)^3}\, \frac{e^{ipR_n}}{p_0^2 + p^2 + o(p^2)}	
\end{equation}
Здесь интегрирование ведётся по зоне Бриллюэна. Однако, если $R_n \gg a$, $p_0^{-1} \gg a$, 
то основной вклад в интеграл вносит область малых $p$. Поэтому можно считать, что
\begin{equation}
	\psi_\Lambda(n) 
	\propto \frac{V}{N}\int \frac{d^3 p}{(2\pi)^3}\, \frac{e^{ipR_n}}{p_0^2 + p^2},
\end{equation}
где интеграл берётся по всем $p$. Такой интеграл можно взять, сначала перейдя в полярные координаты, а затем --- с помощью вычетов.
Ответ получается такой:
\begin{equation}
	\psi_\Lambda(n) \propto \frac{e^{-p_0R_n}}{R_n}
\end{equation}
\subsection{Сильносвязанное состояние}
Пусть теперь $E_\Lambda$ лежит далеко от зоны, а $\epsilon_p \sim 0$. Тогда
\begin{equation}
	\psi_\Lambda(n) 
	\propto \frac{V}{N}\int \frac{d^d p}{(2\pi)^d}\, \frac{e^{ipR_n}}{E_\Lambda - \epsilon_p}
	\approx \frac{V}{N}\int \frac{d^d p}{(2\pi)^d}\, 
			\frac{e^{ipR_n}}{E_\Lambda}\left(1 + \frac{\epsilon_p}{E_\Lambda}\right)
\end{equation}
Тогда $\psi_\Lambda(0) \approx 1$, а 
$\psi_\Lambda(n) \approx \frac{C_n}{E_\Lambda}$, где $C_n$ --- коэффициенты из
(\ref{hamiltonian}). Таким образом, если $C_n$ отличны от нуля только для ближайших
атомов, то волновая функция для всех остальных атомов обращается в нуль (в этом приближении).
 \end{document}
