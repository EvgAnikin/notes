\documentclass{article}

\usepackage{amsmath}
\usepackage[english, russian]{babel}
\usepackage[utf8]{inputenc}
\usepackage[T2A, T1]{fontenc}

\title{Краевые состояния в одномерной цепочке}
\author{Евгений Аникин}

\begin{document}

\maketitle
В этом листке я искал краевые состояния у двух разных полубесконечных цепочек. Одна 
из них состоит из различных атомов, соединённых одинаковыми связями. Другая, наоборот, ---
из одинаковых атомов, но с разными связями. В результате получилось, что у цепочки первого
типа краевого состояния нет, а у цепочки второго типа его наличие зависит от типа разорванной
связи.

\section{Цепочка с разными связями}
\subsection{Бесконечная в обе стороны цепочка}
Рассмотрим цепочку из атомов, где связи между атомами чередуются по величине.
\begin{equation}
	H = \sum_n t_1(a_n^\dagger b_n + a_n b_n^\dagger) + 
		t_2 (a_{n+1}^\dagger b_n + a_{n+1} b_n^\dagger)
\end{equation}
После такого же преобразования Фурье, как и в предыдущем случае, гамильтониан примет вид
\begin{equation}
	H = \sum_p 
			(t_1 e^{-\frac{ipa}{2}} + t_2 e^{\frac{ipa}{2}}) a_p^\dagger b_p + 
			(t_1 e^{\frac{ipa}{2}} + t_2 e^{-\frac{ipa}{2}}) a_p b_p^\dagger
\end{equation}
Собственные значения гамильтониана ---
\begin{equation}
	E_p^{1,2} = \pm \epsilon_p = \pm|t_1 e^{\frac{ipa}{2}} + t_2 e^{-\frac{ipa}{2}}| = 
		\pm\sqrt{t_1^2 + t_2^2 + 2t_1t_2 \cos{pa}}
\end{equation}
Введём обозначение 
\begin{equation}
	e^{i\phi} = \frac{t_1 e^{-\frac{ipa}{2}} + t_2 e^{\frac{ipa}{2}}}
				{\sqrt{t_1^2 + t_2^2 + 2t_1t_2 \cos{pa}}}
\end{equation}
Тогда гамильтониан диагонализуется преобразованием
\begin{equation}
	\left(
	\begin{matrix}
		a_p \\
		b_p
	\end{matrix}
	\right)
	=
	\frac12
	\left(
	\begin{matrix}
		1 & -e^{i\phi}	\\
		e^{-i\phi} & 1
	\end{matrix}
	\right)
	\left(
	\begin{matrix}
		x_p \\
		y_p
	\end{matrix}
	\right)
\end{equation}
Отсюда получаются выражения для функций Грина в импульсном представлении:
\begin{equation}
	\label{gaa}
	G^R_0 (\omega, p, A, A) =  G^R_0 (\omega, p, B, B) = \frac{1}{2}\left(
		\frac{1}{\omega - \epsilon_p + i\delta} + 
					\frac{1}{\omega + \epsilon_p + i\delta}\right)
\end{equation}
\begin{equation}
	G^R_0 (\omega, p, A, B) = \frac12 \left(\frac{e^{i\phi}}{\omega - \epsilon_p + i\delta} -
						\frac{e^{i\phi}}{\omega + \epsilon_p + i\delta} \right)
\end{equation}
\begin{equation}
	G^R_0 (\omega, p, B, A) = \frac12 \left(\frac{e^{-i\phi}}{\omega - \epsilon_p + i\delta} -
						\frac{e^{-i\phi}}{\omega + \epsilon_p + i\delta} \right)
\end{equation}
Здесь аргументы $A$, $B$ соответствуют операторам $a_p$ и $b_p$. 
\subsection{Полубесконечная цепочка}
Как и в прошлый раз, введём возмущение $V = \Delta E a_0^\dagger a_0$, где $\Delta E$ очень велико.
Это приведёт к совершенно такому же, как и раньше, уравнению Дайсона. Решение уравнения Дайсона ---
\begin{multline}
	G^R(\omega, m, s, n, s') = 
		G^R_0(\omega, m,s,n,s') - 
		\frac{G^R_0(\omega, m, s, 0, A)G^R_0(\omega, 0, A, n, s')}{G^R_0(\omega, 0, A, 0, A)},\\
		s,s' = A,B
\end{multline}
Уровни энергии даются, как видно, нулями функции $G^R_0(\omega, 0,A,0,A)$, а волновая функция
связанного состояния пропорциональна $G^R_0(\omega, m,s,0,A)$.

В нашем случае, как нетрудно убедиться, этот уровень энергии --- $E = 0$. Действительно, подставим 
$\omega = 0$ в (\ref{gaa}). Получается
\begin{equation}
	G^R_0 (\omega, p, A, A) = 
		-\frac{1}{2\epsilon_p} + 
					\frac{1}{2\epsilon_p} = 0
\end{equation}
Отсюда следует, что функции Грина в узельном представлении, 
составленные из операторов $a$, обращаются в нуль для всех $m,n$, и, таким образом, волновая 
функция связанного состояния равна нулю во всех узлах $a$ (на мой взгляд, это довольно
удивительно).

Чтоы найти волновую функцию в узлах $b$, вычислим функцию Грина $G_0^R(0,m,B,0,A)$.
\begin{multline}
	G_0^R(0,m,B,0,A) = -\int_{-\pi}^{\pi} \frac{dk}{2\pi} 
			\frac{e^{-ik(m + \frac12)}e^{-i\phi}}{\epsilon_p} = \\
			= -\int_{-\pi}^{\pi} \frac{dk}{2\pi} 
			\frac{t_1e^{-ikm} + t_2e^{-ik(m+1)}}{t_1^2 + t_2^2 + 2t_1t_2 \cos{k}} 
\end{multline}
Сделаем сдвиг переменной интегрирования $k \to k + \pi$. Получится
\begin{equation}
	G_0^R(0,m,B,0,A) = (-1)^{m+1}\int_{-\pi}^{\pi} \frac{dk}{2\pi}
		\frac{t_1e^{-ikm} - t_2e^{-ik(m+1)}}{t_1^2 + t_2^2 - 2t_1t_2 \cos{k}} 
\end{equation}
Последний интеграл берётся, как и в прошлом листке. Полюс определяется уравнением
\begin{equation}
	\cos{k} = \frac12 \left(\frac{t_1}{t_2} + \frac{t_2}{t_1}\right)
\end{equation}
Для определённости будем считать, что $t_1 > t_2$. Тогда 
\begin{equation}
	k = \pm i\log\frac{t_1}{t_2}
\end{equation}
После несложных преобразований получим
\begin{equation}
	G_0^R(0,m,B,0,A) =
	\left\{
	\begin{matrix}
		(-1)^{m+1} \frac{t_2^m}{t_1^{m+1}} \quad \mbox{при} \quad m \ge 0 \\
		0 \quad \mbox{при} \quad m < 0
	\end{matrix}
	\right.
\end{equation}
Получается, что при $t_1 > t_2$ краевое состояние есть только у правой половины цепочки.
Его волновая функция ---
\begin{equation}
	\psi(n, B) = \sqrt{1 - \left(\frac{t_2}{t_1}\right)^2} \left(\frac{t_2}{t_1}\right)^n
\end{equation}

\subsection{Цепочка с одинаковыми связями}
\subsubsection{Бесконечная цепочка}
Рассмотрим цепочку с чередующимися потенциалами атомов и одинаковыми связями.
\begin{equation}
	H = \sum_n \xi(a_n^\dagger a_n - b_n^\dagger b_n) + ta_n^\dagger(b_n + b_{n-1}) +
			ta_n(b_n^\dagger + b_{n-1}^\dagger)
\end{equation}
Сделаем преобразование Фурье:
\begin{equation}
	a_n = \frac{1}{\sqrt{N}} \sum e^{-ipan} a_p
\end{equation}
\begin{equation}
	b_n = \frac{1}{\sqrt{N}} \sum e^{-ipa(n+\frac{1}{2})} b_p
\end{equation}
После преобразования Фурье гамильтониан примет вид
\begin{equation}
	H = \sum_p \xi (a_p^\dagger a_p - b_p^\dagger b_p) + 2t\cos{\frac{pa}{2}} a_p^\dagger b_p
			+ 2t\cos{\frac{pa}{2}}a_pb_p^\dagger
\end{equation}
Гамильтониан, таким образом, задаётся матрицей
\begin{equation}
	\left(\begin{matrix}
		\xi & 2t\cos{\frac{pa}{2}} \\
		2t\cos{\frac{pa}{2}} & -\xi
	\end{matrix}\right)
\end{equation}
Собственные значения энергии ---
\begin{equation}
	E_p = \pm \epsilon_p = \pm \sqrt{\xi^2 + 4t^2 \cos^2{\frac{pa}{2}}}
\end{equation}
Введём обозначение $\cos\alpha = \xi/\epsilon_p$. Тогда матрица гамильтониана запишется 
в виде
\begin{equation}
	\epsilon_p \left(
		\begin{matrix}
			\cos{\alpha} & \sin{\alpha} \\
			\sin{\alpha} & -\cos{\alpha} 
		\end{matrix}
		\right)
\end{equation}
Это --- матрица отражения, и её собственные векторы очевидны: 
$(\cos{\frac{\alpha}{2}}, \sin{\frac{\alpha}{2}})$ и 
$(-\sin{\frac{\alpha}{2}}, \cos{\frac{\alpha}{2}})$.
Гамильтониан можно теперь диагонализовать преобразованием
\begin{equation}
	\left(
	\begin{matrix}
		a_p \\
		b_p
	\end{matrix}
	\right)
	=
	\frac12
	\left(
	\begin{matrix}
		\cos{\frac{\alpha}{2}} & -\sin{\frac{\alpha}{2}} \\
		\sin{\frac{\alpha}{2}} & \cos{\frac{\alpha}{2}}
	\end{matrix}
	\right)
	\left(
	\begin{matrix}
		x_p \\
		y_p
	\end{matrix}
	\right)
\end{equation}
Оператор $x_p$ рождает состояние с положительной энергией, а $y_p$ --- с отрицательной.
После этого легко вычислить функции Грина операторов $a_p$, $b_p$, они получаются такими:
\begin{equation}
	\label{first}
	G_0^R(\omega, p, A, A) = \frac{\cos^2{\frac{\alpha}{2}}}{\omega - \epsilon_p + i\delta}+
				\frac{\sin^2{\frac{\alpha}{2}}}{\omega + \epsilon_p + i\delta}
\end{equation}
\begin{equation}
	G_0^R(\omega, p, A, B) =  G_0^R(\omega, p, B, A)
			= \frac{\sin{\frac{\alpha}{2}}\cos{\frac{\alpha}{2}}}
					{\omega - \epsilon_p + i\delta}+
				\frac{\sin{\frac{\alpha}{2}}\cos{\frac{\alpha}{2}}}
					{\omega + \epsilon_p + i\delta}
\end{equation}
\begin{equation}
	\label{last}
	G_0^R(\omega, p, B, B) = \frac{\sin^2{\frac{\alpha}{2}}}{\omega - \epsilon_p + i\delta}+
				\frac{\cos^2{\frac{\alpha}{2}}}{\omega + \epsilon_p + i\delta}
\end{equation}
Каждое из двух слагаемых в уравнениях выше происходит от функций Грина 
$\langle \mathop{T_\tau} x_p(\tau) x_p^\dagger(\tau') \rangle$, 
$\langle \mathop{T_\tau} y_p(\tau) y_p^\dagger(\tau') \rangle$.

Функции Грина (\ref{first}) -- (\ref{last}) можно переписать в более удобном виде:
\begin{equation}
	G_0^R(\omega, p, A, A) = \frac{\omega + \xi}{\omega^2 - \epsilon_p^2 + i\delta}
\end{equation}
\begin{equation}
	G_0^R(\omega, p, A, B) = \frac{2t\cos{pa}}{\omega^2 - \epsilon_p^2 + i\delta}
\end{equation}
\begin{equation}
	G_0^R(\omega, p, B, B) = \frac{\omega - \xi}{\omega^2 - \epsilon_p^2 + i\delta}
\end{equation}
\subsubsection{Полубесконечная цепочка}
Как и раньше, введём возмущение $V = \Delta E a_0^\dagger a_0$. Уровни энергии снова 
будут определяться уравнением
$G_0^R(\omega,0,A,0,A) = 0 $.
Имеем 
\begin{multline}
	G_0^R(\omega,0,A,0,A) = \int_{-\pi}^{\pi} \frac{dk}{2\pi}
			\frac{\omega + \xi}{\omega^2 - \epsilon_k^2 + i\delta} =  \\
					=\int_{-\pi}^{\pi} \frac{dk}{2\pi} 
			\frac{\omega + \xi}{\omega^2 - \xi^2 - 2t^2 - 2t^2\cos{pa} + i\delta} 
\end{multline}
Интеграл обращается в ноль при $\omega = -\xi$, на границе непрерывного спектра. Внутри
запрещённой зоны, то есть в области $|\omega| < \xi$, интеграл строго отрицателен. Отсюда
можно заключить, что в этом случае граничного состояния не существует: никаких новых
полюсов у функции Грина не возникло.

\end{document}
