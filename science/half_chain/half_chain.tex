\documentclass{article}
\usepackage{amsmath}
\usepackage{graphicx}
\usepackage[utf8]{inputenc}
\usepackage[T1, T2A]{fontenc}
\usepackage[english,russian]{babel}

\title{Половина одномерной цепочки в подходе сильной связи}
\author{Евгений Аникин}

\begin{document}
\maketitle
\section{Функция Грина одномерной цепочки}
Пусть в гамильтониане сильной связи есть только переходы между соседними атомами.
Тогда энергия состояния с квазиимпульсом $p$ --- 
\begin{equation}
	\epsilon_p = E + 2C\cos{pa}
\end{equation}
Гамильтониан же выглядит так:
\begin{equation}
	\hat{H} = \sum_n Ew_n^\dagger w_n + Cw_{n+1}^\dagger w + Cw_{n-1}^\dagger w
\end{equation}
Для функции Грина в этом случае может быть получена явная формула.
\begin{equation}
	G_0^R(\omega, m,n) = 
			\frac{1}{N}\sum_p  \frac{e^{-ip(R_m - R_n)}}{\omega - \epsilon_p + i\delta} = 
			\int_{-\pi}^{\pi} 
				\frac{dk}{2\pi} \frac{e^{-ik(m-n)}}{\omega - E - 2C \cos{k} + i\delta}
\end{equation}
Последний интеграл берётся с помощью вычетов, в конечном итоге получаем ответ:
\begin{equation}
	G_0^R(\omega, m,n) = \frac{e^{-(m-n)\kappa}}{2C\sinh{\kappa}}, \qquad
			\cosh \kappa = \frac{\omega  - E}{C}
\end{equation}
\section{Примесь с бесконечной энергией}
Добавим к гамильтониану возмущение
\begin{equation}
	V = \Delta E w_0^\dagger w_0
\end{equation}
Функция Грина в этом случае ---
\begin{equation}
	G^R(\omega, m,n) = G^R_0(\omega, m,n) + 
		\frac{\Delta EG^R_0(\omega, m,0)G^R_0(\omega, 0, n)}{1 - \Delta E G^R_0(\omega,0,0)}
\end{equation}
Если $\Delta E$ очень велика, то получится
\begin{multline}
	G^R(\omega, m,n) = G^R_0(\omega, m,n) - 
		\frac{G^R_0(\omega, m,0)G^R_0(\omega, 0, n)}{G^R_0(\omega,0,0)} = \\
		= \frac{e^{-|m-n|\kappa} - e^{-|m|\kappa- |n|\kappa}}{2C\sinh \kappa}
\end{multline}
Видно, что $G^R$ отлична от нуля только в случае, если $m$ и $n$ одного знака. Это значит, 
что левая и правая половины цепочки стали отделены друг от друга.
\section{Разорванная связь}
Пусть теперь возмущение
\begin{equation}
	V = -C(w_1^\dagger w_0 + w_0^\dagger w_1)
\end{equation}
Это приведёт к разрыву цепочки. Уравнение Дайсона для такого возмущения ---
\begin{equation}
	\mathcal{G}(\omega, m,n) = \mathcal{G}_0(\omega, m,n) - 
			C\mathcal{G}_0(\omega, m, 0) \mathcal{G}(\omega, 1, n) -
			C\mathcal{G}_0(\omega, m, 1) \mathcal{G}(\omega, 0, n)
\end{equation}
Такому же уравнению удовлетворяет и запаздывающая функция Грина. 
\begin{multline}
	G^R(\omega, m,n) = G^R_0(\omega, m,n) - 
			CG^R_0(\omega, m, 0) G^R(\omega, 1, n) - \\
			-CG^R_0(\omega, m, 1) G^R(\omega, 0, n)
\end{multline}
Так как левая и правая части цепочки теперь никак между собой не связаны, должно быть
\begin{equation}
	\label{condition}
	G^R(\omega, 0, 1) = 0
\end{equation}
Пользуясь этим условием, легко решить уравнение Дайсона. В процессе решения последовательно
получаются выражения
\begin{equation}
	G^R(\omega,0,0) =G^R(\omega,1,1) = \frac{e^{-\kappa}}{C}
\end{equation}
\begin{equation}
	G^R(\omega, n,1) = \frac{e^{-\kappa|n-1|} - e^{-(|n|+1)\kappa}}{2C\sinh\kappa}
\end{equation}
\begin{equation}
	G^R(\omega, n,0) = \frac{e^{-\kappa|n|} - e^{-(|n-1|+1)\kappa}}{2C\sinh\kappa}
\end{equation}
\begin{equation}
		G^R(\omega,m,n)= \frac{e^{-|m-n|\kappa} - e^{-(m+n)\kappa}}{2C\sinh \kappa}
			\quad \mbox{для} \quad m,n > 0
\end{equation}
\begin{equation}
		G^R(\omega,m,n)= \frac{e^{-|m-n|\kappa} - e^{-(|m-1|+|n-1|)\kappa}}{2C\sinh \kappa}
			\quad \mbox{для} \quad m,n < 0
\end{equation}
\begin{equation}
		G^R(\omega,m,n)= 0
			\quad \mbox{для } m,n \mbox{ по разные стороны от нуля}
\end{equation}
Функция Грина получается точно такая же, как в предыдущем параграфе. 
\end{document}
