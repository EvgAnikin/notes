\documentclass{article}
\usepackage{amsmath}
\usepackage{amsfonts}
\usepackage{graphicx}
\usepackage[utf8]{inputenc}
\usepackage[T1, T2A]{fontenc}
\usepackage[english,russian]{babel}

\title{Парамагнетизм Паули \\и диамагнетизм Ландау}
\author{Anikin Evgeny, 128}

\begin{document}
\maketitle
\section{Задача}
Задача заключалась в том, чтобы найти температурную зависимость отношения 
восприимчивостей Ландау и Паули. 

\emph{Ответ:} отношение восприимчивостей не зависит от температуры.
\begin{equation}
    \frac{\chi_\mathrm{dia}}{\chi_\mathrm{para}} = -\frac{1}{3}
\end{equation}
\section{Парамагнетизм}
Если пренебречь квантованием орбитального движения электрона в магнитном поле, то
термодинамический потенциал электронного газа представляется в виде
\begin{equation}
    \Omega = \Omega_0(\mu - \mu_B H) + \Omega_0(\mu + \mu_B H),
\end{equation}
где $\Omega_0$ --- термодинамический потенциал бесспиновых фермионов. Разложим $\Omega$ по 
малым $H$: 
\begin{equation}
    \Omega = 2\Omega_0 + \mu_B^2 H^2 \frac{\partial^2 \Omega_0}{\partial \mu^2} = 
            2\Omega_0 - \mu_B^2 H^2 \frac{\partial N}{\partial \mu}
\end{equation}
Восприимчивость --- 
\begin{equation}
    \chi_{\mathrm{para}} = -\frac{\partial^2 \Omega}{\partial H^2} =  
        2\mu_B^2 \frac{\partial N}{\partial \mu}
\end{equation}
Преобразуем:
\begin{equation}
    \frac{\partial N}{\partial \mu} = 
        \frac{\partial}{\partial \mu}
              \int \frac{d^3 p}{(2\pi\hbar)^3}\frac{1}{1 + e^{\frac{\epsilon - \mu}{T}}} =
        \frac{\nu_F}{\sqrt{\mu}} 
                \int \frac{\sqrt{\epsilon}\, e^{\frac{\epsilon-\mu}{T}} \,d\epsilon}
                          {T\left(1 + e^{\frac{\epsilon-\mu}{T}}\right)^2}
\end{equation}
Таким образом,
\begin{equation}
    \label{para}
    \chi_{\mathrm{para}} = 2\nu_F\mu_B^2
        \frac{1}{\sqrt{\mu}} 
                \int \frac{\sqrt{\epsilon}\, e^{\frac{\epsilon-\mu}{T}} \,d\epsilon}
                          {T\left(1 + e^{\frac{\epsilon-\mu}{T}}\right)^2}
\end{equation}
\section{Диамагнетизм}
Уровни энергии электрона в магнитном поле (будем считать его параллельным $Oz$) ---
\begin{equation}
    E_{n,p_z,\sigma} = 2\mu_B H \left(n + \frac{1}{2}\right) + 
                       \frac{p_z^2}{2m}
\end{equation}
Магнетон Бора ---
\begin{equation}
    \mu_B = \frac{e\hbar}{2mc},
\end{equation}
кратность вырождения уровня Ландау ---
\begin{equation}
    N = \frac{eHL_xL_y}{2\pi\hbar}
\end{equation}
Значит, $\Omega$--потенциал --- 
\begin{equation}
    \Omega = -T\sum_{n,\sigma} \frac{eHV}{2\pi\hbar c} 
                \int_{-\infty}^{\infty} \frac{dp_z}{2\pi\hbar}
                \log{\left\{1 + \exp{\left(\frac{\mu - E_{n,p_z}}{T}\right)}\right\}}
\end{equation}
Сумму по $n$ можно преобразовать в интеграл с помощью формулы
\begin{equation}
    \int_0^L f(x) dx = h\sum_{k=0}^{N-1} f(h(k+{\textstyle \frac12}))
                       +{\textstyle\frac{h^2}{24}} \left(f'(L) - f'(0)\right) + O(h^4)
\end{equation}
В результате получается
\begin{equation}
    \Omega = \Omega_0 + \frac{m}{2\pi^2\hbar^3} \frac{\mu_B^2 H^2}{12}
                \int\frac{dp_z}{1 + \exp{\displaystyle\left(\frac{\epsilon_{p_z} - \mu}{T}\right)}} 
\end{equation}
Если перейти к интегрированию по $\epsilon$ и проинтегрировать по частям, получится
\begin{equation}
    \Omega = \Omega_0 + \frac{\nu_F \mu_B^2 H^2}{6} \frac{1}{\sqrt{\mu}}
                \int \frac{\sqrt{\epsilon}\, e^{\frac{\epsilon-\mu}{T}} \,d\epsilon}
                          {T\left(1 + e^{\frac{\epsilon-\mu}{T}}\right)^2}
\end{equation}
Отсюда получаем восприимчивость (лишняя двойка берётся от двух проекций спина):
\begin{equation}
    \chi_{\mathrm{dia}} = -\frac{\partial^2 \Omega}{\partial H^2} = 
                -\frac{2\nu_F \mu_B^2}{3} \frac{1}{\sqrt{\mu}}
                \int \frac{\sqrt{\epsilon}\, e^{\frac{\epsilon-\mu}{T}} \,d\epsilon}
                          {T\left(1 + e^{\frac{\epsilon-\mu}{T}}\right)^2}
\end{equation}
Сравнивая с \eqref{para}, получим
\begin{equation}
    \frac{\chi_\mathrm{dia}}{\chi_\mathrm{para}} = -\frac{1}{3}
\end{equation}
\end{document}
