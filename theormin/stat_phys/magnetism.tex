\documentclass{article}
\usepackage{amsmath}
\usepackage{graphicx}
\usepackage[utf8]{inputenc}
\usepackage[T1, T2A]{fontenc}
\usepackage[english,russian]{babel}

\title{Парамагнетизм Паули \\и диамагнетизм Ландау}
\author{Anikin Evgeny, 128}

\begin{document}
\maketitle
Уровни энергии электрона в магнитном поле (будем считать его параллельным $Oz$) ---
\begin{equation}
    E_{n,p_z,\sigma} = 2\mu_B B \left(n + \frac{1}{2}\right) + 
                       \frac{p_z^2}{2m} + \sigma \mu_B B
\end{equation}
Магнетон Бора ---
\begin{equation}
    \mu_B = \frac{e\hbar}{2mc},
\end{equation}
кратность вырождения уровня Ландау ---
\begin{equation}
    N = \frac{eBL_xL_y}{2\pi\hbar}
\end{equation}
Значит, $\Omega$--потенциал --- 
\begin{equation}
    \Omega = \sum_{n,\sigma} \frac{eBV}{2\pi\hbar c} 
                \int_{-\infty}^{\infty} \frac{dp_z}{2\pi\hbar}
                \log{\left\{1 + \exp{\left(\frac{\mu - E_{n,p_z,\sigma}}{T}\right)}\right\}}
\end{equation}
Сумму по $n$ можно преобразовать в интеграл с помощью формулы
\begin{equation}
    \int_0^L f(x) dx = h\sum_{k=0}^{N-1} f(h(k+{\textstyle \frac12}))
                       +{\textstyle\frac{h^2}{24}} \left(f'(L) - f'(0)\right) + O(h^4)
\end{equation}
В результате получается
\begin{multline}
    \Omega = -T\sum_\sigma\frac{mV}{2\pi\hbar^2} \int_{-\infty}^{\infty} 
                    \frac{dp_z}{2\pi\hbar}\int_0^\infty d\epsilon 
                           \log{\left\{1 + \exp{\left(\frac{\mu - \epsilon - 
                                \frac{p_z^2}{2m}-\sigma\mu_B B}{T} \right)} \right\}}
\end{multline}
\end{document}
