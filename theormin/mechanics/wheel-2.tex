\documentclass{article}
\usepackage{amsmath}
\usepackage{graphicx}
\usepackage[utf8]{inputenc}
\usepackage[T1, T2A]{fontenc}
\usepackage[english,russian]{babel}

\title{Задачи к теорминимуму}
\author{Anikin Evgeny, 121}

\begin{document}
\maketitle
\section{Уточнение задачи про эллипс}
В старых обозначениях можно написать такие формулы:
\begin{equation}
	\xi = r(\alpha - \beta) \sin\beta
\end{equation}
\begin{equation}
	y = r(\alpha - \beta) \cos\beta
\end{equation}
\begin{equation}
	x = l(\alpha - \beta) + \xi
\end{equation}
\begin{equation}
	\label{beta}
	\tan \beta = -\frac1r \frac{dr}{d\theta} (\alpha - \beta)
\end{equation}
Ещё есть полезное соотношение
\begin{equation}
	\frac{dx}{d\alpha} = y
\end{equation}
Радиус от угла зависит так:
\begin{equation}
	r(\theta) = R\left( 
		\frac{\cos^{2}{\theta}}{1 + {\displaystyle \frac{\epsilon}{2}}} 
		+ \frac{\sin^{2}{\theta}}{1 - {\displaystyle \frac{\epsilon}{2}}} 
		\right)^{-1}
\end{equation}
Разложение до второго порядка по $\epsilon$:
\begin{equation}
	r^2(\theta) = R^2\left(1 +\epsilon\cos{2\theta} - 
		\frac{\epsilon^2}{4} + \frac{\epsilon^2}{2}\cos{4\theta}\right)
\end{equation}
\begin{equation}
	r(\theta) = R\left ( 1 + \frac{\epsilon}2 \cos{2\theta}
						- \frac{3}{16} \epsilon^2
						+ \frac{3}{16} \epsilon^2 \cos{4\theta} \right )
\end{equation} 
Решая уравнение \ref{beta} во втором порядке, получим
\begin{equation}
	\beta = \epsilon \sin{2\alpha}
\end{equation}
Теперь, используя формулы, можно найти $x$, $y$ и $\xi$.
\begin{equation}
\end{equation}
\begin{equation}
\end{equation}
\begin{equation}
\end{equation}
\end{document}
