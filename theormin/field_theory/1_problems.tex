\documentclass{article}
\usepackage{amsmath}
\usepackage{graphicx}
\usepackage[utf8]{inputenc}
\usepackage[T1, T2A]{fontenc}
\usepackage[english,russian]{babel}

\title{Задачи к теорминимуму}
\author{Anikin Evgeny, 121}

\begin{document}
\maketitle
\section{Вторая задача}
\subsection{Условие задачи}
В метрике
\begin{equation}
ds^2 = \left(1 - \frac{r_g}{r} + \frac{Z^2}{r^2}\right)dt^2
	- \frac{dr^2}{1 - \cfrac{r_g}{r} + \cfrac{Z^2}{r^2}}
	- r^2 d\phi^2
\end{equation}
найти малое отклонение луча света от прямолинейности.
\subsection{Ответ}

\subsection{Решение}
Уравнение эйконала для луча света выгладит так:
\begin{equation}
\frac{1}{\left(1 - \cfrac{r_g}{r} + \cfrac{Z^2}{r^2}\right)}
			\left(\frac{\partial S}{\partial t}\right)^2 -
\left(1 - \frac{r_g}{r} + \frac{Z^2}{r^2}\right)
			\left(\frac{\partial S}{\partial r}\right)^2
- \frac{1}{r^2} 
			\left(\frac{\partial S}{\partial \phi}\right)^2 = 0
\end{equation}
Ищем решение в виде 
\begin{equation}
	S = -Et + S_r(R) + M\phi
\end{equation}
Тогда для $S_r(R)$ получается выражение
\begin{equation}
S_r(R) = \int_{r_0}^{R} {d} r\sqrt{
		\frac{E^2}{\left( 1 - \frac{r_g}{r} + \frac{Z^2}{r^2} \right)^2} - 
		\frac{M^2}{r^2 \left( 1 - \frac{r_g}{r} + \frac{Z^2}{r^2}\right)}
		}	
\end{equation}
Здесь $r_0$, ближайшая к центру точка траектории луча, --- ноль подкоренного выражения.

Подкоренное выражение можно разложить до членов второго порядка. Получается
так:
\begin{equation}
S_r = \int {d} r\sqrt{
		E^2\left(1 + \frac{2r_g}{r} + \frac{3r_{g}^{2} - 2Z^2}{r^2}\right)
		-\frac{M^2}{r^2}\left(1 + \frac{r_g}{r} + \frac{r_g^2 - Z^2}{r^2}\right)
	}
\end{equation}
Чтобы в дальнейшем можно было успешно раскладывать корень, нужно найти ноль подкоренного
выражения. С точностью до членов второго порядка он равен
\begin{equation}
r_0 = \frac{M}{E} - \frac{r_g}{2} - \frac{E}{2M}\left(\frac{3}{4}r_g^2 - Z^2\right)
\end{equation}
Теперь сдвинем переменную интегрирования:
\begin{equation}
	r = \rho - \frac{r_g}{2} - \frac{E}{2M}\left(\frac{3}{4}r_g^2 - Z^2\right)
\end{equation}
После подстановки интеграл примет вид 
\begin{equation}
S_r = \int\limits_{M/E}^{R'}%
		d\rho \sqrt{
		\left(E^2 - \frac{M^2}{\rho^2}\right)
			\left(1 + \frac{2r_g}{\rho} + \frac{4r_g^2 - 2Z^2}{\rho^2}\right)
		-\frac{M}{\rho^3}\left( \frac{3}{4}r_g^2 - Z^2\right) \left(E - \frac{M}{\rho}\right)
	}
\end{equation}
\begin{equation}
	R' = R + \frac{r_g}{2} +\frac{E}{2M}\left(\frac{3}{4}r_g^2 - Z^2\right) 
\end{equation}
Теперь можно разложить корень. Получается так:

\begin{multline}
\label{integral}
S_r = \int\limits_{M/E}^{R'}
		d\rho \left\{ 
		\sqrt{E^2-\frac{M^2}{\rho^2}} + \frac{r_g}{\rho} 
						\sqrt{E^2-\frac{M^2}{\rho^2}}+\right. \\ 
			\left. \frac{3r_g^2 - 2Z^2}{2\rho^2}	\sqrt{E^2-\frac{M^2}{\rho^2}} - 
			\left( \frac{3}{8} r_g^2 - \frac{1}{2}Z^2\right) \frac{M}{\rho^3} 
						\sqrt{\frac{\rho E - M}{\rho E + M}}
		\right\} 
\end{multline}
Изменение полярного угла при движении от ближайшей точки до бесконечности даётся формулой
\begin{equation}
\phi = -\lim_{R \to \infty}\frac{\partial}{\partial M} S_r(R)
\end{equation}
В интеграле \ref{integral} первое и второе слагаемые расходятся, поэтому
важно, что сначала берётся интеграл, а затем --- предел. Третье и четвёртое
слагаемые сходятся. Кроме того, в первом слагаемом важно отличие $R$ 
от $R'$.
\paragraph{Первое слагаемое:}
\paragraph{Второе слагаемое:}
\paragraph{Третье слагаемое:}

\paragraph{Четвёртое слагаемое:}
\end{document}
