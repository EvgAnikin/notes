\documentclass{article}
\usepackage{amsmath}
\usepackage{graphicx}
\usepackage[utf8]{inputenc}
\usepackage[T1, T2A]{fontenc}
\usepackage[english,russian]{babel}

\title{Задачи к теорминимуму}
\author{Anikin Evgeny, 121}

\begin{document}
\maketitle
\section{Вторая задача}
\subsection{Условие задачи}
В метрике
\begin{equation}
ds^2 = \left(1 - \frac{r_g}{r} + \frac{Z^2}{r^2}\right)dt^2
	- \frac{dr^2}{1 - \cfrac{r_g}{r} + \cfrac{Z^2}{r^2}}
	- r^2 d\phi^2
\end{equation}
найти малое отклонение луча света от прямолинейности.
\subsection{Ответ}

\subsection{Решение}
Уравнение эйконала для луча света выгладит так:
\begin{equation}
\frac{1}{\left(1 - \cfrac{r_g}{r} + \cfrac{Z^2}{r^2}\right)}
			\left(\frac{\partial S}{\partial t}\right)^2 -
\left(1 - \frac{r_g}{r} + \frac{Z^2}{r^2}\right)
			\left(\frac{\partial S}{\partial r}\right)^2
- \frac{1}{r^2} 
			\left(\frac{\partial S}{\partial \phi}\right)^2 = 0
\end{equation}
Ищем решение в виде 
\begin{equation}
	S = -Et + S_r(r) + M\phi
\end{equation}
Тогда для $S_r(r)$ получается выражение
\begin{equation}
S_r = \int \mathrm{d} r\sqrt{
		\frac{E^2}{\left( 1 - \frac{r_g}{r} + \frac{Z^2}{r^2} \right)^2} - 
		\frac{M^2}{r^2 \left( 1 - \frac{r_g}{r} + \frac{Z^2}{r^2}\right)}
		}	
\end{equation}
Подкоренное выражение можно разложить по 
\section{Первая задача}
\subsection{Условие задачи}
Для лагранжиана
\begin{equation}
	\label{lagr}
	L = -a^2\sqrt{-\mathrm{det}\left(\eta + \frac{F}{a}\right)} + a^2 + A_\mu j^\mu
\end{equation}
найти поле точечного заряда.
\subsection{Ответ}
\begin{equation}
	E = - \frac{q}{\sqrt{(4\pi r^2)^2 + (q/a)^2}}
\end{equation}
\subsection{Решение}
Детерминант под корнем формулы \ref{lagr} можно преобразовать: прямое вычисление даёт 
\begin{multline}
\mathrm{det}\left(\eta + \frac{F}{a}\right) = 
\frac{1}{a^4}\mathrm{det}\begin{pmatrix}
a & E_1 & E_2 & E_3 \\ 
-E_1 &-a & -B_3 & B_2 \\
-E_2 & B_3 &-a    &-B_1 \\
-E_3 & -B_2 & B_1 &-a \\
\end{pmatrix} = \\
= -1 + \frac{1}{a^2} (\vec{E}^2 - \vec{B}^2) + \frac{1}{a^4} (\vec{E}, \vec{B})^2
\end{multline}
Соответственно, лагранжиан принимает вид
\begin{equation}
	L = -a^2 \sqrt{1 - \frac{1}{a^2} 
		(\vec{E}^2 - \vec{B}^2) - \frac{1}{a^4} (\vec{E}, \vec{B})^2} 
			\,+ a^2 + A_\mu j^\mu
\end{equation}
В переменных $\vec{E}$, $\vec{B}$ уравнения Лагранжа примут вид
\begin{equation}
	\label{dive}
	-\partial_k \frac{\partial L}{\partial E_k} = j^0
\end{equation}
\begin{equation}
	\label{roth}
	\partial_{0}\frac{\partial L}{\partial E_k} + 
	\epsilon_{ijk}\partial_{i}\frac{\partial L}{\partial B_j} = j^k
\end{equation}
Частные производные:
\begin{equation}
	\frac{\partial L}{\partial E_k} = 
		\frac{E_k + \frac{1}{a^2}(\vec{E}, \vec{B})B_k}
		{\sqrt{
		1 - 
		\frac{1}{a^2}(\vec{E}^2 - \vec{B}^2) - 
		\frac{1}{a^4} (\vec{E}, \vec{B})^2}
		}
\end{equation}

\begin{equation}
	\frac{\partial L}{\partial B_k} = 
		\frac{-B_k + \frac{1}{a^2}(\vec{E}, \vec{B})E_k}
		{\sqrt{
		1 - 
		\frac{1}{a^2}(\vec{E}^2 - \vec{B}^2) - 
		\frac{1}{a^4} (\vec{E}, \vec{B})^2}
		}
\end{equation}

Теперь найдём поле точечного заряда. Это будет сферически симметричное решение с $\vec{B} = 0$.
В этом случае уравнение \ref{roth} выполняется автоматически, а уравнение \ref{dive} принимает 
вид 
\begin{equation}
	\partial_k \frac{E_k}{\sqrt{1 - \frac{1}{a^2} \vec{E}^2}} = -q\delta^{3}(x)
\end{equation}
Теперь ясно, что
\begin{equation}
	\frac{E}{\sqrt{1 - \frac{1}{a^2} {E}^2}} = -\frac{q}{4\pi r^2}
\end{equation}
и, окончательно,
\begin{equation}
	E = - \frac{q}{\sqrt{(4\pi r^2)^2 + (q/a)^2}}
\end{equation}
\end{document}
