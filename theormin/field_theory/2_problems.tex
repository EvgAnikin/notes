\documentclass{article}
\usepackage{amsmath}
\usepackage{graphicx}
\usepackage[utf8]{inputenc}
\usepackage[T1, T2A]{fontenc}
\usepackage[english,russian]{babel}

\title{Задачи к теорминимуму}
\author{Anikin Evgeny, 121}

\begin{document}
\maketitle
\section{Третья задача}
\subsection{Условие задачи}
Найти излучение системы зарядов с точностью до членов порядка $c^{-5}$.
\subsection{Ответ}
\begin{equation}
	W = \frac{2\ddot{\mathbf{d}^2}}{3c^3} + 
			\frac{\dddot{D^{kl}}\dddot{D^{kl}}}{180c^5} + \frac{2\ddot{M^2}}{3c^3} + 
			\frac{4}{15c^4} \left( \dddot{T^{kaa}} \ddot{\mathbf{d}}^k -
						\frac{1}{2} \dddot{T^{aak}}\ddot{\mathbf{d}}^k \right )
\end{equation}
\begin{equation}
	T^{kij} = \int \frac{j^k}{c}y^iy^j \, d^3 y
\end{equation}
\subsection{Решение}
Общая формула для запаздывающего потенциала:
\begin{equation}
	A_\omega^{\mu} = \int\frac{j^{\mu}}{c} \frac{e^{ik|x - y|}}{|x - y|}\, d^3 y
\end{equation}
Вычислим $A^k$, а $A^0$ и напряжённости полей определим из условия Лоренца.
Разложим член $e^{ikr}/r$:
\begin{equation}
	\frac{e^{ik|x - y|}}{|x - y|} = \sum_m \frac{(-1)^m}{m!}
			\frac{\partial}{\partial x^{i_1}\!\!\!\!\!}\,\, \dots 
				\frac{\partial}{\partial x^{i_m}\!\!\!\!\!}\,\,
				\frac{e^{ikr}}{r}
					y^{i_1} \dots y^{i_m}
\end{equation}

Тогда для $A^{\mu}$ имеем
\begin{equation}
	A_{\omega}^{k} = \sum_m \frac{(-1)^m}{m!}
			\frac{\partial}{\partial x^{i_1}\!\!\!\!\!}\,\, \dots 
				\frac{\partial}{\partial x^{i_m}\!\!\!\!\!}\,\,
				\frac{e^{ikr}}{r}
					\int \frac{j^k}{c} y^{i_1} \dots y^{i_m} \, d^3 y
\end{equation}
Выпишем первые три члена:
\begin{equation}
	A_{\omega}^{k} = \frac{e^{ikr}}{r} \int \frac{j^k}{c} \, d^3 y - 
					\frac{\partial}{\partial x^{i}\!\!\!}\,\frac{e^{ikr}}{r} 
								\int \frac{j^k}{c}y^i \, d^3 y + 
					\frac{1}{2}\frac{\partial}{\partial x^{i}\!\!\!}\,
					\frac{\partial}{\partial x^{j}\!\!\!}\,
						\frac{e^{ikr}}{r} 
								\int \frac{j^k}{c}y^iy^j \, d^3 y  
\end{equation}
Пользуясь сохранением заряда и интегрируя по частям, можно написать следующие равенства:
\begin{equation}
	{\mathbf d}^k = \int  \frac{j_0}{c} y^k\, d^3 x = \frac{i}{kc} \int j^{k}\, d^3 y
\end{equation}
\begin{equation}
	D^{kl} = \int  \frac{j_0}{c} (3y^ky^l - \delta^{kl}r^2)\, d^3 y =
			\frac{i}{kc} \int (3j^k y^l + 3j^l y^k - 2\delta_{kl}j^ay^a) d^3y
\end{equation}
Для магнитного момента можно написать
\begin{equation}
	\epsilon^{klm}M^{m} = \frac{1}{2c} \int j^k y^l - j^l y^k\, d^3 y
\end{equation}
Теперь вычислим магнитное поле.
На расстояниях много больше длины волны волна почти плоская, поэтому можно дифференцировать
только экспоненту.
Получается
\begin{equation}
	A_{\omega}^{k} = \frac{e^{ikr}}{r} \left(\int \frac{j^k}{c} \, d^3 y - 
					ik n^i\int \frac{j^k}{c}y^i \, d^3 y -
					\frac{1}{2}k^2 n^i n^j\int \frac{j^k}{c}y^iy^j \, d^3 y + \dots\right)
\end{equation}
К интегралу во втором слагаемом можно безбоязненно прибавить слагаемое, пропорциональное
$\delta^{ka}$ (это следует из калибровочной инвариантности). Поэтому вектор-потенциал
можно переписать в виде
\begin{equation}
	A_{\omega}^{k} = \frac{e^{ikr}}{r} \left(-ik\mathbf{d}^k - \frac16k^2n^iD^{ik} -
			ik\epsilon^{klm}n^lM^k - k^2 n_i n_j T^{kij} +  \dots\right)
\end{equation}
\begin{equation}
	T^{kij} = \int \frac{j^k}{c}y^iy^j \, d^3 y
\end{equation}
Перейдём от $A^{\mu}_{\omega}$ к $A^{\mu}$.
Тогда легко получается, что
\begin{equation}
	A^{k} = \frac{1}{cr}\left(\dot{\mathbf{d}}^k + \frac{1}{6c}n^i \ddot{D^{ik}} 
								+ \epsilon^{klm}n^l\dot{M}^m
								+ \frac{1}{2c}n_i n_j \ddot{T}^{kij} 
								\right)
\end{equation}
И далее, для плоской волны 
\begin{equation}
	\vec{B} = -\frac{1}{c} \vec{n} \times \frac{\partial \vec{A}}{\partial t}
\end{equation}
\begin{multline}
	B^i = -\frac{1}{c^2r} \left(\epsilon^{ijk}n^j\ddot{\mathbf{d}^k} \right. +
						\frac{1}{6c}\epsilon^{ijk}n^jn^l\dddot{D^{kl}} + \\
						+\left. n^i n^j \ddot{M}^j - \ddot{M}^i + 
						\frac{1}{2c}\epsilon^{ijk}n^jn^a n^b \dddot{T^{kab}} \right)	
\end{multline}
Вектор Пойнтинга везде параллелен $\vec{n}$, а его модуль ---
\begin{multline}
	|\vec{S}| = \frac{c}{4\pi} |\vec{B}|^2 = \frac{1}{4\pi c^3 r^2}
				\left(\epsilon^{ijk}n^j\ddot{\mathbf{d}^k} \right. +
					\frac{1}{6c}\epsilon^{ijk}n^jn^l\dddot{D^{kl}} + \\
					+n^i n^j \ddot{M}^j - \ddot{M}^i + 
					\left.\frac{1}{2c}\epsilon^{ijk}n^jn^a n^b\dddot{T^{kab}}\right)^2	
\end{multline}
Полный поток энергии --- это интеграл $|\vec{S}|$ по сфере, при этом будем учитывать только
слагаемые порядка до $k^2$.
Ответ получается таким:
\begin{equation}
	W = \frac{2\ddot{\mathbf{d}^2}}{3c^3} + 
			\frac{\dddot{D^{kl}}\dddot{D^{kl}}}{180c^5} + \frac{2\ddot{M^2}}{3c^3} + 
			\frac{4}{15c^4} \left( \dddot{T^{kaa}} \ddot{\mathbf{d}}^k -
						\frac{1}{2} \dddot{T^{aak}}\ddot{\mathbf{d}}^k \right )
\end{equation}
\end{document}
