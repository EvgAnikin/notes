\section{Локализованные состояния в подходе сильной связи}
Рассмотрим очень простую ситуацию: пусть в решётке, описываемой
гамильтонианом \eqref{hamiltonian}, есть единственная примесь. Её проще всего описать 
возмущением следующего вида:
\begin{equation}
	\label{pert}
	V = \Delta E a_0^{\dagger} a_0
\end{equation}
Оказывается, что около такой примеси возникнет локализованное состояние. 

Наиболее удобно искать энергию этого состояния и волновую функцию, найдя 
запаздывающую функцию Грина для этого гамильтониана. Энергия состояния, как хорошо 
известно, будет определяться полюсом функции Грина (см. \cite{Abrikosov})


Нетрудно также получить, что функция Грина даётся выражением
\begin{equation}
	\label{general}
	G^R(\omega, m,n) = \sum_\lambda \psi_\lambda(m)\psi_\lambda^{*}(n) 
			\frac{1}{\omega - E_\lambda + i\delta}
\end{equation}
(Здесь ведётся суммирование по всем состояниям). Отсюда можно легко найти волновую функцию
локализованного состояния:
\begin{equation}
	\label{res}
	\operatorname{\mathrm{res}} G^R(\omega, m,n) |_{\omega = E_\Lambda} = 
		\psi_{\Lambda}(m)^{*}\psi_{\Lambda}(n)
\end{equation}

Чтобы найти функцию Грина для гамильтониана с возмущением, нужна, разумеется, 
функция Грина однородной решётки. Она даётся формулой
\begin{equation}
	G_0^R(\omega,m,n) = 
			\frac{1}{N}\sum_p \frac{e^{-ip(R_m - R_n)}}{\omega - E_p + i\delta}.
\end{equation}

Новая функция Грина может быть найдена из уравнения Дайсона.
(Его можно получить в технике Келдыша \cite{Keldysh1964}, \cite{Arseev2015} сразу на запаздывающую функцию Грина, или
перейти к запаздывающей от мацубаровской заменой $i\omega \to \omega + i\delta$.)
\begin{equation}
	G^R(\omega_n,m,n) = 
		G^R_0(\omega_n, m,n) + 
		\Delta E G^R_0(\omega_n, m,0)G^R(\omega_n, 0,n)
\end{equation}
Решение этого уравнения --- 
\begin{equation}
	\label{solution}
	G^R(\omega, m,n) = G^R_0(\omega, m,n) + 
		\frac{\Delta EG^R_0(\omega, m,0)G^R_0(\omega, 0, n)}{1 - \Delta E G^R_0(\omega,0,0)}
\end{equation}
Полюса функции Грина даются уравнением
\begin{equation}
	\Delta E G^R_0(\omega, 0,0) = 1,
\end{equation}
то есть 
\begin{equation}
	 \frac{\Delta E}{N}\int  \frac{\rho(\epsilon)\,d\epsilon}{\omega - \epsilon + i\delta}= 1
\end{equation}
Пользуясь формулой \eqref{res}, получим выражение для волновой функции связанного
состояния.
\begin{equation}
	\label{wavefunction}
	\psi_\Lambda(n) = \frac{G_0^R(E_\Lambda, n,0)}
					{\sqrt{-\frac{\partial G_0^R}{\partial \omega}(E_\Lambda, 0,0)}}
\end{equation}
Приведём без вывода волновые функции в некоторых конкретных примерах. 

Во--первых, в случае, когда $E_\Lambda$ очень близко к краю зоны (например, нижнему), 
волновая функция связанного
состояния даётся приближённой формулой
\begin{equation}
	\psi_\Lambda(n) = \sqrt{\frac{p_0}{2\pi \nu}} \frac{e^{-p_0 R_n}}{R_n}
\end{equation}
($\nu$ --- объём элементарной ячейки, $p_0^2 = -2mE_\Lambda$, $m$ --- эффективная масса.
Энергия отсчитывается здесь от нижнего края зоны)
Такая волновая функция сильно делокализована: она отлична от нуля не только на примесном
атоме, но и на многих атомах вокруг примесного.

Во-вторых, в случае, когда $p_0$ велико, точнее, $p_0\ll a^{-1}$, волновая функция ---
$\psi_\Lambda(0) \approx 1$, а 
$\psi_\Lambda(n) \approx \frac{t_n}{E_\Lambda}$, где $t_n$ --- коэффициенты из гамильтониана.
Волновая функция оказывается сильно локализованной и отлична от нуля только на нескольких 
атомах.

Наконец, в случае одномерной цепочки, когда гамильтониан --- 
\begin{equation}
	\label{ham1d}
	\hat{H} = \sum_n Ea_n^\dagger a_n + ta_{n+1}^\dagger a + ta_{n-1}^\dagger a,
\end{equation}
функция Грина может быть найдена точно. Ответ выглядит так:
\begin{multline}
	G_0^R(\omega, m,n) = 
			\int_{-\pi}^{\pi} 
				\frac{dk}{2\pi} \frac{e^{-ik(m-n)}}{\omega - E - 2t \cos{k} + i\delta}
	= \frac{e^{-(m-n)\kappa}}{2t\sinh{\kappa}}, \\
			\cosh \kappa = \frac{\omega  - E}{2t}
\end{multline}
