\section{Краевые состояния}
Рассмотрим одномерную цепочку \eqref{ham1d} с возмущением \eqref{pert} и устремим $\Delta E$
к бесконечности. Физически это означает добавление бесконечно высокой потенциальной стенки,
в результате чего мы получаем две полуцепочки, не связанные между собой. Далее можно 
обычным способом искать локализованные состояния. (Они называются таммовскими, так 
как впервые были рассмотрены Таммом, см. \cite{Tamm1933}, \cite{Shockley1939})

Разумеется, сказанное относится не только к цепочке \eqref{ham1d}, но и к любой другой модели
сильной связи. Всегда можно ввести такое возмущение, которое разделит систему на две
несвязанные части. Например, в случае плоскости возмущение должно быть примерно таким:
\begin{equation}
	\hat{V} = \Delta E \sum_m a^\dagger_{m,0} a_{m,0}, \quad \Delta E \to \infty
\end{equation}
Это означает, что мы добавляем целую линию примесных атомов с очень большой энергией.

В частном случае гамильтониана \eqref{ham1d} функция Грина получается такой:
\begin{multline}
	G^R(\omega, m,n) = G^R_0(\omega, m,n) - 
		\frac{G^R_0(\omega, m,0)G^R_0(\omega, 0, n)}{G^R_0(\omega,0,0)} = \\
		= \frac{e^{-|m-n|\kappa} - e^{-|m|\kappa- |n|\kappa}}{2t\sinh \kappa}
\end{multline}
Здесь нет локализованных состояний: у функции Грина нет изолированных полюсов. Локализованные
состояния возникают в более сложных случаях, некоторые из которых будут рассмотрены ниже.

\section{Два примера одномерной цепочки}
Так как в самой простой цепочке локализованных состояний не оказалось, усложним цепочку.
А именно, рассмотрим цепочку чередующихся атомов с разными свойствами. 

Во--первых, можно рассмотреть цепочку, где все матричные элемементы перехода между 
соседними атомами одинаковы, а энергии атомов различаются. Гамильтониан такой цепочки ---
\begin{equation}
	H = \sum_n \xi(a_n^\dagger a_n - b_n^\dagger b_n) + ta_n^\dagger(b_n + b_{n-1}) +
			ta_n(b_n^\dagger + b_{n-1}^\dagger)
\end{equation}
Во--вторых, возможен случай, когда энергии атомов одинаковы, но, напротив, различаются 
матричные элементы. Гамильтониан в этом случае ---
\begin{equation}
	H = \sum_n t_1(a_n^\dagger b_n + a_n b_n^\dagger) + 
		t_2 (a_{n+1}^\dagger b_n + a_{n+1} b_n^\dagger)
\end{equation}
Оказывается, что краевое состояние есть только у второй цепочки, причём его наличие 
зависит от типа последней связи в решётке.

Рассмотрим подробнее вторую цепочку. Без ограничения общности будем считать, что
$t_1 > t_2 > 0$ .

После преобразования Фурье гамильтониан примет вид
\begin{equation}
	H = \sum_p 
			(t_1 e^{-\frac{ipa}{2}} + t_2 e^{\frac{ipa}{2}}) a_p^\dagger b_p + 
			(t_1 e^{\frac{ipa}{2}} + t_2 e^{-\frac{ipa}{2}}) a_p b_p^\dagger
\end{equation}
Отсюда получаем энергетический спектр, содержащий две зоны.
\begin{equation}
	E_p^{1,2} = \pm \epsilon_p = \pm|t_1 e^{\frac{ipa}{2}} + t_2 e^{-\frac{ipa}{2}}| = 
		\pm\sqrt{t_1^2 + t_2^2 + 2t_1t_2 \cos{pa}}
\end{equation}
Введём обозначение 
Гамильтониан диагонализуется преобразованием
\begin{equation}
	\left(
	\begin{matrix}
		a_p \\
		b_p
	\end{matrix}
	\right)
	=
	\frac12
	\left(
	\begin{matrix}
		1 & -e^{i\phi}	\\
		e^{-i\phi} & 1
	\end{matrix}
	\right)
	\left(
	\begin{matrix}
		x_p \\
		y_p
	\end{matrix}
	\right),
\end{equation}
\begin{equation}
	e^{i\phi} = \frac{t_1 e^{-\frac{ipa}{2}} + t_2 e^{\frac{ipa}{2}}}
				{\sqrt{t_1^2 + t_2^2 + 2t_1t_2 \cos{pa}}}
\end{equation}
Отсюда получаются выражения для функций Грина в импульсном представлении:
\begin{equation}
	\label{gaa}
	G^R_0 (\omega, p, A, A) =  G^R_0 (\omega, p, B, B) = \frac{1}{2}\left(
		\frac{1}{\omega - \epsilon_p + i\delta} + 
					\frac{1}{\omega + \epsilon_p + i\delta}\right)
\end{equation}
\begin{equation}
	G^R_0 (\omega, p, A, B) = \frac12 \left(\frac{e^{i\phi}}{\omega - \epsilon_p + i\delta} -
						\frac{e^{i\phi}}{\omega + \epsilon_p + i\delta} \right)
\end{equation}
\begin{equation}
	G^R_0 (\omega, p, B, A) = \frac12 \left(\frac{e^{-i\phi}}{\omega - \epsilon_p + i\delta} -
						\frac{e^{-i\phi}}{\omega + \epsilon_p + i\delta} \right)
\end{equation}
Здесь аргументы $A$, $B$ соответствуют операторам $a_p$ и $b_p$. 

Чтобы рассмотреть половину цепочки, введём возмущение 
$V = \Delta E a_0^\dagger a_0$, где $\Delta E$ очень велико.
Это приведёт к уже знакомому нам уравнению Дайсона. Его решение --- 
\begin{multline}
	G^R(\omega, m, s, n, s') = 
		G^R_0(\omega, m,s,n,s') - 
		\frac{G^R_0(\omega, m, s, 0, A)G^R_0(\omega, 0, A, n, s')}{G^R_0(\omega, 0, A, 0, A)},\\
		s,s' = A,B
\end{multline}
Уровни энергии даются, как видно, нулями функции $G^R_0(\omega, 0,A,0,A)$, а волновая функция
связанного состояния пропорциональна $G^R_0(\omega, m,s,0,A)$.

В нашем случае, как нетрудно убедиться, этот уровень энергии --- $E = 0$. Действительно, подставим 
$\omega = 0$ в (\ref{gaa}). Получается
\begin{equation}
	G^R_0 (\omega, p, A, A) = 
		-\frac{1}{2\epsilon_p} + 
					\frac{1}{2\epsilon_p} = 0
\end{equation}
Отсюда следует, что функции Грина в узельном представлении, 
составленные из операторов $a$, обращаются в нуль для всех $m,n$, и, таким образом, волновая 
функция связанного состояния равна нулю во всех узлах $a$. 

Волновая функция оказывается равной 
\begin{equation}
	\begin{split}
	\psi(n,A) &  = 0, \\
	\psi(n, B) & = 
	\left\{
	\begin{matrix}
		{\displaystyle
 		\sqrt{1 - \left(\frac{t_2}{t_1}\right)^2}(-1)^n \left(\frac{t_2}{t_1}\right)^n}
									&	\quad \mbox{при} \quad n \ge 0 \\
		0 & \quad \mbox{при} \quad n < 0
	\end{matrix}
	\right.,\\
	\end{split}
\end{equation}

Краевое состояние есть только у правой половины цепочки, у которой последнее перекрытие --- 
$t_2$, то есть маленькое.

Теперь рассмотрим цепочку с одинаковыми элементами перекрытия. Все вычисления для неё
проводятся точно так же, поэтому просто приведём их результаты:
\begin{equation}
	E_p^{1,2} = \pm \epsilon_p = \pm \sqrt{\xi^2 + 4t^2 \cos^2{\frac{pa}{2}}}
\end{equation}
\begin{equation}
	\begin{split}
	G_0^R(\omega, p, A, A) = \frac{\omega + \xi}{\omega^2 - \epsilon_p^2 + i\delta}\\
	G_0^R(\omega, p, A, B) = \frac{2t\cos{pa}}{\omega^2 - \epsilon_p^2 + i\delta}\\
	G_0^R(\omega, p, B, B) = \frac{\omega - \xi}{\omega^2 - \epsilon_p^2 + i\delta}\\
	\end{split}
\end{equation}
Уровни энергии для системы с возмущением 
определяются уравнением $G_0^R(\omega,0,A,0,A) = 0$, или 
{\sloppy

}

\begin{equation}
	0=\int_{-\pi}^{\pi} \frac{dk}{2\pi} 
		\frac{\omega + \xi}{\omega^2 - \xi^2 - 2t^2 - 2t^2\cos{pa} + i\delta} 
\end{equation}
Интеграл обращается в ноль при $\omega = -\xi$, на границе непрерывного спектра. Внутри
запрещённой зоны, то есть в области $|\omega| < \xi$, интеграл строго отрицателен. Отсюда
можно заключить, что в этом случае граничного состояния не существует: никаких новых
полюсов у функции Грина не возникло.
