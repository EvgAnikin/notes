\section{Спин--орбитальное взаимодействие}
Модель Кейна--Меле, которую мы рассмотрим ниже, описывает спин--орбитальное 
взаимодействие. В связи с этим возникает
вопрос, как явно построить гамильтониан сильной связи из спин--орбитального гамильтониана.
Будем исходить из того, что изначально электроны в атомах находятся в $s$--состояниях,
а перекрытие атомных орбиталей мало. Тогда в приближении сильной связи считается, что
$\langle \psi_1 | \psi_2 \rangle = 0$ ($\psi_1, \psi_2$ --- орбитали разных атомов),
а отличны от нуля только матричные элементы $\langle \psi_1 | \hat{H} | \psi_2 \rangle$.
Оказывается, что симметрия накладывает на них жёсткие ограничения.

Гамильтониан спин--орбитального взаимодействия выглядит так:
\begin{equation}
	\hat{V}_{SO} = \frac{1}{4m^2c^2} \vec{\sigma} \cdot \vec{\nabla} U \times \vec{p}
\end{equation}
Матричные элементы $V_{SO}$ ---
\begin{equation}
\langle \psi_1 | \hat{V}_{SO} | \psi_2 \rangle = 
	\frac{-i}{4m^2c^2}\int d^3 x\, \psi_{1\alpha}^{*} \vec{\sigma}_{\alpha \beta} \cdot
		\vec{\nabla} U \times \vec{\nabla} \psi_{2\beta}
\end{equation}
Выясним, как ведут себя эти матричные элементы по отношению к симметрии относительно 
плоскости. Пусть при симметрии волновые функции $\psi_1, \psi_2$ переходят в себя. Тогда 
после пространственной замены переменных вектор
\begin{equation}
	\vec{a}_{\alpha \beta} = 
	\frac{-i}{4m^2c^2}\int d^3\, x \psi_{1\alpha}^{*} 
		\vec{\nabla} U \times \vec{\nabla} \psi_{2\beta}
\end{equation}
должен, с одной стороны, перейти в себя, с другой, --- в симметричный относительно 
\emph{нормали} к плоскости. Это значит, что $\vec{a} \propto \vec{n}$, и
\begin{equation}
\langle \psi_{1\alpha} | \hat{V}_{SO} | \psi_{2\beta} \rangle \propto \vec{\sigma}_{\alpha \beta} \cdot \vec{n}
\end{equation}
Таким образом,
если плоская решётка симметрична относительно плоскости себя самой, то членов,
переворачивающих спин, нет.
Если же плоскость симметрии перпендикулярна плоскости решётки, то 
\begin{equation}
\langle \psi_{1\alpha} | \hat{V}_{SO} | \psi_{2\beta} \rangle 
	\propto 
	\vec{\sigma}_{\alpha \beta} \cdot \vec{z} \times \vec{d},
\end{equation}
где $\vec{d}$ --- вектор вдоль линии пересечения плоскостей, а $\vec{z}$ --- нормаль к 
решётке. Этот член в гамильтониане называется членом Рашбы (Rashba term, см. \cite{Giglberger2007})
Из тех же соображений следует, что при преобразовании симметрии, оставляющем 
неизменной решётку, но перемещающем $\psi_{1,2}$, матричные элементы получают дополнительный
знак $-$ (проще всего сказать, что нормаль $\vec{n}$ --- псевдовектор). Наконец, можно
добавить, что при поворотах матричные элементы должны переходить друг в друга без 
всяких изменений знака.



