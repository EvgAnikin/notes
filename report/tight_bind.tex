\section{Подход сильной связи}
	Хорошо известно, что волновые функции частицы в периодическом потенциале 
	имеют блоховский вид: $\psi = e^{ipx} u_p(x)$, где $u_p(x)$ --- периодическая 
	функция, а квазиимпульс $p$ лежит в зоне Бриллюэна. Пока будем рассматривать
	только одну энергетическую зону. Тогда, если ввести операторы
	уничтожения $a_p$ частиц в таких состояниях, то 
	гамильтониан для электронов в решётке запишется так:
	\begin{equation}
		H = \sum_p \epsilon_p a_p^{\dagger} a_p,
	\end{equation}
	В подходе сильной связи вводят новые операторы уничтожения $a_n$:
	\begin{equation}
		a_n = \frac{1}{\sqrt{N}} \sum_p e^{-ipR_n} a_p
	\end{equation}
	Эти операторы соответствуют состояниям, волновые функции $w_n(x)$ которых называются 
	функциями Ванье \cite{Anselm1978}. 
	Функция Ванье локализована около $n$--го узла решётки, и, что
	важно, близка к атомной волновой функции, если перекрытие между атомными
	волновыми функциями не слишком велико.

	Гамильтониан в подходе сильной связи записывается через эти операторы $a_n$. 
	\begin{equation}
		\label{hamiltonian}
		H = \sum_k E\,a_k^{\dagger}a_k + \sum_{k\ne l} t_{k-l}\, a_k^{\dagger} a_l
	\end{equation}
	Попутно получается  выражение для энергетического спектра.
	\begin{equation}
		\epsilon_p = E + \sum_{n\ne 0} t_n e^{ipR_n}
	\end{equation}
	Коэффициенты $E$, $t_n$ --- это просто коэффициенты фурье--разложения функции 
	$\epsilon_p$. Известно, что эти коэффициенты быстро убывают у для любой 
	разумной функции. Следовательно, при написании гамильтониана типа \eqref{hamiltonian}
	можно ограничиться только несколькими первыми членами.

	В подходе сильной связи очень удобно рассматривать такие системы, у которых 
	нарушена пространственная однородность. В качестве такой неоднородности может выступить
	примесный атом либо граница кристалла. Оба эти случая будут рассмотрены ниже.

