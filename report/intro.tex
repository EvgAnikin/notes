	\newpage
	\begin{abstract}
		В данной работе изучаются свойства локализованных состояний в 
		моделях сильной связи. На примере одномерных цепочек показано,
		что наличие краевых состояний сильно зависит как от внутренности
		цепочки, так от формы её края. Для модели Кейна--Меле 
		аналитически найден закон дисперсии краевых состояний. Показано,
		что спин--орбитальное взаимодействие s--состояний приводит
		к гамильтониану Кейна--Меле и к взаимодействию Рашбы.
	\end{abstract}
	\section{Введение}
	В статье Кейна и Меле \cite{Kane2005} рассматривается модель плоской 
	шестиугольной решётки со спин--орбитальным 
	взаимодействием в приближении сильной связи. Эта модель реализует так называемый
	топологический изолятор: внутри решётки отличен от нуля топологический индекс 
	\cite{Hasan2010},
	способный принимать значения $0, 1$. Это приводит к тому, что на границе решётки 
	должны образовываться краевые состояния, обладающие интересными свойствами.
	Во--первых, они фильтрованы по спину: невозможно изменить направление движения
	частицы, не изменив её спин. Во--вторых, никаким T--инвариантным возмущением 
	невозможно заставить частицы в этих состояниях рассеяться назад.

	Для того, чтобы разобраться в утверждениях
	этой статьи, нужно понять несколько вещей:

	{\bf во--первых}, каким образом строятся гамильтонианы сильной связи;

	{\bf во--вторых}, как в подходе сильной связи возникают локализованные
	состояния;

	{\bf в--третьих}, как включить в гамильтониан сильной связи спин--орбитальное 
	взаимодействие и какое влияние это оказывает на локализованные состояния.
