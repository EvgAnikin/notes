	\newpage
	\begin{abstract}
        Потом напишу.
	\end{abstract}
	\section{Введение}
    Топологические изоляторы --- это кристаллы с особого рода зонной структурой.
    В отличие от обычных кристаллов, в них невозможно однозначно задать фазу функций 
    Блоха во всей зоне Бриллюэна. Причина этого в том, что отличен от нуля
    топологический инвариант 
    \begin{equation}
        \label{TKNN}
        N = \frac{1}{2\pi i} 
            \int d^2 k\, \left[\partial_i \langle u | \partial_j u \rangle -
            \partial_j \langle u | \partial_i u \rangle \right]
    \end{equation}
    Кроме того, оказывается, что холловская проводимость, измеренная в квантах 
    сопротивления, равна $N$.
    \begin{equation}
        \sigma_{xy} = \frac{e^2}{2\pi \hbar} N
    \end{equation}
    Та же самая картина наблюдается и в случае с эффектом Холла (зону Бриллюэна там 
    заменяет магнитная зона Бриллюэна). Особенность топологических изоляторов в том,
    что в них наблюдается (спиновый) эффект Холла в отсутствие магнитного поля. Другая
    их особенность --- в существовании бездиссипативных краевых состояний, что является
    следствием bulk--boundary correspondence. 

    Топологические изоляторы могут быть реализованы в системах со спин--орбитальным 
    взаимодействием. Холловская проводимость неинвариантна по отношению к обращению
    времени, поэтому в системе с $\mathrm{T}$--симметрией, каковой является кристалл 
    при отсутствии магнитного поля, и при от отсутствии спин--орбитального взаимодействия
    $N$ обязательно равно нулю. Действительно, $\mathrm{T}$--симметрия в этом случае 
    действует просто как комплексное сопряжение. 

    Иная ситуация возникает, когда есть
    спин--орбитальное взаимодействие. В этом случае $N$ может быть отлично от нуля
    для каждой из компонент спина. Вследствие инвариантности по отношению к обращению
    времени для двух компонент спина $N$ будет одинаково по модулю и противоложно 
    по знаку.
