	\newpage
	\begin{abstract}
        Из модели сильной связи для полупроводника с сильным спин--орбитальным
        расщеплением выведен эффективный гамильтониан квантовой ямы.
        Показано, что этот гамильтониан описывает топологический изолятор.
        Для эфективного гамильтониана найден спектр одиночной примеси.
	\end{abstract}
	\section{Введение}
    Топологические изоляторы --- это кристаллы с особого рода зонной структурой.
    Они обладают многими замечательными особенностями, отличающими их от обычных 
    изоляторов и полупроводников. 
    В топологических изоляторах невозможно однозначно задать фазу функций 
    Блоха во всей зоне Бриллюэна. Причина этого в том, что отличен от нуля
    топологический инвариант 
    \begin{equation}
        \label{TKNN}
        N = \frac{1}{2\pi i} 
            \int d^2 k\, \left[\partial_i \langle u | \partial_j u \rangle -
            \partial_j \langle u | \partial_i u \rangle \right]
    \end{equation}
    Этот же инвариант определяет холловскую проводимость, измеренную в квантах 
    магнитного потока:
    \begin{equation}
        \sigma_{xy} = \frac{e^2}{2\pi \hbar} N
    \end{equation}
    Таким образом, в топологических изоляторах можно наблюдать эффект Холла в 
    отсутствие магнитного поля. Наконец, точно так же, как в эффекте Холла, 
    на границе топологических изоляторов всегда возникают краевые состояния,
    пересекающие запрещённую зону.

    Эти краевые состояния обладают свойством киральности: электроны со спином 
    вверх движутся в одну сторону, а со спином вниз --- в другую. Кроме того, состояния 
    с разным направлением движения составляют крамерсовский дублет. Из этого 
    несложно получить, что эти состояния не могут рассеиваться друг в друга 
    под действием какого--либо возмущения, то есть граница топологического 
    изолятора представляет из себя идеальный одномерный проводник.

    Несмотря на то, что подобное поведение частично подтверждается в экспериментах,
    существуют также и эксперименты \cite{Gusev2011}, указывающие на наличие сопротивления
    у краевых мод. Поэтому представляют интерес механизмы, которые могут приводить
    к рассеянию краевых мод друг в друга.

    Будущая цель нашей работы --- исследовать стабильность краевых состояний 
    в случае, когда в объёме присутствуют глубокие примеси (или, 
    иначе говоря, возможность рассеяния с края на край). На глубоких примесях всегда
    возникают свзанные состояния, которые могут лежать посередине запрещённой зоны. Это
    значит, что у них может быть ненулевое перекрытие с краевыми модами. На настоящий 
    момент найден спектр одиночной примеси в простейшей модели топологического изолятора.
%    Топологические изоляторы реализованы экспериментально в квантовых ямах $HgTe$. 
%    Теллурид ртути --- узкозонный полпроводник, 
%    Только системы со спин--орбитальным 
%    взаимодействием могут быть топологическими изоляторами. Впервые 
%    Дело в том, что холловская проводимость неинвариантна по отношению к обращению
%    времени, поэтому в системе с $\mathrm{T}$--симметрией, каковой является кристалл 
%    при отсутствии магнитного поля, и при от отсутствии спин--орбитального взаимодействия
%    $N$ обязательно равно нулю. Иная ситуация возникает, когда есть
%    спин--орбитальное взаимодействие. В этом случае $N$ может быть отлично от нуля
%    для каждой из компонент спина. 

    


    
