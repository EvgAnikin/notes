\section{Случай двух зон}
В случае двух зон $\mathrm{TKNN}$--инварианту можно придать особенно наглядную 
форму. Любой гамильтониан для двух зон можно записать в виде
\begin{equation}
    H(k) = h_0(k) + \vec{h}(k)\vec{\sigma} = h_0(k) + |h|
            \begin{matrix}
                \cos{\theta} & \sin{\theta}e^{-i\phi} \\
                \sin{\theta}e^{i\phi} & -\cos{\theta}
            \end{matrix}
\end{equation}
Его можно явно диагонализовать. Собственные векторы равны
\begin{equation}
    \left(\begin{matrix}
        \cos{\frac{\theta}{2}} \\
        \sin{\frac{\theta}{2}}e^{i\phi}
    \end{matrix}\right),
    \left(\begin{matrix}
        -\sin{\frac{\theta}{2}}e^{-i\phi}\\
        \cos{\frac{\theta}{2}} 
    \end{matrix}\right)
\end{equation}
и соответствуют энергиям 
\begin{equation}
    E(k) = h_0(k) \pm |\vec{h}|
\end{equation}
Непосредственная подстановка (например, первого) собственного вектора в формулу \eqref{TKNN}
приводит к 
\begin{equation}
    N = \frac{1}{4\pi} \int \sin{\theta}\,d\theta d\phi
\end{equation}
Таким образом, $N$ равно количеству ``оборотов'' $\vec{h}$ вокруг единичной сферы.
