\documentclass[10pt]{beamer} 
\usepackage{amsmath}
\usepackage{graphicx}
\usepackage{caption}
\usepackage[utf8]{inputenc}
\usepackage[T1, T2A]{fontenc}
\usepackage[english,russian]{babel}

\mode<presentation>
\usetheme{Warsaw}

\title{Основные причины смерти}
\author[К. Кизина]{Ксения Кизина \\ Преподаватель: Мишина С. А.}
\institute{МГМУ имени Сеченова}
\date{}

\begin{document}
	\begin{frame}
		\titlepage
	\end{frame}

	\begin{frame}
		\frametitle{Почему важно исследовать причины смерти}
		\begin{columns}
			\begin{column}{0.5\textwidth}
				\includegraphics[width=0.8\linewidth]{who.jpg}

				\includegraphics[width=0.8\linewidth]{keyboard.jpg}
			\end{column}
			\begin{column}{0.5\textwidth}
				\begin{block}{}
							
				   Учет того, сколько человек умирает ежегодно и почему,
				   является важнейшим способом оценки эффективности системы
				   здравоохранения в стране. С помощью этих цифр органы
				   общественного здравоохранения определяют, правильно ли они 
				   фокусируют свою деятельность. 	
				\end{block}
			 	\begin{block}{}
				   Многие развивающиеся страны не имеют систем для оценки причин смерти. 
				   Прогресс
				   в этой области  позволит улучшить здоровье населения и 
				   сократить смертность.
			   	\end{block}
			\end{column}
		\end{columns}
	\end{frame}

	\begin{frame}
		\frametitle{Ведущие причины смерти в мире}
		\begin{columns}
			\begin{column}{0.5\textwidth}
				\begin{figure}
					\caption{Данные ВОЗ(2012)}
					\includegraphics[width=\textwidth]{chart_death.png}
				\end{figure}
			\end{column}
			\begin{column}{0.5\textwidth}

				Общая смертность за 2012 год составила \alert{56 миллионов} человек.

				\begin{block}{Основные болезни}
					\begin{itemize}
						\item Ишемическая болезнь сердца
						\item Инсульт
						\item Хроническая обструктивная болезнь лёгких
						\item Респираторные инфекции нижних дыхательных путей
					\end{itemize}
				\end{block}
			\end{column}
		\end{columns}
	\end{frame}
	\begin{frame}
		\frametitle{Ведущие причины смерти в мире}
				\begin{block}{Доля различных болезней в смертности}
					\begin{itemize}
						\item Хронические болезни: 68\% 
						\item Травмы: 9\%
						\item Инфекционные, материнские и неонатальные болезни и
						расстройства пищевого происхождения: 23\%
					\end{itemize}
				\end{block}
				Главная причина смерти в мире --- сердечно--сосудистые заболевания. 
				Они являются причиной смерти в 30 \% всех случаев. 
	\end{frame}

	\begin{frame}
		\frametitle{Смертность и уровень дохода в стране}	
		\begin{columns}
			\begin{column}{0.5\textwidth}
				\only<1>{
				\begin{block}{Страны с высоким уровнем дохода}
					\begin{itemize}
						\item Преобладают хронические неинфекционные болезни
						\item Единственная важная инфекционная причина смерти ---
						инфекции нижних дыхательных путей
						\item Низкая детская смертность: около 1 \% от всех случаев
						\item Большая часть смертей приходится на лиц пожилого возраста
					\end{itemize}
				\end{block}
				}
				\only<2>{
					\begin{figure}
						\includegraphics[width=\linewidth]{high_who.png}
						\caption{Страны с высоким уровнем дохода}
					\end{figure}
				}
			\end{column}
			\begin{column}{0.5\textwidth}
				\only<1>{
				\begin{block}{Страны с низким уровнем дохода}
					\begin{itemize}
						\item Преобладают инфекционные заболевания: ВИЧ, диарея, малярия
						и туберкулёз
						\item Высокая детская смертность: почти 40 \% от всех случаев 
					\end{itemize}
				\end{block}
				\includegraphics[width=\linewidth]{malaria.jpg}
				}
				\only<2>{
					\begin{figure}
						\includegraphics[width=\linewidth]{low_who.jpg}
						\caption{Страны с низким уровнем дохода}
					\end{figure}
				}
			\end{column}
		\end{columns}
	\end{frame}

	\begin{frame}
		\frametitle{Сердечно--сосудистые заболевания}
		{
		\small
		Сердечно--сосудистые заболевания (CCЗ) --- главная причина смерти во всём мире.
		\begin{block}{Они включают в себя:}
			\begin{itemize}
				\item Ишемическую болезнь сердца
				\item Болезнь сосудов головного мозга
				\item Болезнь периферических артерий
				\item Ревмокардит
				\item Врождённые пороки сердца
				\item Тромбоз глубоких вен и эмболию лёгких	
			\end{itemize}
		\end{block}
		Основными факторами риска болезней сердца и инсульта являются неправильное питание, физическая инертность, употребление табака и вредное употребление алкоголя.

		Смертность от этих болезней можно существенно сократить, выстраивая эффективную политику здравоохранения, пропаганды здорового образа жизни. Решающую роль играет ранняя диагностика.

		Более 75 \% смертей от ССЗ происходят в странах с низким и средним уровнем дохода.
		}
		
	\end{frame}
		
	\begin{frame}
		\frametitle{Онкологичечкие заболевания}
		{\small
		Рак --- одна из основных причин смерти во всём мире. Число случаев рака, предположительно, сильно возрастёт в ближайшем будущем.
		\begin{block}{Основные типы рака}		
			\begin{itemize}
\item    рак легких – 1,59 миллиона случаев смерти;
\item    рак печени – 745 000 случаев смерти;
\item    рак желудка – 723 000 случаев смерти;
\item    рак толстого кишечника – 694 000 случаев смерти;
\item    рак молочной железы – 521 000 случаев смерти;
\item    рак пищевода – 400 000 случаев смерти (1).
			\end{itemize}
		\end{block}
Употребление табака, употребление алкоголя, нездоровое питание и отсутствие физической активности являются основными факторами риска развития рака в мире. Факторами риска развития рака являются некоторые хронические инфекции, особенно в странах с низким и средним уровнем дохода. 
		}
		
	\end{frame}

	\begin{frame}
		\frametitle{ВИЧ-инфекция}
		{\small
		Смертность от ВИЧ в мире устойчиво снижается (2.2 миллиона в 2005 году и 1.8 миллиона в 2010 году), 
		однако ВИЧ остаётся одной из основных причин смерти в мире.

		\begin{block}{Основные факторы риска}
			\begin{itemize}				
     \item Незащищённые половые контакты
	\item Инфекции, передаваемые половым путём
	\item Употребление наркотиков
    	\item Небезопасные инъекции, переливания крови, медицинские процедуры, включающие нестерильные разрезы или прокалывание;
    	\item Cлучайные травмы от укола иглой, в том числе среди работников здравоохранения.
			\end{itemize}
		\end{block}
		}
	\end{frame}

	\begin{frame}
		\frametitle{Хроническая обструктивная болезнь лёгких}
		{\small 
		В 2012 году от ХОБЛ умерло более 3 миллионов человек. Большая часть этих смертей приходится на
		страны с низким уровнем дохода.

		\begin{alertblock}{}
		Основной причиной развития ХОБЛ является табачный дым.
		\end{alertblock}
		\begin{block}{Другие факторы риска}
			\begin{itemize}
    \item загрязнение воздуха внутри помещений 
    \item загрязнение атмосферного воздуха
    \item наличие пыли и химических веществ на рабочих местах
    \item частые инфекции нижних дыхательных путей в детстве

			\end{itemize}
		\end{block}
		}
	\end{frame}

	\begin{frame}
		\frametitle{Детская смертность}
		В 2012 году умерли 6.6 миллиона детей в возрасте до 5 лет.

		99\% этих случаев смерти произошли в странах с низким и средним уровнем дохода.

		\begin{block}{Причины детской смертности}
			\begin{itemize}
				\item Пневмония
				\item Недоношенность
				\item Родовая асфиксия и родовая травма
				\item Диарея
			\end{itemize}
		\end{block}
	\end{frame}
	
	\begin{frame}
		\frametitle{Смертность и курение}
		\only<1>{
		\begin{columns}	
			\begin{column}{0.5\textwidth}
				\begin{figure}
				\includegraphics[width=0.9\linewidth]{copd1.jpg}
				\end{figure}
				\begin{figure}
				\includegraphics[width=0.9\linewidth]{smoking.jpg}
				\end{figure}
			\end{column}
			\begin{column}{0.5\textwidth}
				
				\begin{block}{Курение --- главная причина смертности?}
				Употребление табака является важной причиной 
				возникновения многих наиболее смертоносных болезней 
				в мире, в том числе  
				\begin{itemize}
					\item Сердечно--сосудистых заболеваний
					\item Хронической обструктивной болезни лёгких
					\item Рака лёгких
				\end{itemize}
				\end{block}
%				\begin{block}{}
				Курение часто является скрытой причиной заболевания, 
				которое регистрируется в качестве причины наступления смерти. 
%				\end{block}
				
			\end{column}
		\end{columns}
		}
		\only<2>{
			\alert{\bf Каждый десятый взрослый умирает от употребления табака!}
		}
	\end{frame}

	\begin{frame}
		\frametitle{Прогнозы}
			\begin{block}{Прогноз Всемирной организации здравоохранения}
			\begin{itemize}
				\item В связи с ростом продолжительности жизни увеличится смертность от
				онкологических и сердечно--сосудистых заболеваний.
				\item По той же причине возрастёт роль старческого слабоумия
				\item Смертность от инфекционных заболеваний должна уменьшиться
			\end{itemize}
			\end{block}
	\end{frame}
\end{document}
