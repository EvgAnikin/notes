\documentclass{beamer}
\usepackage{amsmath}
\usepackage[english, russian]{babel}
\usepackage[utf8]{inputenc}
\usepackage[T2A, T1]{fontenc}
\usepackage{graphicx}
\usepackage{wrapfig}

\mode<presentation>
\usetheme{Warsaw}

\bibliographystyle{unsrt}


\title{Рассеяние краевых мод в топологических изоляторах}
\author[Е. Аникин]{Евгений Аникин \\
	научный руководитель\\
	чл.-к.~РАН~д.~ф--м.~н.~П.И. Арсеев}
\institute{ФИАН им. Лебедева}
\date{}

\begin{document}

\begin{frame}
    \titlepage
\end{frame}

\begin{frame}
    \frametitle{Двумерные топологические изоляторы}
    \begin{itemize}
        \item Отличен от нуля $\mathrm{TKNN}$--инвариант:
            \begin{equation}
                \label{TKNN}
                N = \frac{1}{2\pi i} 
                    \int d^2 k\, \left(\partial_x \langle u | \partial_y u \rangle -
                    \partial_y \langle u | \partial_x u \rangle \right)
            \end{equation}
        \item Невозможно задать функцию Блоха во всей зоне Бриллюэна
        \item Есть киральные краевые состояния
    \end{itemize}
\end{frame}

\begin{frame}
    \frametitle{Смысл $\mathrm{TKNN}$--инварианта}
    \begin{itemize}
        \item Холловская проводимость выражается через $N$:
            \begin{equation}
                \sigma_{xy} = \frac{e^2}{2\pi \hbar} N
            \end{equation}
        \item $N$ --- набег фазы функции Блоха на границе двух карт, покрывающих зону
            Бриллюэна:
            \begin{equation}
                N = \frac{1}{2\pi}\int_\gamma d\phi
            \end{equation}
    \end{itemize}
\end{frame}

\begin{frame}
    \frametitle{Соответствие объём--граница}
\end{frame}

\begin{frame}
    \frametitle{Квантовые ямы HgTe}
    \begin{itemize}
        \item Первая экспериментальная реализация двумерных топологических изоляторов
    
        \item Эффективный гамильтониан для E1, H1 подуровней:
        \begin{equation}
           \label{eff_so_ham}
            \scalebox{0.8}{%
            $
            H = \left(\begin{matrix}
                    \xi + \frac{1}{m}(2 - \cos{p_x} - \cos{p_y}) & 
                            2t(\sin{p_x} - i\sin{p_y})   \\
                    2t(\sin{p_x} + i\sin{p_y}) & 
                           - \xi - \frac{1}{m}(2 - \cos{p_x} - \cos{p_y}) \\
                \end{matrix}\right)
            $
            }
        \end{equation}
        Описывает топологический изолятор при $\xi < 0$
    \end{itemize}
\end{frame}

\end{document}
