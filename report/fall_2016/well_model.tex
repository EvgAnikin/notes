\section{Модель квантовой ямы в приближении сильной связи}
Реальные топологические изоляторы, как уже отмечалось, являются системами с сильным 
спин--орбитальным взаимодейтвием. Для этого мы построили простую модель сильной связи,
учитывающую валентную $p$--зону и $s$--зону проводимости. Мы рассмотрели плоскую
квадратную решётку из атомов, на каждом из которых ``сидят'' состояния
с $p_x$--, $p_y$--, $p_z$-- и $s$--орбиталями и двумя 
возможными проекцими спина. В модели учитывались перекрытия
орбиталей соседних атомов, а также спин--орбитальное взаимодействие. 

Для написания
спин--орбитального гамильтониана $p$--зоны необходимо перейти к состояниям с определённым
полным моментом. Эти состояния выражаются через $p_x$, $p_y$, $p_z$ орбитали следующим образом:
\begin{equation}
	\label{transform1}
	\begin{gathered}
        \Psi_{\frac{3}{2},\frac{3}{2}} = \frac{X + iY}{\sqrt{2}}\alpha\\
        \Psi_{\frac{3}{2}, \frac{1}{2}} = \sqrt{\frac{1}{3}}\frac{X + iY}{\sqrt{2}}\beta -
                                         \sqrt{\frac{2}{3}} Z\alpha\\
        \Psi_{\frac{3}{2}, -\frac{1}{2}} = -\sqrt{\frac{1}{3}}\frac{X - iY}{\sqrt{2}}\alpha -
                                         \sqrt{\frac{2}{3}} Z\beta\\
        \Psi_{\frac{3}{2},-\frac{3}{2}} = -\frac{X - iY}{\sqrt{2}}\beta
	\end{gathered}
\end{equation}
\begin{equation}
	\label{transform2}
	\begin{gathered}
        \Psi_{\frac{1}{2}, \frac{1}{2}} = \sqrt{\frac{2}{3}}\frac{X + iY}{\sqrt{2}}\beta +
                                         \sqrt{\frac{1}{3}} Z\alpha\\
        \Psi_{\frac{1}{2}, -\frac{1}{2}} = -\sqrt{\frac{2}{3}}\frac{X - iY}{\sqrt{2}}\alpha-
                                         \sqrt{\frac{1}{3}} Z\beta\\
	\end{gathered}
\end{equation}
Здесь $X,Y,Z$ --- атомные орбитали, $\alpha,\beta$ --- состояния со спином вверх и вниз.

Как хорошо известно, в атоме гамильтониан спин--орбитального взаимодейстия имеет вид
\begin{equation}
    H_{\mathrm{SO}} =  A(\vec{S}, \vec{L}) = \frac{A}{2}(J^2 - L^2 - S^2)
\end{equation}
Если орбитальный момент фиксирован, то энергия определяется полным моментом. Таким образом,
спин--орбитальное взаимодействие приводит к расщеплению состояний с моментами $\frac32$ и
$\frac12$.

Гамильтониан имеет вид
\begin{multline}
   \label{well_model_ham}
   H  = \sum_{m,n} (E_s + 4t_s) s_{mn}^{\dagger} s_{mn} - t_s s_{mn}^{\dagger}
                          (s_{m+1,n} + s_{m-1,n} + s_{m,n+1} + s_{m,n-1}) \\
          + t_{sp} s_{mn}^{\dagger} (-p_{m+1,n}^x + p_{m-1,n}^x - p^y_{m,n+1} + p^y_{m,n-1})
                                                                + \mathrm{h.c.}  \\
          + (p_{mn}^x)^{\dagger}(t_{\parallel}(p_{m+1,n}^x + p_{m-1,n}^x) +
                                 t_{\perp} (p^x_{m,n+1} + p^x_{m,n-1})) \\
          + (p_{mn}^x)^{\dagger}(t_{\perp}(p_{m+1,n}^y + p_{m-1,n}^y) +
                                 t_{\parallel} (p^y_{m,n+1} + p^y_{m,n-1})) \\
          + (p_{mn}^z)^{\dagger} t_3 (p_{m+1,n}^z + p_{m-1,n}^z +
                                p^z_{m,n+1} + p^z_{m,n-1}) \\
          -\frac{E_{SO}}{3}
                \begin{matrix}
                    \left(\begin{matrix}
                        p_x^\dagger & p_y^\dagger & p_z^\dagger
                    \end{matrix}\right) \\
                    \\
                    \\
                \end{matrix}
        		\left(\begin{matrix}
        			1 & i & -1 \\
        			-i & 1 & i \\
        			-1 & -i & 1
        		\end{matrix} \right)
                \left(\begin{matrix}
                    p_x \\ 
                    p_y \\
                    p_z
                \end{matrix}\right) + \\
        + \text{всё то же самое с переворотом проекции спина}
\end{multline}

Тогда спин--орбитальный гамильтониан запишется так:
\begin{equation}
	H_{\mathrm{full}} = -\Delta E_{SO} 
			(a_{\frac{1}{2}, \frac{1}{2}}^\dagger a_{\frac{1}{2}, \frac{1}{2}} +
			a_{\frac{1}{2}, -\frac{1}{2}}^\dagger a_{\frac{1}{2}, -\frac{1}{2}})
\end{equation}
После простого преобразования получается последнее слагаемое в \eqref{well_model_ham}.
