\section{Квантовая яма $\mathrm{HgTe}$}
В предыдущем параграфе речь шла о кубическом кристалле. Теперь рассмотрим квантовую яму 
и покажем, что она является топологическим изолятором, причём её можно описывать 
с помощью эффективного гамильтониана \eqref{eff_so_ham}.

Для простоты рассмотрим несколько нереалистичную ситуацию: пусть яма представляет собой 
слой толщиной в один атом. Это приведёт к тому, что в гамильтониане \eqref{hfull} нужно
будет выкинуть все слагаемые, содержащие $p_z$. В результате гаммильтониан 
$6\times6$ распадётся на два блока $3\times3$.

Рассмотрим один из этих блоков. Он имеет вид
\begin{equation}
    \label{block}
    H = \begin{pmatrix}
            E_s + \frac{1}{m^*}(2 - \cos{p_x} - \cos{p_y}) &
            -\frac{P}{\sqrt{2}}(\sin{p_x} + i\sin{p_y}) &
            \frac{P}{\sqrt{6}}(\sin{p_x} - i\sin{p_y}) \\
            -\frac{P}{\sqrt{2}}(\sin{p_x} + i\sin{p_y}) &
            (t_\parallel + t_\perp)(\cos{p_x} + \cos{p_y}) &
            -\frac{1}{\sqrt{3}}(t_\parallel - t_\perp)(\cos{p_x} - \cos{p_y}) \\
            \frac{P}{\sqrt{6}}(\sin{p_x} - i\sin{p_y}) &
            -\frac{1}{\sqrt{3}}(t_\parallel - t_\perp)(\cos{p_x} - \cos{p_y}) &
            \left(\frac{t_\parallel}{3} + \frac{5t_\perp}{3}\right)(\cos{p_x} + \cos{p_y})
        \end{pmatrix}
\end{equation}
Обратим внимание на одно важное отличие \eqref{block} от \eqref{hfull}. В \eqref{hfull}
дырочные состояния с $p = 0$ четырёхкратно вырождены из--за кубической симметрии. 
Выкидывая слагаемые с $p_z$, мы нарушаем симметрию, поэтому в \eqref{block} вырождение 
дырочных состояний снимается. 

Предположим, что расстояние между центром $s$--зоны и одной из дырочных зон много меньше,
чем расстояние между дырочными зонами. Тогда другую дырочную зону при малых $k$ можно 
исключить из рассмотрения, после чего гамильтониан для оставшихся зон будет иметь вид
\eqref{eff_so_ham}.

Более аккуратно эффективный гамильтониан можно вывести, рассматривая более реалистичную
ситуацию в приближении эффективной массы. В \cite{Bernevig2006} авторы находят волновую 
функцию связанного состояния в яме CdTe--HgTe--CdTe при $k_x = k_y = 0$, после чего
ищут закон дисперсии в рамках $k\cdot p$ метода. Оказывается, что при некоторой критической
толщине квантовой ямы уровни размерного квантования E1 и H1 становятся вырожденными, из--за
чего становится возможным пренебречь их взаимодействием с другими уровнями размерного
квантования и пользоваться эффективным двухуровневым гамильтонианом. 
