\section{Простые примеры топологических изоляторов}
Простейший гамильтониан с отличным от нуля $TKNN$--инвариантом 
(он рассматривался в \cite{Qi2006}) ---
\begin{equation}
    H = \left(\begin{matrix}
            \xi + \frac{1}{m}(2 - \cos{p_x} - \cos{p_y}) & 2t(\sin{p_x} - i\sin{p_y})   \\
            2t(\sin{p_x} + i\sin{p_y}) & - \xi - \frac{1}{m}(2 - \cos{p_x} - \cos{p_y}) \\
        \end{matrix}\right)
\end{equation}
Он возникает как эффективный гамильтониан, описывающий зону проводимости и зону
тяжёлых дырок в более реалистичной модели полупроводника со спин--орбитальным
взаимодействием (см. ниже).

Другой пример --- модель Кейна--Меле \cite{Kane2005}, 
описывающая спин--орбитальное взаимодействие в 
графене. Она состоит из двух копий для противоположных проекций спина, причём у каждой
из копий $\mathrm{TKNN}$--инвариант отличен от нуля. Гамильтониан --- 
\begin{equation}
		\hat{H} = 
			\sum_{\langle ij \rangle \alpha} t c^\dagger_{i\alpha} c_{j\alpha} + 
				\sum_{\langle\langle ij \rangle\rangle \alpha\beta} 
					it_2 \nu_{ij} s^z_{\alpha \beta} c^\dagger_{i\alpha} c_{i\beta}
\end{equation}
Суммирование в первом слагаемом идёт по соседним ячейкам, а во втором --- по соседним
ячейкам \emph{одной подрешётки}, при этом $\nu_{ij} = \pm 1$, и его знак зависит
от ориентации кратчайшего пути, соединяющего две ячейки.

В импульсном представлении для этой модели
\begin{equation}
    \begin{gathered}
    	H = |h|\left(
            \begin{matrix}
                \cos{\theta} & \sin{\theta}e^{-i\phi} \\
                \sin{\theta}e^{i\phi} & -\cos{\theta}
            \end{matrix}
    	\right),\\
    	|h|\cos{\theta} = 2t_2 (\sin{px} - \sin{py} - \sin{p(x-y)}), \\
    	|h|\sin{\theta}e^{i\phi} = t(1 + e^{-ipy} + e^{ip(x-y)}) 
    \end{gathered}
\end{equation}
