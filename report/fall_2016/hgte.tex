\section{Квантовая яма $\mathrm{HgTe}$}
Квантовые ямы CdTe--HgTe--CdTe --- первая экспериментальная реализация
топологических изоляторов. Это было предсказано теоретически в \cite{Bernevig2006} и
обнаружено экспериментально в \cite{Konig2007}. 
%Отличие $\mathrm{TKNN}$--инварианта от нуля в них
%достигается за счёт инверсии зон:  
%в этих квантовых ямах валентная зона и зона проводимости образованы (в центре
%зоны Бриллюэна) соответственно из 
%состояний $s$-- и $p$-- типа, то есть не так, как обычно устроены полупроводники. 
%

Спектр такой квантовой ямы можно найти в рамках $k\cdot p$ метода. Спектры HgTe и CdTe 
нужно рассматривать в модели Кейна, явно учитывая $s$--зону проводимости и зоны 
лёгких и тяжёлых дырок. Гамильтониан модели Кейна --- 
\begin{equation}
    \begin{gathered}
        \begin{split}
            H &= \begin{pmatrix}
                    E_c + H_c &&  T \\
                    T^\dagger && E_v + H_v
                \end{pmatrix},\\
            H_c& = \frac{(2F+1)\hbar^2k^2}{2m}I_{2\times2},\\
            H_v& = -\frac{\hbar^2}{2m}((\gamma_1 + \frac52\gamma_2)k^2 - 
                2\gamma_2(\vec{k}\vec{J})^2), 
        \end{split}\\
        T = \begin{pmatrix}
                 -\frac{1}{\sqrt{2}}Pk_{+} && \sqrt{\frac23}Pk_z && 
                                        \frac{1}{\sqrt{6}}k_{-} && 0  \\
                  0 && -\frac{1}{\sqrt{6}}k_{+} && \frac{2}{3}P k_z 
                                        && \frac{1}{\sqrt{2}}P k_{-}
            \end{pmatrix}
    \end{gathered}
\end{equation}
Здесь $H_c$ -- гамильтониан зоны проводимости, $H_v$ --- гамильтониан валентной зоны
(гамильтониан Латтинжера), $T$ --- ``взаимодейстие'' зон. Вид гамильтониана определяется
из соображений симметрии, а численные значения констант известны из эксперимента 
(см. \cite{Novik2005}).

Зонная структура HgTe --- инверсная, то есть $s$--зона в нём лежит ниже $p$--зоны и
фактически является валентной зоной.
В CdTe, напротив, порядок зон нормальный, $E_c > E_v$. Это приводит к тому, что
уровни размерного квантования, соответствующие электронам проводимости и тяжёлым 
дыркам, расположены по--разному в зависимости от толщины ямы. При некоторой критической
толщине эти уровни пересекаются. Это позволяет написать эффективный гамильтониан, учитывающий
только подуровни электронов проводимости и тяжёлых дырок. Оказывается, что он имеет вид
\eqref{eff_so_ham}
