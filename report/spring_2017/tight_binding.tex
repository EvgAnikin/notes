\section{Введение: Модель сильной связи (tight-binding model) \\для топологического изолятора}
Простейший топологический изолятор описывается гамильтонианом \cite{Bernevig2006}
\begin{equation}
    \label{BHZ}
    H = \left(\begin{matrix}
            \xi + \frac{1}{m}(2 - \cos{p_x} - \cos{p_y}) & 2t(\sin{p_x} - i\sin{p_y})   \\
            2t(\sin{p_x} + i\sin{p_y}) & - \xi - \frac{1}{m}(2 - \cos{p_x} - \cos{p_y}) \\
        \end{matrix}\right)
\end{equation}
Спектр этого гамильтониана несложно вычислить:
\begin{equation}
    E_p^2 = (\xi + \frac{1}{m}(2 - \cos{p_x} - \cos{p_y}))^2 + 4t^2(\sin^2{p_x} + \sin^2{p_y})
\end{equation}
Как видно, гамильтониан описывает две симметричные зоны, щель между которыми равна $2|\xi|$. 
Оказывается, при $\xi > 0$ гамильтониан описывает обычный изолятор, тогда как при $\xi < 0$ ---
топологический. 

Гамильтониан \eqref{BHZ} задан во всей зоне Бриллюэна, поэтому, делая обратное 
преобразование Фурье, можно можно перейти к решёточному гамильтониану (также называемому
гамильтонианом сильной связи или tight--binding model).

Решёточный гамильтониан выглядит так:
\begin{multline}
    H_{\mathrm{lattice}} = \sum_{mn} \left\{
        a_{mn}^\dagger\left( \left(\xi + \frac{2}{m}\right) a_{mn}
                 -\frac{1}{2m}(a_{m+1,n} + a_{m-1,n} + a_{m,n+1} + a_{m,n-1})\right)\right. \\
        -it a_{mn}^\dagger(b_{m+1,n} -b_{m-1,n} - i(b_{m,n+1} - b_{m,n-1}))\\
        -it b_{mn}^\dagger(a_{m+1,n} -a_{m-1,n} + i(a_{m,n+1} - a_{m,n-1}))\\
        -\left. b_{mn}^\dagger\left( \left(\xi + \frac{2}{m}\right) b_{mn}
                 -\frac{1}{2m}(b_{m+1,n} + b_{m-1,n} + b_{m,n+1} + b_{m,n-1})\right) \right\}
\end{multline}
Здесь $a_{mn}$, $b_{mn}$ --- операторы уничтожения состояний зоны проводимости и валентной зоны
соответственно на узле $(m,n)$. Непосредственно обобщая этот решёточный гамильтониан, можно
рассматривать решётку конечного размера, добавлять примеси и так далее. 
