	\newpage
%	\begin{abstract}
%        Из модели сильной связи для полупроводника с сильным спин--орбитальным
%        расщеплением выведен эффективный гамильтониан квантовой ямы.
%        Показано, что этот гамильтониан описывает топологический изолятор.
%        Для эфективного гамильтониана найден спектр одиночной примеси.
%	\end{abstract}
	\section{Введение}
    Двумерные топологические изоляторы --- это двумерные кристаллы 
    с особого рода зонной структурой:
    в них невозможно однозначно задать фазу функций 
    Блоха во всей зоне Бриллюэна. Причина этого в том, что (в простейшем случае) 
    отличен от нуля
    топологический инвариант \cite{Kohmoto1985}
    \begin{equation}
        \label{TKNN}
        N = \frac{1}{2\pi i} 
            \int d^2 k\, \left[\partial_i \langle u | \partial_j u \rangle -
            \partial_j \langle u | \partial_i u \rangle \right]
    \end{equation}
    В реалистичном случае этот инвариант отличен от нуля для каждой из компонент спина,
    но имеет для них противоположные знаки.
    Этот инвариант определяет спин--холловскую проводимость, 
    измеренную в квантах 
    магнитного потока:
    \begin{equation}
        \sigma_{xy} = \frac{e^2}{2\pi \hbar} N
    \end{equation}
    
    Наконец, точно так же, как в эффекте Холла, 
    на границе топологических изоляторов всегда возникают краевые состояния,
    пересекающие запрещённую зону, см. \cite{Hasan2010}.

    Эти краевые состояния обладают свойством киральности: электроны со спином 
    вверх движутся в одну сторону, а со спином вниз --- в другую. Кроме того, состояния 
    с разным направлением движения составляют крамерсовский дублет. Поэтому
    эти состояния не могут рассеиваться друг в друга 
    под действием какого--либо возмущения, то есть граница топологического 
    изолятора представляет из себя идеальный одномерный проводник.

    В многочисленных экспериментах (например, \cite{Konig2007}, \cite{Gusev2011})
    показано, что квантовая яма HgTe является топологическим изолятором при 
    некоторых параметрах. Наличие краевых состояний подтверждается измерениями
    нелокального сопротивления. Однако, хотя краевые состояния и
    определяют транспорт в квантовой яме, этот транспорт оказывается не баллистическим 
    \cite{Gusev2011}.
    Поэтому представляют интерес механизмы, которые могут приводить
    к рассеянию краевых мод друг в друга.

    Целью настоящей работы является изучение нескольких механизмов, которые
    потенциально могут приводить к рассеянию краевых состояний. Мы рассмотрели 
    точечную примесь в топологическом изоляторе, нашли для неё связанные состояния 
    и обсудили возможность образования примесных уровней в запрещённой зоне. 
    (Похожие вещи обсуждались в \cite{Lu2011}. Также 
    методом прямой диагонализации показана устойчивость краевых состояний к сильным
    потенциальным примесям и, наоборот, локализация магнитным беспорядком. Наконец, 
    обсуждается возможность спонтанного нарушения симметрии около края, вследствие чего
    краевые состояния перестают быть защищёнными. Все вычисления проделаны в эффективной
    модели сильной связи, описывающей два уровня размерного квантования квантовой ямы
    HgTe. 
    
%    Топологические изоляторы реализованы экспериментально в квантовых ямах $HgTe$. 
%    Теллурид ртути --- узкозонный полпроводник, 
%    Только системы со спин--орбитальным 
%    взаимодействием могут быть топологическими изоляторами. Впервые 
%    Дело в том, что холловская проводимость неинвариантна по отношению к обращению
%    времени, поэтому в системе с $\mathrm{T}$--симметрией, каковой является кристалл 
%    при отсутствии магнитного поля, и при от отсутствии спин--орбитального взаимодействия
%    $N$ обязательно равно нулю. Иная ситуация возникает, когда есть
%    спин--орбитальное взаимодействие. В этом случае $N$ может быть отлично от нуля
%    для каждой из компонент спина. 

    


    
