\newpage
\section{Локализация краевых состояний в присутствии магнитных примесей}
На границе топологического изолятора возникают киральные краевые моды: электроны 
с разными спинами распространяются в ротивоположных направлениях. Рассеяние этих мод 
друг в друга под действием $T$--инвариантного возмущения невозможно, так как они образуют
крамерсовский дублет. Магнитные примеси, однако, допускают возможность рассеяния. В связи
с этим возникает вопрос о возможности андерсоновской локализации краевых состояний. В 
