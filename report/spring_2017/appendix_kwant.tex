\section{Задача рассеяния}
\label{app:kwant}
Задача рассеяния ставится таким образом:
к системе, описываемой гамильтнианом $H_s$, присоединяются трансляционно--инвариантные 
контакты. Любой набор контактов можно описать, задав элементарную ячейку с гамильтонианом 
$H_L$ и матрицу перехода между ячейками $V_L$. Кроме того, нужно задать матрицу перехода
между системой и контактами. Полный гамильтониан системы с контактами
принимает вид
\begin{equation}
    H = \begin{pmatrix}
            H_S             & V_{LS}     & 0           & \ddots \\
            V_{LS}^\dagger  & H_L        & V_L         & 0  \\
            0               & V_L^\dagger& H_L         & V_L    \\
            \ddots          & 0          & V_L^\dagger & \ddots \\
        \end{pmatrix}
\end{equation}
Задачу рассеяния для такого гамильтониана можно свести к неоднородной системе линейных 
уравнений, которая может быть эффективно решена численно. Такой подход оказывается 
гораздо выгоднее, чем диагонализация гамильтониана $H_S$. 

Покажем, как свести задачу рассеяния к системе линейных уравнений. 
Волновая функция в задаче рассеяния имеет вид 
\begin{equation}
    \Psi = (\Psi_S, \Psi_V(0), \Psi_V(1), \dots)
\end{equation}
Так как контакты трансляционно инвариантны, $\Psi_V(k)$ можно искать в виде суперпозиции
плоских волн: $\Psi_V(k) = \Psi_V e^{ipk}$. Квазиимпульсы и волновые функции определяются 
уравнениями
\begin{equation}
    \label{modes}
    \begin{gathered}
        (V_L^\dagger e^{-ip} + H_L + V_L e^{ip} - E) \Psi_V = 0\\
        \det{(V_L^\dagger e^{-ip} + H_L + V_L e^{ip} - E)} = 0
    \end{gathered}
\end{equation}
Если размер матрицы $H_L$ --- $l$, то у уравнений \eqref{modes} будет $2l$ решений, из 
которых $l$ соответствуют падающим либо растущим волнам, а $l$ --- исходящим либо 
затухающим.

Пусть волновая функция рассеяния в контактах имеет вид
\begin{equation}
    \Psi_L(k) = \Psi_m e^{-ip_m k} + \sum_{n} r_{mn} \Psi_n e^{ip_n k}
\end{equation}
Здесь $m$ соответствует падающей моде, а $n$ --- затухающим либо исходящим волнам. Волновая 
функция $\Psi_S$ в регионе рассеяния, разумеется, тоже неизвестна. Подставляя волновую функцию
в уравнение Шрёдингера, получим уравнения на $\Psi_S$ и $r_{mn}$.
\begin{equation}
    \begin{gathered}
        H_S \Psi_S + \sum_n r_{mn} V_{LS} \Psi_n = -V_{LS} \Psi_m\\
        V_{LS}^\dagger\Psi_S  - \sum_n r_{mn}V_L \Psi_n e^{-ip_n} = V_L e^{ip_m} \Psi_m
    \end{gathered}
\end{equation}
Легко убедиться, что здесь уравнений столько же, сколько неизвестных, и систему можно решить.

Описанный метод решения задачи рассеяния используется в пакете Kwant \cite{Groth2014}.
