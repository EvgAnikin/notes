\section{Топологический инвариант}
Кратко изложим, что представляет из себя TKNN--инвариант для произвольного 
зонного гамильтониана. Пусть гамильтониан системы в импульсном представлении даётся некоторой
матрицей $H(k)$, где $k$ --- волновой вектор зоны Бриллюэна.  
Для каждого $k$ гамильтониан всегда можно диагонализовать
и получить набор собственных векторов $u_i(k)$. Будем считать, что ни при каком $k$ нет 
вырождения спектра (это соответствует случаю изолятора). 
Выбор собственного базиса неоднозначен: каждый вектор
можно домножить на произвольную фазу. Очевидно, что в каждой окрестности $k$ можно выбрать
фазу каждого вектора так, чтобы она непрерывно зависела от $k$. Однако, вообще говоря, 
эту фазу может оказаться невозможно однозначно задать во всей зоне Бриллюэна.

Рассмотрим одну из ветвей спектра гамильтониана: $u_i(\vec{k})$ 
(в дальнейшем опустим индекс $i$). Предположим для простоты, что зону Бриллюэна
можно разбить на две части $A$ и $B$, 
в каждой из которых фазу $u(k)$  можно определить однозначно. 
Назовём функции Блоха в этих двух частях $u_a(k)$ и $u_b(k)$. На границе $A$ и $B$ (назовём
её $\gamma$) должно выполняться равенство
\begin{equation}
   u_b(k) = e^{-i\phi(k)}u_a(k).
\end{equation}
Определим число Черна, иначе называемое $\mathrm{TKNN}$--инвариант 
(см. \cite{Kohmoto1985, Thouless1982}), как
\begin{equation}
    \label{chern_number_definition}
    N = \frac{1}{2\pi}\int_\gamma d\phi
\end{equation}
С помощью простого вычисления %(используется формула Стокса) 
$N$ можно записать как интеграл по всей зоне Бриллюэна:
\begin{equation}
    N = \frac{1}{2\pi i}\int_\gamma \langle u_a |\vec{\nabla}_k u_a \rangle  - 
                             \langle u_b |\vec{\nabla}_k u_b \rangle d\vec{k} = 
        \frac{1}{2\pi i} 
            \int d^2 k\, \left[\partial_x \langle u | \partial_y u \rangle -
            \partial_y \langle u | \partial_x u \rangle \right]
\end{equation}
В последнем интеграле подыынтегральное выражение калибровочно--инвариантно, что
позволяет опустить индексы $a,b$. 

%Пусть $u(\vec{k})$ однозначно задана в зоне Бриллюэна, \emph{однако вместо периодических 
%граничных условий выполнены более слабые условия:} 
%\begin{equation}
%    \begin{gathered}
%        u(k_x, 2\pi) = e^{i\phi_x(k_x)} u(k_x, 0),\\
%        u(2\pi, k_y) = e^{i\phi_y(k_y)} u(0, k_y)
%    \end{gathered}
%\end{equation}
%Разумно считать, что это всегда возможно, так как квадрат топологически тривиален.
%В этом случае можно определить инвариант как оборот фазы $u(k)$ при движении по границе
%зоны Бриллюэна.
