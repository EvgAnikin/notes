\newpage
\section{Реконструкция края}
В данном разделе рассматривается реконструкция края (аналогично \cite{Wang2017}) 
в модели сильной связи \eqref{BHZ} из \cite{Bernevig2006}.

Как известно, на границе топологического изолятора всегда возникают краевые состояния. Если 
$T$--симметрия не нарушена, то
состояния с противоположными спинами и импульсами образуют крамерсовский дублет. Поэтому они
не могут рассеиваться друг в друга ни на каком $T$--инвариантном возмущении.

При учёте электрон--электронного взаимодействия, однако, $T$--симметрия может спонтанно
нарушиться. Когда такое нарушение происходит в объёме проводника, это называется магнетизмом 
Стонера. Для случая топологического изолятора же интересна ситуация, 
когда спонтанная намагниченность возникает около его края. Это приведёт к тому, что
краевые состояния больше не будут топологически защищёнными. Могут, в частности, появиться
дополнительные краевые моды, пересекающие запрещённую зону, или их набор
может стать различным для спина вверх и спина вниз (именно это и называется
реконструкцией).

В дальнейшем будет использоваться приближение Хартри--Фока для точечного отталкивания:
\begin{equation}
    V_{\mathrm{int}} = g\sum_i \hat{n}_{i\uparrow} \hat{n}_{i\downarrow}
\end{equation}
\begin{equation}
    V_{\mathrm{Hartree-Fock}} = g\sum_i \hat{n}_{i\uparrow} \bra \hat{n}_{i\downarrow}\ket + 
                               \bra\hat{n}_{i\uparrow}\ket \hat{n}_{i\downarrow} - 
                               \bra\hat{n}_{i\uparrow}\ket \bra\hat{n}_{i\downarrow}\ket 
\end{equation}

Согласно \cite{Wang2017}, для топологического изолятора с резкой границей реконструкции 
не происходит. Однако реконструкция возможна, если около края есть плавный отталкивательный
потенциал вида
\begin{equation}
    
\end{equation}
