\section{Проводимость полосы топологического изолятора\\
         в формализме Ландауэра--Буттикера}
Проводимость полосы можно вычислять в формализме Ландауэра--Буттикера, как
это делается, например, в \cite{Li2009}. А именно, для системы, состоящей
из полосы топологического изолятора и присоединённых к ней бесконечных
контактов, решается задача рассеяния. Проводимость даётся формулой
Ландауэра:
\begin{equation}
    G = \frac{2_se^2}{2\pi \hbar} \Tr{\hat{t}^\dagger \hat{t}}
\end{equation}
Здесь $\hat{t}$ --- матрица рассеяния из одного контакта в другой. 

Задача рассеяния в такой постановке может быть эффективно решена численно
(см. приложение~\ref{app:kwant}). Для этого существует библиотека Kwant 
\cite{Groth2014} для языка программирования Python.

Мы рассматривали контакты, описываемые гамильтонианом
\begin{equation}
    H_{\mathrm{Lead}} = \sum_i (E + 4t)c_{i}^\dagger a_{i} - \sum_{<i,j>} t c_{i}^\dagger c_{j}
\end{equation}
Они соединялись с топологическим изолятором посредством членов $c_i^\dagger a_{j,\sigma}$, 
где $c$ --- операторы контакта, а $a$ --- топологического изолятора.

В \cite{Li2009} (а также в многих других работах) исследуется зависимость проводимости 
бруска от беспорядка. При энергии, лежащей в щели, проводимость зависит от амплитуды
беспорядка следующим образом:
%\begin{figure}
%    \includegraphics[width=0.5\linewidth]{}
%\end{figure}
Видно, что проводимость начинает уменьшаться только при очень большой амплитуде беспорядка,
сравнимой с характерной энергией связи между узлами.

Мы провели аналогичные симуляции, реализуя беспорядок несколько по--другому. В \cite{Li2009}
к энергии каждого узла добавлялась случайная величина, равномерно распределённая в 
$[-W/2, W/2]$
