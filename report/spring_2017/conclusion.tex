\section{Заключение}
Таким образом, мы рассмотрели два возможных механизма для рассеяния краевых состояний:
спонтанное нарушение симметрии и примесные состояния в щели. Реалистичность обоих механизмов
вызывает вопросы: нарушение симметрии происходит только для гладкого потенциала 
около края, а состояния в щели возникают только для довольно глубоких 
примесей. Тем не менее, требуется аккуратная оценка применимости рассмотренных моделей и
их сравнение с экспериментально реализуемыми топологическими изоляторами. Для модели 
с примесными состояниями в щели также требуется теоретическое и численное исследование
транспортных свойств.
