\newpage
\section{Заключение}
Мы исследовали влияние беспорядка на краевые состояния топологического 
изолятора в BHZ--модели, гамильтониан которой
--- эффективный двухуровневый гамильтониан для зон размерного квантования в яме 
CdTe--HgTe--CdTe. Был рассмотрен не только слабый потенциальный беспорядок, но и беспорядок 
в виде глубоких редко расположенных примесей, создающих ненулевую плотность состояний в щели.
Для отдельно взятой примеси был найден спектр связанных состояний: оказалось, что для
достаточно глубокой примеси возникают два связанных состояния, одно из которых при определённой
глубине примеси может лежать в середине зоны, тем самым давая вклад в плотность состояний. 
Именно такие примеси использовались 
для численного моделирования полосы топологического изолятора в 
формализме Ландауэра--Буттикера. Оказалось, что 
что краевые состояния оказываются устойчивыми не просто к слабому потенциальному беспорядку, 
но и к беспорядку, создающему ненулевую плотность состояний в щели. 
