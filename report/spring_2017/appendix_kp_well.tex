\newpage
\section{Уровни размерного квантования\\ в квантовой яме HgTe}
\label{app:dim_quant}
Гамильтониан Кейна имеет вид
\begin{equation}
    \label{kane_ham}
    H = \begin{pmatrix}
                E_c + \frac{\hbar^2 k^2}{2m_s}E_{2\times 2} & T \\
                T^\dagger & E_v + H_{L}
        \end{pmatrix}
\end{equation}
\begin{equation*}
    H_L = -\frac{\hbar^2}{2m_0}\left[
            \left(\gamma_1 + \frac{5}{2}\gamma_2\right)k^2 -
            2\gamma_2(\vec{k} \cdot \vec{J})^2 - \right. 
            - \left.2(\gamma_3 - \gamma_2)(\{J_x J_y\} + \{J_x J_z\} + \{J_y J_z\})
            \vphantom{\frac{1}{2}}\right]
\end{equation*}
\begin{equation*}
    T = P\begin{pmatrix}
           -\frac{1}{\sqrt{2}}k_{+} & \sqrt{\frac{2}{3}}k_z  
                    & \frac{1}{\sqrt{6}} k_{-} & 0 \\
            0 & -\frac{1}{\sqrt{6}} k_{+} 
                    & \sqrt{\frac{2}{3}}k_z & \frac{1}{\sqrt{2}} k_{-} 
         \end{pmatrix}
\end{equation*}
Так как
коэффициенты в гамильтониане Кейна зависят явным образом от $z$, нужно сделать замену
$k_z \to i\partial_z$. При этом, чтобы гамильтониан остался эрмитовым,
$\frac{k_z^2}{2m} \to -\partial_z \frac{1}{2m} \partial_z$ (и аналогично --- для других
членов). Все величины, зависящие от $z$, равны значениям для HgTe при $-d/2 < z < d/2$,
для CdTe --- в противном случае.

Уровни размерного квантования можно относительно просто найти для $k_x, k_y = 0$. В этом 
случае гамильтониан значитально упрощается. Уровни тяжёлых дырок оказываются полностью
отщеплёнными для каждой проекции спина и описываются эффективным гамильтонианом
\begin{equation}
    H_{\mathrm{HH}} = E_v(z) + \frac{1}{2m}
                        \frac{\partial}{\partial z} 
                        (\gamma_1(z) - 2\gamma_2(z)) 
                        \frac{\partial}{\partial z} 
\end{equation}
Также отщепляются $s$--зона вместе с зоной лёгких дырок. Они описываются эффективным
гамильтонианом 
\begin{equation}
    H_{\mathrm{s,LH}} = \begin{pmatrix}
                            E_c - \frac{\hbar^2}{2m}
                                 \frac{\partial}{\partial z}(1 + 2F) 
                                 \frac{\partial}{\partial z} &
                                 \sqrt{\frac{2}{3}}P k_z \\
                                 \sqrt{\frac{2}{3}}P k_z &
                                 E_v +  \frac{\hbar^2}{2m}
                                 \frac{\partial}{\partial z}(\gamma_1 + 2\gamma_2)
                                 \frac{\partial}{\partial z} 
                        \end{pmatrix}
\end{equation}
Для обоих гамильтонианов можно получить алгебраические уравнения на уровни энергии. Эти
уравнения решаются численно.
