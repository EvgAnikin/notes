\documentclass{article}
\usepackage{amsmath}
\usepackage{bbm}
\usepackage{graphicx}
\usepackage{ntheorem}
\usepackage[utf8]{inputenc}
\usepackage[T1, T2A]{fontenc}
\usepackage[english,russian]{babel}

\newcommand{\bra}{\langle}
\newcommand{\ket}{\rangle}
\newcommand{\p}{\partial}

\theorembodyfont{\normalfont}
\newtheorem{problem}{Задача}

\title{Второе задание по допглавам}
\author{Anikin Evgeny}

\begin{document}
\maketitle
\section{Метод Лапласа}
\begin{problem}
    Решить уравнение 
    \begin{equation}
        u'' - zu = 0
    \end{equation}
    Получить два линейно независимых решения в виде интегралов по контуру и найти их 
    асимптотики при вещественных $z$, $z \to \pm \infty$.
\end{problem}
\begin{problem}
    С помощью метода Лапласа решить уравнение на полиномы Эрмита и найти производящую
    функцию полиномов Эрмита.
\end{problem}
    
\section{Гипергеометрическая функция}
\begin{problem}
    Пусть $z_0$ --- особая точка коэффициентов $p(z)$ и $q(z)$ уравнения
    \begin{equation}
        \label{second_order_eq}
        u'' + p(z)u' + q(z)u = 0
    \end{equation}
    Доказать, что уравнение имеет решение в виде степенного ряда 
    \begin{equation}
        u(z) = z^{\nu}\sum_{k=0}^{\infty} c_k (z - z_0)^k 
    \end{equation}
    тогда и только тогда, когда 
    \begin{equation}
        \label{regularity_conditions}
        \begin{gathered}
            p(z) = \frac{p'(z)}{z - z_0},\\
            q(z) = \frac{q'(z)}{(z - z_0)^2},
        \end{gathered}
    \end{equation}
    где $p(z)$, $q(z)$ --- регулярные в $z_0$ функции. Как найти $\nu$?

    Особые точки дифференциальных уравнений, в которых выполняются условия 
    \eqref{regularity_conditions}, называются
    \emph{регулярными}.
\end{problem}   
\begin{problem}
    Доказать, что уравнение \eqref{second_order_eq} имеет решение вида
    \begin{equation}
        u(z) = z^{\nu}\sum_{k=0}^{\infty} c_k z^{-k}
    \end{equation}
    в окрестности $z = \infty$ тогда и только тогда, когда
    \begin{equation}
        \label{regularity_conditions_on_inf}
        \begin{gathered}
            p(z) = \frac{1}{z}\sum_{k=0}^{\infty} p_k z^{-k}\\
            q(z) = \frac{1}{z^2}\sum_{k=0}^{\infty} q_k z^{-k}
        \end{gathered}
    \end{equation}
    Эти условия представляют собой очевидное обобщение $\eqref{regularity_conditions}$ на
    случай $z = \infty$. В этом случае говорят, что $z = \infty$ --- регулярная особая точка.
\end{problem}
\begin{problem}
    Найти общий вид уравнения второго порядка с
    \begin{enumerate}
        \item двумя регулярными особыми точками;
        \item тремя регулярными особыми точками.
    \end{enumerate}
    Имеется в виду, что никаких других особых точек нет.
\end{problem}
\begin{problem}
    Найти решение гипергеометрического уравнения
    \begin{equation}
        \label{hypergeom}
        z(z-1)u'' + (-\gamma + (\alpha + \beta + 1)z)u' + \alpha \beta u = 0
    \end{equation}
    в виде ряда по $z$: $\sum_{k=0}^{\infty} c_k z^k$, причём $c_0 = 1$.
\end{problem}
\begin{problem}
    Доказать, что произвольное уравнение с тремя регулярными особыми точками
    сводится к гипергеометрическому, если перевести особые точки
    дробно--линейным преобразованием в $0,1, \infty$ и сделать после этого замену 
    $u = z^\mu (z-1)^\nu v$. \emph{Указание:} $\mu$ и $\nu$ следует выбрать таким образом,
    чтобы существовали регулярные решения в окрестности $0$ и $1$.
\end{problem}    
\begin{problem}
    Получить вырожденную гипергеометрическую функцию в виде ряда 
    и уравнение на неё, сделав в    
    \eqref{hypergeom} замену $z = z'/\beta$ и перейдя к пределу $\beta \to \infty$.
\end{problem}
\begin{problem}
    Выразить полиномы Лежандра через гипергеометрическую функцию. \emph{Указание}: использовать
    уравнение на полиномы Лежандра.
\end{problem}
\begin{problem}
    Свести радиальное уравнение Шрёдингера для атома водорода к уравнению на 
    вырожденную гипергеометрическую функцию
    и найти уровни энергии.
    \begin{equation}
        -\frac{\hbar^2}{2m}\frac{\partial^2 \psi}{\partial x^2} + 
            \left(-\frac{Ze^2}{r} + \frac{\hbar^2l(l+1)}{2mr^2}\right)\psi = E\psi
    \end{equation}
\end{problem}
\begin{problem}
    Методом Лапласа решить уравнение на вырожденную гипергеометрическую функцию.
\end{problem}
\end{document}
