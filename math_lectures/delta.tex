\documentclass{article}
\usepackage{amsmath}
\usepackage{graphicx}
\usepackage[utf8]{inputenc}
\usepackage[T1, T2A]{fontenc}
\usepackage[english,russian]{babel}

\newcommand{\bra}{\langle}
\newcommand{\ket}{\rangle}
\newcommand{\p}{\partial}

\title{Дельта--функция}
\author{Anikin Evgeny}

\begin{document}
\maketitle
\section{Дельта--функция}
Дельта--функция Дирака --- это ''функция``, удовлетворяющая следующим формальным 
свойствам:
\begin{equation}
    \delta(x) = 0 \quad \text{при} \quad x \ne 0
\end{equation}
\begin{equation}
    \int_{-a}^{a} \delta(x) \,dx = 1, \quad a > 0
\end{equation}
Конечно, она является не обычной функцией, а обобщённой. Определение и свойства обобщённых
функций можно посмотреть в ''Уравнениях математической физики`` Владимирова: это полезно,
но необязательно.

Дельта--функцию можно представлять как предел узкого и высокого ''горба`` с центром 
в начале координат. В некотором смысле можно записать
\begin{equation}
    \delta(x) = \lim_{a \to 0} \frac{1}{\pi} \frac{a}{x^2 + a^2}
\end{equation}
\begin{equation}
    \delta(x) = \lim_{a \to 0} \frac{1}{\sqrt{\pi}a} e^{-\frac{x^2}{a^2}}
\end{equation}

Важнейшее свойство $\delta$--функции выражается равенством
\begin{equation}
    \label{delta_int}
    \int f(x) \delta(x - x_0) \, dx = f(x_0)
\end{equation}
Оно очевидно из следующих соображений: в точках, отличных от $x_0$, дельта--функция равна
нулю. В точке $x_0$ же у дельта--функции наблюдается крайне резкий и высокий 
максимум. В этом случае можно в окрестности $x_0$ приближённо заменить $f(x)$ на константу,
что и приводит к \eqref{delta_int}
\end{document}
