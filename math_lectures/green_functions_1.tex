\documentclass{article}
\usepackage{amsmath}
\usepackage{bbm}
\usepackage{graphicx}
\usepackage[utf8]{inputenc}
\usepackage[T1, T2A]{fontenc}
\usepackage[english,russian]{babel}

\newcommand{\bra}{\langle}
\newcommand{\ket}{\rangle}
\newcommand{\p}{\partial}

\title{Функции Грина --- продолжение}
\author{Anikin Evgeny}

\begin{document}
\maketitle
\section{Функция Грина как обратный оператор}
Начнём с того, что любая функция $g(x,y)$ 
двух переменных задаёт линейный оператор $\hat{L}_g$, действующий 
на пространстве функций следующим образом:
\begin{equation}
    \hat{L}_g\phi(x) = \int  g(x,y) \phi(y) dy
\end{equation}
Несложно заметить здесь сходство с правилами перемножения матриц.

Функция $g(x,y)$ может быть (и часто является) не обычной, а обобщённой функцией. Например,
единичный оператор задаётся дельта--функцией:
\begin{equation}
    \int \delta(x-y) \phi(y) dy = \phi(x)
\end{equation}
Дифференциальные операторы задаются производными от дельта--функции. Например,
\begin{equation}
    \frac{d}{dx} \phi(x) = \int \delta'(x-y) \phi(y) dy
\end{equation}

Вооружившись этим знанием, нетрудно заметить, что уравнение на функцию Грина
\begin{equation}
    \hat{L}G(x,y) = \delta(x-y)
\end{equation}
есть в точности уравнение на оператор, обратный $\hat{L}$. В операторном виде
\begin{equation}
    \hat{L} \hat{G} = \mathbbm{1},
\end{equation}
или даже 
\begin{equation}
    \hat{G} = \hat{L}^{-1}
\end{equation}

\section{Некоторые замечания}
В предыдущем листке подразумевалось, что речь идёт о задаче эволюции во времени. 
Возможна также ситуация, когда в задаче поставлены граничные условия. А именно, будем решать
уравнение 
\begin{equation}
    \hat{L}\phi = f(x),
\end{equation}
где $x$ --- координата в $n$--мерном пространстве, функция $f$ задана в области
$M$, и есть какие--нибудь граничные условия на $\partial M$ 
(например, $\phi(x) = 0$ при $x \in \partial M$). 
\end{document}
