\documentclass{article}
\usepackage{amsmath}
\usepackage{graphicx}
\usepackage[utf8]{inputenc}
\usepackage[T1, T2A]{fontenc}
\usepackage[english,russian]{babel}

\newcommand{\bra}{\langle}
\newcommand{\ket}{\rangle}
\newcommand{\p}{\partial}

\title{Функции Грина}
\author{Anikin Evgeny}

\begin{document}
\maketitle
\section{Определение}
{\it Функцией Грина} $G(x,y)$ линейного дифференциального оператора $\hat{L}$,
действующего на функции $f(x)$, называется
решение уравнения
\begin{equation}
    \hat{L}G(x,y) = \delta(x - y)
\end{equation}
У однородного уравнения $\hat{L}\phi = 0$, вообще говоря, есть решения. В этом случае
функция Грина определена с точностью до решения однородного уравнения; тогда выделяют 
{\it запаздывающую} и {\it опережающую} функции Грина. 

{\it Запаздывающая} функция Грина определяется дополнительным условием $G(x,y) = 0$ при
$x < y$. Это эквивалентно тому, что в задаче Коши мы ставим начальные условия 
\begin{equation}
    \begin{gathered}
        G(x_0, y) = 0,\
        G'(x_0, y) = 0,\
        \dots,\
        G^{(n)}(x_0, y) = 0\
    \end{gathered}
\end{equation}
для какого--нибудь $x_0 < y$.

{\it Опережающая} функция Грина --- аналогично, только с условием $G(x,y) = 0$ при
$x > y$.

\section{Общее решение неоднородного линейного уравнения}
Неоднородное линейное уравнение --- это уравнение вида 
\begin{equation}
    \hat{L}\phi = f(x),
\end{equation}
где $f(x)$ --- произвольная функция $x$. Воспользуемся формальным представлением $f(x)$ 
в виде ''суммы`` дельта--функций:
\begin{equation}
    f(x) = \int \delta(x-x') f(x')\, dx'
\end{equation}
Отсюда следует, что решение уравнения --- 
\begin{equation}
    \phi(x) = \int_{-\infty}^{\infty} G(x - x') f(x') \, dx'
\end{equation}
Пусть нас интересует вынужденное решение, то есть движение изначально покоившейся системы
при включении взаимодействия. Тогда мы должны использовать 
запаздывающую функцию Грина. Так как при $x - x' < 0$ запаздывающая функция Грина
обращается в ноль, в интеграле можно заменить пределы интегрирования:
\begin{equation}
    \phi(x) = \int_{-\infty}^{x} G(x - x') f(x') \, dx'
\end{equation}
\section{Функция Грина частицы в вязкой среде}
Движение частицы в вязкой среде описывается уравнением 
\begin{equation}
    (m\frac{d}{dt} + \beta)v = F(t)
\end{equation}
Найдём функцию Грина оператора
\begin{equation}
    \hat{L} = m\frac{d}{dt} + \beta
\end{equation}
Это значит, что мы должны решить уравнение
\begin{equation}
    \label{eq_dissip}
    \left(m\frac{d}{dt} + \beta\right)G(t,t_0) = \delta(t - t_0)
\end{equation}
Ясно, что $G(t, t_0)$ является функцией только $t - t_0$, потому что 
оператор $\hat{L}$ не зависит от времени явно. Поэтому можно положить $t_0 = 0$, а вместо
$G(t,t_0)$ писать $G(t - t_0)$.

Задачу можно решить двумя способами. 
\subsection{Элементарный способ}
Будем искать запаздывающую функцию Грина. Тогда при $t < 0$ $G(t) = 0$. При $t > 0$ решение
уравнения известно с первого курса: 
\begin{equation}
    G(t) = C e^{-\frac{\beta t}{m}}
\end{equation}
Это решение содержит неизвестную константу $C$. Её нужно определить из того условия, 
чтобы в правой части уравнения \eqref{eq_dissip} действительно стояла дельта--функция.
Подставим в это уравнение $G(t) = Ce^{-\frac{\beta t}{m}} \theta(t)$ 
(здесь $\theta(t)$ --- тэта--функция, равная $1$ при $t > 0$ и $0$ при $t < 0$):
\begin{equation}
    \left(m\frac{d}{dt} + \beta\right)Ce^{-\frac{\beta t}{m}} \theta(t) = 
        m C e^{-\frac{\beta t}{m}} \delta(t) = mC\delta(t)
\end{equation}
(Мы воспользовались тем, что $\theta(t)' = \delta(t)$, а также 
$f(t)\delta(t) = f(0) \delta(t)$.) Получается, что $C = m^{-1}$.
\subsection{Фурье--разложение}
Будем искать решение в виде интеграла Фурье:
\begin{equation}
    G(t) = \int \frac{d\omega}{(2\pi)} G(\omega) e^{-i\omega t}
\end{equation}
Нам понадобится разложение дельта--функции:
\begin{equation}
    \delta(t) = \int \frac{d\omega}{(2\pi)} e^{-i\omega t}
\end{equation}
Подставляя всё это в уравнение \eqref{eq_dissip}, получим, что
\begin{equation}
    (-im\omega + \beta) G(\omega) = 1
\end{equation}
и, следовательно,
\begin{equation}
    G(t) = \int \frac{d\omega}{(2\pi)} \frac{e^{-i\omega t}}{\beta - im\omega}
\end{equation}
Этот интеграл можно взять c помощью вычетов. Для этого необходимо замкнуть
контур в комплексной плоскости так, чтобы интеграл по ''дополнительной`` части контура
стремился к нулю. Это обеспечивает экспонента в числителе: она мала (экспоненциально) либо 
при $\mathrm{Im}(\omega) > 0$, либо при $\mathrm{Im}(\omega) < 0$ в зависимости от знака
$t$.

Пусть $t < 0$. Тогда контур замыкается вверх, где подынтегральная функция не имеет полюсов.
Значит, 
\begin{equation}
    G(t) = 0 
\end{equation}

Пусть $t > 0$. Тогда контур замыкается вниз, где имеется полюс при 
$\omega = -\frac{i\beta}{m}$. 
Таким образом,
\begin{equation}
    G(t) = -2\pi i \left. \mathrm{res}\left(\frac{e^{-i\omega t}}{2\pi(\beta - im\omega)}\right)
                   \right|_{\omega = -\frac{i\beta}{m}}  = e^{-\frac{\beta t}{m}}
\end{equation}
Обратим внимание, что запаздывающая функция Грина здесь получилась автоматически.

\section{Функция Грина гармонического осциллятора}
Уравнение на функцию Грина имеет вид
\begin{equation}
    \ddot{x} + \omega_0^2 x = \delta(t)
\end{equation}
Обсудим здесь, как работает для гармонического осциллятора преобразование
Фурье. Дело в том, что фурье--компонента $G(\omega)$ оказывается сингулярной:
\begin{equation}
    x(\omega) = \frac{1}{\omega_0^2 - \omega^2}
\end{equation}
Для того, чтобы в интеграле
\begin{equation}
    x(t) = \int \frac{d\omega}{2\pi} \frac{e^{-i\omega t}}{\omega_0^2 - \omega^2}
\end{equation}
как--то справиться с расходимостями, нужно сместить контур интегрирования по $\omega$ 
в комплексную плоскость. Это можно сделать разными способами, и мы получим таким образом
разные функции Грина. 

Можно потребовать, чтобы получалась запаздывающая функция Грина. Тогда нетрудно понять,
что оба полюса нужно обходить сверху. Так же, как и раньше, замыкаем контур с разных сторон
в зависимости от знака $t$. В результате получается 
\begin{equation}
    x(t) = \frac{1}{\omega_0}\sin{\omega_0 t}\cdot\theta(t)
\end{equation}

\section{Функция Грина волнового уравнения}
Функция Грина волнового уравнения в трёхмерии определяется уравнением 
\begin{equation}
    \left[\frac{\partial^2}{\partial t^2} -  \Delta^2 \right]\phi = \delta(\vec{x},t)
\end{equation}
Преобразование Фурье в данном случае записывается в виде
\begin{equation}
    \phi(\vec{x},t) = \int \frac{d\omega}{2\pi} \frac{d^3 k}{(2\pi)^3} 
        e^{-i\omega t + i\vec{k}\vec{x}} \phi(\omega, \vec{k})
\end{equation}
Здесь следует обратить внимание на знаки в экспоненте. 

Дельта--функция --- 
\begin{equation}
    \delta(\vec{x},t) = \int \frac{d\omega}{2\pi} \frac{d^3 k}{(2\pi)^3} 
        e^{-i\omega t + i\vec{k}\vec{x}} 
\end{equation}
В результате получим функцию Грина в виде интеграла:
\begin{equation}
    G(x,t) = \int \frac{d\omega}{(2\pi)} \frac{d^3 k}{(2\pi)^3} 
            \frac{e^{-i\omega t + i\vec{k}\vec{x}}}{k^2 - \omega^2}
\end{equation}
Точно так же, как и в случае с осциллятором,
обход полюсов при интегрировании по $\omega$ нужно производить сверху, чтобы получить
запаздывающую функцию.

Удобно выполнить сначала интегрирование по $\omega$, а затем --- по $k$. 
Интеграл по $\omega$ совершенно такой же, как для гармонического осциллятора.  Для $t > 0$
получится
\begin{equation}
    G(x,t) = \int \frac{d^3 k}{(2\pi)^3} 
            \frac{1}{k} e^{i\vec{k}\vec{x}} \sin{kt}
\end{equation}
Чтобы взять интеграл по $k$, перейдём к полярным координатам. Выберем направление оси, от
которой отсчитыается угол $\theta$, вдоль вектора $\vec{x}$. Тогда интеграл переписывается
в виде
\begin{equation}
    G(x,t) = \int \frac{(2\pi) k^2\,dk\,d\cos{\theta}}{(2\pi)^3} 
            \frac{1}{k}e^{ikr \cos{\theta}}\sin{kt}
\end{equation}
После элементарных преобразований и интегрирования по $\cos{\theta}$ получается
\begin{equation}
    G(x,t) = -\frac{i}{(2\pi)^2 r} \int_0^{\infty} \sin{kt} (e^{ikr} - e^{-ikr}) \,dk
\end{equation}
Во втором слагаемом можно сделать замену $k \to -k$. После этого, несложно убедиться,
\begin{multline}
    G(x,t) = -\frac{i}{(2\pi)^2 r} \int_{-\infty}^{\infty} \sin{kt}\cdot e^{ikr} \,dk = \\
        -\frac{1}{8\pi^2 r} \int_{-\infty}^{\infty} (e^{ik(r+t)} - e^{ik(r-t)}) dk = \\
        -\frac{1}{4\pi r} (\delta(t + r) - \delta(t - r))
\end{multline}
Теперь вспомним, что $t > 0$ (из--за правила обхода полюсов) и $r > 0$. Значит, 
$\delta(t + r)$ можно просто выбросить. Таким образом, для всех $t$
\begin{equation}
    G(x,t) = \frac{1}{4\pi r} \delta(t - r)\theta(t)
\end{equation}
Это решение описывает расходящуюся сферическую бесконечно узкую волну. 

Его можно также переписать в релятивистски--инвариантном виде, используя свойства
дельта--функции.
\begin{equation}
    G(x,t) = -\frac{(t + r)}{4\pi r} \delta((t - r)(t + r)) \theta(t) = 
        \frac{1}{2\pi} \delta(t^2 - r^2)
\end{equation}
\end{document}
