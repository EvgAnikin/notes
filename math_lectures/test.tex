\documentclass{article}
\usepackage{amsmath}
\usepackage{bbm}
\usepackage{graphicx}
\usepackage{ntheorem}
\usepackage[utf8]{inputenc}
\usepackage[T1, T2A]{fontenc}
\usepackage[english,russian]{babel}

\newcommand{\bra}{\langle}
\newcommand{\ket}{\rangle}
\newcommand{\p}{\partial}

\theorembodyfont{\normalfont}
\newtheorem{problem}{Задача}

\title{Контрольная}
\author{Anikin Evgeny}

\begin{document}
    \begin{problem}
        Методом Лапласа решить уравнение модифицированное уравнение Бесселя 
        нулевого порядка:
        \begin{equation}
            u'' + \frac{1}{x}u' - u = 0
        \end{equation}
        Найти решение, конечное при $x = 0$. Найти асимптотику этого решения 
        при $x \to \infty$ (не методом перевала).
    \end{problem}
    \begin{problem}
        Получить уравнение на присоединённые функции Лежандра $P_{lm}(\cos{\theta})$.
        (Собственные функции оператора квадрата момента импульса имеют вид 
        $P_{lm}(\cos{\theta})e^{im\phi}$.)
        Свести его к гипергеометрическому уравнению и найти решение, регулярное при
        $\cos{\theta} = 1$.
    \end{problem}
    \begin{problem}
        Методом перевала найти асимптотику полиномов Эрмита, считая, что 
        $x^2, n \to \infty$, а $x^2/n$ остаётся конечным. Интегральная формула
        для полиномов Эрмита ---
        \begin{equation}
            H_n(x) = \frac{n!}{2\pi i} \oint \frac{dz}{z^{n+1}} e^{2xz - z^2}
        \end{equation}
        Рассмотреть два случая расположения перевальных точек. Какие из них 
        дают вклад?
        Вычислить асимптотику в том случае, когда обе точки важны.
    \end{problem}
\end{document}
